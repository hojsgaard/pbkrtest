\documentclass{article}

\special{html: 
  <link title="main" media="all" rel="stylesheet" href="http://people.math.aau.dk/~sorenh/css/main.css"
 type="text/css" />
}

\special{html: 
  <img style="width: 100px; height: 76px;" src="logo.jpg" alt="">
}

\def\R{\href{http://www.R-project.org}{\texttt{R}}}
\def\code#1{\texttt{#1}}

\def\pbkr{\href{http://cran.r-project.org/web/packages/pbkrtest/index.html}{pbkrtest}}
\def\lme{\href{http://cran.r-project.org/web/packages/lme4/index.html}{lme4}}

\section*{pbkrtest - Test in linear mixed effects models based on parametric
bootstrap approaches and Kenward-Roger modification of F-tests }
\label{sec:pack-graph-modell}

The \pbkr\ package is an \R\ package for tests in linear mixed effects models
based on parametric bootstrap approaches and Kenward-Roger
modification of F-tests

\begin{itemize}
\item The reference to pbkrtest is:
  \href{http://www.jstatsoft.org/v59/i09}{Halekoh, U., and H�jsgaard,
    S. (2014) A Kenward-Roger Approximation and Parametric Bootstrap
    Methods for Tests in Linear Mixed Models - the R Package
    pbkrtest. J. Stat. Soft. Vol. 59, Issue 9}. 
\item  Please do cite the above mentioned
  paper if you publish work where \pbkr\ has been used.

\item Vignettes:
  \href{http://cran.r-project.org/web/packages/pbkrtest/vignettes/pbkrtest-introduction.pdf}{On the usage of the pbkrtest package}
\end{itemize}

\section{Development versions of the packages}
\label{sec:devel-vers-pack}

Development versions of the packages may be available \href{./devel}{here}.

\section{Performance issues}

Calculation of the the adjusted degrees of freedom for the
Kenward-Roger approximation can be computationally demanding because it
requires inversion of an $N \times N$ matrix where $N$ is the number
of observations. 

Possible remedies for this:

\begin{itemize}
\item On linux (ubuntu) I have observed that changing the BLAS to ATLAS-BLAS
  generally provides a speed-up with a factor 3-6 (compared with R's
  default BLAS). That helps in some situations.

\item Parametric bootstrap is an alternative, and while also
  computationally intensive, parametric bootstrap can be parallelized
  (facilities exist in \pbkr).

\item Lastly, an alternative to Kenward-Roger is to implement a
  Satterthwaites approximation. It is on the todo-list.

\end{itemize}




\section{Reporting unexpected behaviours (bugs)}
\label{sec:report-unexp-behav}

When reporting unexpected behaviours, bugs etc.\ in the packages,
PLEASE supply: 
\begin{enumerate}
\item A reproducible example in terms of a short code
fragment.
\item The data. The preferred way of sending the data "mydata" is to copy and paste the result from running
\code{dput(mydata)}.
\item The result of running the \code{sessionInfo()}
function. 

\end{enumerate}


\section{FAQ (frequently asked questions)}
\label{sec:faq-frequently-asked}

\begin{description}
\item[Q:] Do these methods work for generalized linear mixed models ?

\item[A:] Parametric bootstrap is available for
  generalized linear mixed models.
  We are not aware of any developments for approximate
  F--tests in the spirit of Kenward-Roger for generalized linear
  models. 

\item[Q:] Are these models implemented for mixed models fitted with
  the nlme package?

\item[A:] No. We focus attention on the \lme\ package.

\end{description}


S�ren H�jsgaard
sorenh [at] math [dot] aau [dot] dk  



\special{html: 
<!-- Site Meter -->
<script type="text/javascript" src="http://s24.sitemeter.com/js/counter.js?site=s24sorenh">
</script>
<noscript>
<a href="http://s24.sitemeter.com/stats.asp?site=s24sorenh" target="_top">
<img src="http://s24.sitemeter.com/meter.asp?site=s24sorenh" alt="Site Meter" border="0"/></a>
</noscript>
<!-- Copyright (c)2009 Site Meter -->
}

\end{document}

