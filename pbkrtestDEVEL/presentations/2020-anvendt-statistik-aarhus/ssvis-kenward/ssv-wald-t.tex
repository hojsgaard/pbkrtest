\documentclass[]{article}
\usepackage{lmodern}
\usepackage{amssymb,amsmath}
\usepackage{ifxetex,ifluatex}
\usepackage{fixltx2e} % provides \textsubscript
\ifnum 0\ifxetex 1\fi\ifluatex 1\fi=0 % if pdftex
  \usepackage[T1]{fontenc}
  \usepackage[utf8]{inputenc}
\else % if luatex or xelatex
  \ifxetex
    \usepackage{mathspec}
  \else
    \usepackage{fontspec}
  \fi
  \defaultfontfeatures{Ligatures=TeX,Scale=MatchLowercase}
\fi
% use upquote if available, for straight quotes in verbatim environments
\IfFileExists{upquote.sty}{\usepackage{upquote}}{}
% use microtype if available
\IfFileExists{microtype.sty}{%
\usepackage{microtype}
\UseMicrotypeSet[protrusion]{basicmath} % disable protrusion for tt fonts
}{}
\usepackage[margin=1in]{geometry}
\usepackage{hyperref}
\hypersetup{unicode=true,
            pdftitle={Hvad skal vi med t-testet},
            pdfborder={0 0 0},
            breaklinks=true}
\urlstyle{same}  % don't use monospace font for urls
\usepackage{color}
\usepackage{fancyvrb}
\newcommand{\VerbBar}{|}
\newcommand{\VERB}{\Verb[commandchars=\\\{\}]}
\DefineVerbatimEnvironment{Highlighting}{Verbatim}{commandchars=\\\{\}}
% Add ',fontsize=\small' for more characters per line
\usepackage{framed}
\definecolor{shadecolor}{RGB}{248,248,248}
\newenvironment{Shaded}{\begin{snugshade}}{\end{snugshade}}
\newcommand{\KeywordTok}[1]{\textcolor[rgb]{0.13,0.29,0.53}{\textbf{#1}}}
\newcommand{\DataTypeTok}[1]{\textcolor[rgb]{0.13,0.29,0.53}{#1}}
\newcommand{\DecValTok}[1]{\textcolor[rgb]{0.00,0.00,0.81}{#1}}
\newcommand{\BaseNTok}[1]{\textcolor[rgb]{0.00,0.00,0.81}{#1}}
\newcommand{\FloatTok}[1]{\textcolor[rgb]{0.00,0.00,0.81}{#1}}
\newcommand{\ConstantTok}[1]{\textcolor[rgb]{0.00,0.00,0.00}{#1}}
\newcommand{\CharTok}[1]{\textcolor[rgb]{0.31,0.60,0.02}{#1}}
\newcommand{\SpecialCharTok}[1]{\textcolor[rgb]{0.00,0.00,0.00}{#1}}
\newcommand{\StringTok}[1]{\textcolor[rgb]{0.31,0.60,0.02}{#1}}
\newcommand{\VerbatimStringTok}[1]{\textcolor[rgb]{0.31,0.60,0.02}{#1}}
\newcommand{\SpecialStringTok}[1]{\textcolor[rgb]{0.31,0.60,0.02}{#1}}
\newcommand{\ImportTok}[1]{#1}
\newcommand{\CommentTok}[1]{\textcolor[rgb]{0.56,0.35,0.01}{\textit{#1}}}
\newcommand{\DocumentationTok}[1]{\textcolor[rgb]{0.56,0.35,0.01}{\textbf{\textit{#1}}}}
\newcommand{\AnnotationTok}[1]{\textcolor[rgb]{0.56,0.35,0.01}{\textbf{\textit{#1}}}}
\newcommand{\CommentVarTok}[1]{\textcolor[rgb]{0.56,0.35,0.01}{\textbf{\textit{#1}}}}
\newcommand{\OtherTok}[1]{\textcolor[rgb]{0.56,0.35,0.01}{#1}}
\newcommand{\FunctionTok}[1]{\textcolor[rgb]{0.00,0.00,0.00}{#1}}
\newcommand{\VariableTok}[1]{\textcolor[rgb]{0.00,0.00,0.00}{#1}}
\newcommand{\ControlFlowTok}[1]{\textcolor[rgb]{0.13,0.29,0.53}{\textbf{#1}}}
\newcommand{\OperatorTok}[1]{\textcolor[rgb]{0.81,0.36,0.00}{\textbf{#1}}}
\newcommand{\BuiltInTok}[1]{#1}
\newcommand{\ExtensionTok}[1]{#1}
\newcommand{\PreprocessorTok}[1]{\textcolor[rgb]{0.56,0.35,0.01}{\textit{#1}}}
\newcommand{\AttributeTok}[1]{\textcolor[rgb]{0.77,0.63,0.00}{#1}}
\newcommand{\RegionMarkerTok}[1]{#1}
\newcommand{\InformationTok}[1]{\textcolor[rgb]{0.56,0.35,0.01}{\textbf{\textit{#1}}}}
\newcommand{\WarningTok}[1]{\textcolor[rgb]{0.56,0.35,0.01}{\textbf{\textit{#1}}}}
\newcommand{\AlertTok}[1]{\textcolor[rgb]{0.94,0.16,0.16}{#1}}
\newcommand{\ErrorTok}[1]{\textcolor[rgb]{0.64,0.00,0.00}{\textbf{#1}}}
\newcommand{\NormalTok}[1]{#1}
\usepackage{graphicx,grffile}
\makeatletter
\def\maxwidth{\ifdim\Gin@nat@width>\linewidth\linewidth\else\Gin@nat@width\fi}
\def\maxheight{\ifdim\Gin@nat@height>\textheight\textheight\else\Gin@nat@height\fi}
\makeatother
% Scale images if necessary, so that they will not overflow the page
% margins by default, and it is still possible to overwrite the defaults
% using explicit options in \includegraphics[width, height, ...]{}
\setkeys{Gin}{width=\maxwidth,height=\maxheight,keepaspectratio}
\IfFileExists{parskip.sty}{%
\usepackage{parskip}
}{% else
\setlength{\parindent}{0pt}
\setlength{\parskip}{6pt plus 2pt minus 1pt}
}
\setlength{\emergencystretch}{3em}  % prevent overfull lines
\providecommand{\tightlist}{%
  \setlength{\itemsep}{0pt}\setlength{\parskip}{0pt}}
\setcounter{secnumdepth}{0}
% Redefines (sub)paragraphs to behave more like sections
\ifx\paragraph\undefined\else
\let\oldparagraph\paragraph
\renewcommand{\paragraph}[1]{\oldparagraph{#1}\mbox{}}
\fi
\ifx\subparagraph\undefined\else
\let\oldsubparagraph\subparagraph
\renewcommand{\subparagraph}[1]{\oldsubparagraph{#1}\mbox{}}
\fi

%%% Use protect on footnotes to avoid problems with footnotes in titles
\let\rmarkdownfootnote\footnote%
\def\footnote{\protect\rmarkdownfootnote}

%%% Change title format to be more compact
\usepackage{titling}

% Create subtitle command for use in maketitle
\newcommand{\subtitle}[1]{
  \posttitle{
    \begin{center}\large#1\end{center}
    }
}

\setlength{\droptitle}{-2em}

  \title{Hvad skal vi med t-testet}
    \pretitle{\vspace{\droptitle}\centering\huge}
  \posttitle{\par}
    \author{Søren Højsgaard\footnote{University of Aalborg, Denmark}}
    \preauthor{\centering\large\emph}
  \postauthor{\par}
      \predate{\centering\large\emph}
  \postdate{\par}
    \date{June 12, 2018}

%% preamble

%\usepackage{bbm,bm}

\usepackage{fancyvrb}
\def\bm{ }
\newcommand{\lmer}{\texttt{lmer()}}
\newcommand{\betab}{\bm{\beta}}
\newcommand{\phib}{\bm{\phi}}
\newcommand{\phiAhat}{ {\hat \Phi}_A}
\newcommand{\Lb}{L}
\newcommand{\ssb}{{\hat \sigma}}
\newcommand{\sigmab}{{\sigma}}
\newcommand{\Sigmab}{{\Sigma}}
\newcommand{\transp}{^{\top}}
\newcommand{\inv}{^{-1}}
\def\pkg#1{\texttt{#1}}

\newcommand{\var}{\mathbb{V}ar}
\newcommand{\cov}{\mathbb{C}ov}
\newcommand{\EE}{\mathbb{E}}

\usepackage{etoolbox} 
\makeatletter 
\preto{\@verbatim}{\topsep=0pt \partopsep=0pt } 
\makeatother

\newcommand{\ssss}{\vspace{5mm}}

\begin{document}
\maketitle

{
\setcounter{tocdepth}{3}
\tableofcontents
}
\subsection{Hvad skal vi med t-testet?}\label{hvad-skal-vi-med-t-testet}

Betragt en lineær regressionsmodel \[
 y_i = \alpha + \beta x_i + e_i
\]

Variansen på estimatet \(\hat\beta\) er \(\sigma^2_\beta=\sigma^2 c\),
hvor \(\sigma^2\) er residualvariansen, dvs. variansen på \(y_i\)'erne
(eller på \(e_i\)'erne om man vil, det er det samme) og \(c\) er en
konstant hvis værdi er let at regne ud, men vi skal ikke gøre dette her.

\subsection{\texorpdfstring{\(t\)--testet og
\(z\)--testet}{t--testet og z--testet}}\label{ttestet-og-ztestet}

I introducerende statistikkurser lærer man, at når man skal teste
hypotesen at \(\beta=\beta_0\) (hvor \(\beta_0\) er et givet tal) så
skal man skelne mellem situationerne hvor \(\sigma^2\) er kendt og
ukendt (i praksis er \(\sigma^2\) næsten aldrig kendt).

Hvis \(\sigma^2\) er kendt, så ser vi på \emph{teststørrelsen} \[
 z = \frac{\hat\beta-\beta_0}{\sqrt{\sigma^2_\beta}} = \frac{\hat\beta-\beta_0}{\sigma_\beta}
\] Dvs. \(z\) ``måler'' hvor mange standardafvigelser estimatet
\(\hat\beta\) ligger fra \(\beta_0\). Numerisk store værdier af \(z\)
bevirker, at man tvivler på hypotesen.

så kan man se på en \(t\)--test størrelse

\[
t = \frac{\hat\beta-\beta_0}{\sqrt{\hat\sigma^2_\beta}}
\]

Dvs. \(t\) ``måler'' hvor mange standardafvigelser estimatet
\(\hat\beta\) ligger fra \(\beta_0\). Numerisk store værdier af \(t\)
bevirker, at man tvivler på hypotesen.

Hvis hypotesen er sand, så skal \(t\) vurderes i en \(t\)--fordeling med
\(N-1\) frihedsgrader, hvor \(N\) er antal observationer. Numerisk store
værdier af \(t\) er får os til at tvivle på hypotesen.

Hvis \(\sigma^2\) er kendt (det er \(\sigma^2\) næsten aldrig i praksis,
men lad os lige lade som om) så er der ingen grund til at bruge et
estimat for \(\sigma^2\) ovenfor. I så fald bliver variansen på
\(\hat\beta\) givet ved \(\sigma^2_\beta=\sigma^2 c\). I dette tilfælde
bliver \(t\)--test størrelsens pendant

Hvis hypotesen er sand, så skal \(u\) vurderes i en
\(N(0,1)\)--fordeling. Numerisk store værdier af \(t\) er får os til at
tvivle på hypotesen.

I et introducerende statistikkursus vil man ofte lære, at hvis antallet
af frihedsgrader \(f\) er stort så vil en \(t_f\)-fordeling (en \(t\)
fordeling med \(f\) frihedsgrader) ligne en standard normal fordeling
(en \(N(0,1)\) fordeling) så meget, at man simpelthen kan vurdere \(t\)
i en \(N(0,1)\) fordeling. Dette svarer præcist til at lade som om at
\(\hat\sigma^2_\beta\) er den sand værdi for spredningen på
\(\hat\beta\).

Hvis man vurderer \(t\) i en \(N(0,1)\) fordeling når der er ganske få
frihedsgrader, så kan resultaterne derimod blive forkerte - meget
forkert endda.

For at illustrere dette laver vi følgende tankeeksperiement: Vi skal
sammenligne en behandling for en given sygdom med en placebo, så vi har
to grupper med \(M\) patienter i hver. Dette kan håndteres med en lineær
regressionsmodel: Lad \(x\) være en variabel der er \(1\) for de
patienter, der har fået behandling og \(0\) for dem, der har fået
placebo. Så vil middelværdien for patienter med placebo være \(\alpha\)
mens middelværdien for patienter med behandlingen være \(\alpha+\beta\)
så behandlingseffekten vil være \(\beta\). Da vi laver eksperimentet
selv, så står det os frit for at vælge \(\beta\) og vi sætter
\(\beta=0\), således der ikke er en behandlingseffekt. Hvis vi tester
hypotesen \(\beta=0\) på niveau \(5\%\) så vil vi med \(5\%\)
sandsynlighed forkaste hypotesen (der jo er sand, for sådan er
eksperimentet lavet). Lad os gentage eksperimentet \(1000\) gange. Så
ville vi skulle forkaste hypotesen omkring \(50\) gange - ellers er der
noget helt galt.

Hvis \(M\) (antal patienter per gruppe) er lille og vi laver et
normalfordelingstest, så vil vi ikke få forkastet hypotesen i \(5\%\) af
tilfældene men måske i \(10\%\) af tilfældene.

\scriptsize

\begin{Shaded}
\begin{Highlighting}[]
\NormalTok{M <-}\StringTok{ }\DecValTok{3}
\NormalTok{beta <-}\DecValTok{0}
\NormalTok{mu <-}\StringTok{ }\KeywordTok{c}\NormalTok{(}\KeywordTok{rep}\NormalTok{(}\DecValTok{0}\NormalTok{, M), }\KeywordTok{rep}\NormalTok{(beta, M))}
\NormalTok{grp <-}\StringTok{ }\KeywordTok{factor}\NormalTok{(}\KeywordTok{c}\NormalTok{(}\KeywordTok{rep}\NormalTok{(}\StringTok{"placebo"}\NormalTok{, M), }\KeywordTok{rep}\NormalTok{(}\StringTok{"behandling"}\NormalTok{, M)))}

\NormalTok{y <-}\StringTok{ }\KeywordTok{rnorm}\NormalTok{(}\DecValTok{2} \OperatorTok{*}\StringTok{ }\NormalTok{M, }\DataTypeTok{mean=}\NormalTok{mu)}
\NormalTok{tb <-}\StringTok{ }\KeywordTok{summary}\NormalTok{(}\KeywordTok{lm}\NormalTok{(y }\OperatorTok{~}\StringTok{ }\NormalTok{grp))}\OperatorTok{$}\NormalTok{coef}
\NormalTok{tb <-}\StringTok{ }\KeywordTok{as.data.frame}\NormalTok{(tb)}
\NormalTok{prt <-}\DecValTok{2} \OperatorTok{*}\StringTok{ }\NormalTok{(}\DecValTok{1} \OperatorTok{-}\StringTok{ }\KeywordTok{pt}\NormalTok{(}\KeywordTok{abs}\NormalTok{(tb}\OperatorTok{$}\StringTok{`}\DataTypeTok{t value}\StringTok{`}\NormalTok{), }\DataTypeTok{df=}\DecValTok{2} \OperatorTok{*}\StringTok{ }\NormalTok{M }\OperatorTok{-}\StringTok{ }\DecValTok{2}\NormalTok{))}
\NormalTok{prn <-}\StringTok{ }\DecValTok{2} \OperatorTok{*}\StringTok{ }\NormalTok{(}\DecValTok{1} \OperatorTok{-}\StringTok{ }\KeywordTok{pnorm}\NormalTok{(}\KeywordTok{abs}\NormalTok{(tb}\OperatorTok{$}\StringTok{`}\DataTypeTok{t value}\StringTok{`}\NormalTok{)))}
\KeywordTok{c}\NormalTok{(prt[}\DecValTok{2}\NormalTok{], prn[}\DecValTok{2}\NormalTok{])}
\end{Highlighting}
\end{Shaded}

\begin{verbatim}
## [1] 0.1219 0.0503
\end{verbatim}

\begin{Shaded}
\begin{Highlighting}[]
\NormalTok{do_sim <-}\StringTok{ }\ControlFlowTok{function}\NormalTok{()\{}
\NormalTok{  y <-}\StringTok{ }\KeywordTok{rnorm}\NormalTok{(}\DecValTok{2} \OperatorTok{*}\StringTok{ }\NormalTok{M, }\DataTypeTok{mean=}\NormalTok{mu)}
\NormalTok{  tb <-}\StringTok{ }\KeywordTok{summary}\NormalTok{(}\KeywordTok{lm}\NormalTok{(y }\OperatorTok{~}\StringTok{ }\NormalTok{grp))}\OperatorTok{$}\NormalTok{coef}
\NormalTok{  tb <-}\StringTok{ }\KeywordTok{as.data.frame}\NormalTok{(tb)}
\NormalTok{  prt <-}\DecValTok{2} \OperatorTok{*}\StringTok{ }\NormalTok{(}\DecValTok{1} \OperatorTok{-}\StringTok{ }\KeywordTok{pt}\NormalTok{(}\KeywordTok{abs}\NormalTok{(tb}\OperatorTok{$}\StringTok{`}\DataTypeTok{t value}\StringTok{`}\NormalTok{), }\DataTypeTok{df=}\DecValTok{2} \OperatorTok{*}\StringTok{ }\NormalTok{M }\OperatorTok{-}\StringTok{ }\DecValTok{2}\NormalTok{))}
\NormalTok{  prn <-}\StringTok{ }\DecValTok{2} \OperatorTok{*}\StringTok{ }\NormalTok{(}\DecValTok{1} \OperatorTok{-}\StringTok{ }\KeywordTok{pnorm}\NormalTok{(}\KeywordTok{abs}\NormalTok{(tb}\OperatorTok{$}\StringTok{`}\DataTypeTok{t value}\StringTok{`}\NormalTok{)))}
  \KeywordTok{c}\NormalTok{(prt[}\DecValTok{2}\NormalTok{], prn[}\DecValTok{2}\NormalTok{])}
\NormalTok{\}}

\KeywordTok{do_sim}\NormalTok{()}
\end{Highlighting}
\end{Shaded}

\begin{verbatim}
## [1] 0.6531 0.6278
\end{verbatim}

\begin{Shaded}
\begin{Highlighting}[]
\NormalTok{Nsim <-}\StringTok{ }\DecValTok{1000}
\NormalTok{sim <-}\StringTok{ }\KeywordTok{replicate}\NormalTok{(Nsim, }\KeywordTok{do_sim}\NormalTok{())}

\KeywordTok{sum}\NormalTok{(sim[}\DecValTok{1}\NormalTok{,] }\OperatorTok{<=}\StringTok{ }\FloatTok{0.05}\NormalTok{) }\OperatorTok{/}\StringTok{ }\NormalTok{Nsim}
\end{Highlighting}
\end{Shaded}

\begin{verbatim}
## [1] 0.055
\end{verbatim}

\begin{Shaded}
\begin{Highlighting}[]
\KeywordTok{sum}\NormalTok{(sim[}\DecValTok{2}\NormalTok{,] }\OperatorTok{<=}\StringTok{ }\FloatTok{0.05}\NormalTok{) }\OperatorTok{/}\StringTok{ }\NormalTok{Nsim}
\end{Highlighting}
\end{Shaded}

\begin{verbatim}
## [1] 0.129
\end{verbatim}

\normalsize

\subsection{F-test og Wald-test}\label{f-test-og-wald-test}

Et alternativt men ækvivalent test (giver samme resultat) er et
\(F\)--test: \[
F = t^2 =\frac{(\hat\beta-\beta_0)^2}{\hat\sigma^2_\beta}
\] er under hypotesen \(F\)-fordelt med een tællerfrihedsgrad og \(N-1\)
nævnerfrihedsgrader.

\[
  W = u^2 =\frac{(\hat\beta-\beta_0)^2}{\sigma^2_\beta}
\] Dette kaldes et Wald--test. Forskellen mellem et Wald-test og et
\(F\)-test er altså, at i det første er variansen kendt; i \(F\)-test
tages der højde for at variansen er estimeret fra data og den usikkerhed
i estimatet, der følger dermed.

--\textgreater{} --\textgreater{} --\textgreater{} --\textgreater{}
--\textgreater{}

--\textgreater{}

--\textgreater{}

--\textgreater{} --\textgreater{} --\textgreater{} --\textgreater{}

--\textgreater{} --\textgreater{} --\textgreater{} --\textgreater{}
--\textgreater{} --\textgreater{}

--\textgreater{}

--\textgreater{}

--\textgreater{} --\textgreater{} --\textgreater{} --\textgreater{}

--\textgreater{} --\textgreater{}

--\textgreater{} --\textgreater{} --\textgreater{} --\textgreater{}
--\textgreater{} --\textgreater{}

--\textgreater{}

--\textgreater{} --\textgreater{} --\textgreater{}

--\textgreater{}

--\textgreater{} --\textgreater{}

--\textgreater{}

--\textgreater{} --\textgreater{} --\textgreater{} --\textgreater{}

--\textgreater{} --\textgreater{} --\textgreater{} --\textgreater{}
--\textgreater{} --\textgreater{} --\textgreater{} --\textgreater{}
--\textgreater{} --\textgreater{}

--\textgreater{} --\textgreater{} --\textgreater{} --\textgreater{}
--\textgreater{} --\textgreater{} --\textgreater{} --\textgreater{}
--\textgreater{} --\textgreater{}


\end{document}
