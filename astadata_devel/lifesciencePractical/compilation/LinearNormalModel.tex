\documentclass[12pt]{article}

\usepackage{a4wide,hyperref}
\usepackage[T1]{fontenc}
\usepackage{url,a4}
\usepackage{boxedminipage,color,xcolor}
\usepackage{shortvrb}
\usepackage{framed}
\usepackage{comment}

\usepackage[inline,nomargin,draft]{fixme}
\usepackage{listings}
\usepackage{bm}
\usepackage{framed}
\usepackage{comment}
\usepackage{amsmath}

\RequirePackage{color,fancyvrb,amsmath,amsfonts}

\MakeShortVerb�

%%
%% Easy matrices:
%%
\newcommand{\matrxr}[2][rrrrrrrrrrrrrrrrrrrrrrrrrrrrrrrrrrrrr]
{\left[
    \begin{array}{#1}
      #2 \\
    \end{array}
  \right]}
% Usage: $\matrxr{-1&-2\\3&4}$

\newcommand{\matrxc}[2][cccccccccccccccccccccccccccccccccccc]
{\left[
    \begin{array}{#1}
      #2 \\
    \end{array}
  \right]}
% Usage: $\matrxl{-1&-2\\3&4}$

\newcommand{\matrxl}[2][lllllllllllllllllllllllllllllllllllll]
{\left[
    \begin{array}{#1}
      #2 \\
    \end{array}
  \right]}
% Usage: $\matrxl{-1&-2\\3&4}$


\def\R{\texttt{R}}
\def\code#1{\texttt{#1}}
\def\ok{\texttt{OK}}
\def\fix{\texttt{FIX}}

\def\red#1{{\bf \textcolor{red}{#1}}}
\def\blue#1{{\bf \textcolor{blue}{#1}}}
\def\purple#1{{\bf \textcolor{purple}{#1}}}

\def\com#1{{\em{\color{red} #1}}}
\def\comi#1{{\bf\sc{\color{darkblue} #1}}\index{#1}}
\def\comic#1{{\bf\sc{\color{darkblue} #1}}\index{#1}}

\def\simn{\sim\mathcal N}
\def\simx{\sim\chi^2}
\def\xx{\chi^2}
\def\norm{\mathcal N}


\DeclareMathOperator{\RR}{\mathbb{R}}
\DeclareMathOperator{\EE}{\mathbb{E}}
\DeclareMathOperator{\var}{\mathbb{V}ar}
\DeclareMathOperator{\cov}{\mathbb{C}ov}
%\DeclareMathOperator{\norm}{N}
\DeclareMathOperator{\spanx}{span}
\DeclareMathOperator{\corr}{Corr}
\DeclareMathOperator{\deter}{det}
\DeclareMathOperator{\trace}{tr}
\DeclareMathOperator{\diag}{diag}
\DeclareMathOperator{\logit}{logit}
\DeclareMathOperator{\odds}{odds}

\def\vnorm#1{\left|\left|#1\right|\right|} 
\def\innerp#1{\left\langle #1 \right\rangle}
\def\inv{^{-1}}
\def\transp{^{\top}}
\definecolor{darkblue}{rgb}{0,0,.5}

%%UHH commands added by UHH
%%some new enviroments
\newtheorem{theorem}{Theorem}[section]
\newenvironment{example}{\medskip\noindent\refstepcounter{theorem}{\bf
Example \thetheorem\ }}{\hfill $\Box$ }
\newenvironment{theoreme}{\medskip\noindent\refstepcounter{theorem}{\bf
Theorem \thetheorem\ }}{}

\definecolor{shadecolor}{gray}{0.91}
\definecolor{howR}{gray}{0.91}
%\specialcomment{howR}{\begin{shaded}\small R code: }{\end{shaded}}
%\specialcomment{howR}{\begin{shaded}\vspace{-4mm}\small}{\vspace{-3mm}\end{shaded}}
%\specialcomment{sblock}{\begin{shaded}\vspace{-4mm}\small}{\vspace{-3mm}\end{shaded}}

\specialcomment{solution}
{\begin{shaded} Solution: }{\end{shaded}}

\specialcomment{topics}
{\begin{shaded} TOPICS: }{\end{shaded}}



\newenvironment{howR}{}{}

\newenvironment{examplea}
                {\begin{example}  \em \openup-1pt }
                {\hfill $\Box$  %\vspace{5mm}
                \end{example}}



%%\excludecomment{topics}
%%\excludecomment{solution}

\usepackage{Sweave}
\begin{document}


\RecustomVerbatimEnvironment{Sinput}{Verbatim}%
    {fontsize=\scriptsize,frame=single,framerule=1pt,
      rulecolor=\color{red},
      fillcolor=\color{yellow}
    }
\RecustomVerbatimEnvironment{Soutput}{Verbatim}%
    {fontsize=\scriptsize, frame=single,framerule=0.1pt}




% \renewenvironment{Schunk}{\linespread{.85}\scriptsize}{}
    
%% Efter preamble
% \definecolor{myGray}{rgb}{0.95,0.95,0.95}

% \makeatletter
% \renewenvironment{Schunk}{
%   \begin{lrbox}{\@tempboxa}
%     \begin{boxedminipage}
%       {\columnwidth}\scriptsize}
%     {\end{boxedminipage}
%   \end{lrbox}%
%   \colorbox{myGray}{\usebox{\@tempboxa}}
% }
% \makeatother


\parindent0pt
\parskip5pt


{Miniproject: Four small projects}

{Ulrich Halekoh and S�ren H�jsgaard \hfill Created: \today}

\hrule


\setkeys{Gin}{width=4in,height=3in}

\section{Diets of Rats}

\begin{topics}
  Linear model with quadratic terms in $x$; calculate $p$--value by
  hand. 
\end{topics}

The growth rate, $y$, of rats was studied for different amounts of
supplement to their diet, $x$.  Data for this exercise can be found in
the data frame 'dietsup' of the course package \verb+LiSciData+.


\begin{enumerate}
\item
Read in the data and plot the rate of growth as a function of the amount of supplement.

\begin{Schunk}
\begin{Sinput}
 data(dietsup, package="LiSciData")
 head(dietsup)
\end{Sinput}
\begin{Soutput}
   x  y
1 10 73
2 10 78
3 15 85
4 20 90
5 20 91
6 25 87
\end{Soutput}
\end{Schunk}

\item
Based on the plot, a model where the growth rate is related to the amount of supplement
through a quadratic equation could be reasonable.
Formulate  and fit such a model with the \verb+glm+-function.
%\[
%E(y_i)=\mu_i= \mu + \alpha x_i + \beta x_i^2, \quad y_i \sim N(\mu_i,\sigma^2).
%\]

What are the estimates of the parameters in this model?

\item
Add the fitted values to your plot of data.




\item
Consider the regression model without the quadratic term.
Write down this model.
%\[
%E(y_i)=\mu_i= \mu + \alpha x_i, \quad y_i \sim N(\mu_i,\sigma^2).
%\]
Fit the model  and plot the residuals against the predicted
values for model checking.

\item
Test whether there is a need for a quadratic term in the model using the \verb+anova+-function.

\item
The test statistic you just found is calculated in the following way.
In the column \verb+Deviance+ of the output of \verb+anova+
one finds actually the {\bf difference}
in the residual deviances $D_0-D_1=$
686.4-45.2=641.2
Hence the F-test statistic is
\[
\frac{ (D_0-D_1)/ (7-8)} {D_1/7}= \frac {641.2 /(8-7)}
{45.2/7}= 99.3
\]

You can calculate the p-value 'by hand'. You calculate the probability that
an F-distributed random variable with degrees of freedom 1 and 7 is larger
than 99.3:
\begin{Schunk}
\begin{Sinput}
 pf(99.3,1,7,lower.tail=FALSE)
\end{Sinput}
\begin{Soutput}
[1] 2.189525e-05
\end{Soutput}
\end{Schunk}
A histogram of 1000 observations from the F-distribution with degrees of 
freedom 1 and 7 showing
the 95 percent quantile of this F-distribution can be created as follows.
\\
Generating random number from the F-distribution with 1 and 7 degrees of freedom
\begin{Schunk}
\begin{Sinput}
 set.seed(998)
 y  <- rf(n=1000,1,7)
\end{Sinput}
\end{Schunk}
The $95\%$ quantile of the distribution
\begin{Schunk}
\begin{Sinput}
 q95 <- qf(0.95,1,7)
\end{Sinput}
\end{Schunk}
and a histogram:
\begin{Schunk}
\begin{Sinput}
 hist(y,probability=TRUE)
 lines(density(y))
 abline(v=q95)
\end{Sinput}
\end{Schunk}
\begin{enumerate}
\item
How large is the area to the right of the vertical line?
\item
On which side from the vertical line in the histogram must  the F-test statistic lie, such that
a F-test is significant? 
\end{enumerate}
\end{enumerate}

\section{Gestational Age}
\begin{topics}
  Linear normal model; two regression lines
\end{topics}


The data are from a study, where for boys and girls their
gestational(prenatal) weights were recorded. In the data set
'gestation' of the course package \verb+LiSciData+ the data are given
in 'long format', e.g. each row consists of measurements of a boy and
a girl.  We first reshape the data, such that each row contains just
the measurements for one boy or girl.  We add a variable \verb+sex+,
and make it a factor, to indicate the sex of the individual.

\begin{enumerate}
\item
Reading the data
\begin{Schunk}
\begin{Sinput}
 data(gestation, package="LiSciData")
 gestation<- reshape(gestation,direction='long',
              varying=list(age=c('boys.age','girls.age'),
              weight=c('boys.weight','girls.weight')),
              v.names=c('age','weight'),
              timevar='sex',times=c('boy','girl'))
 gestation$sex<-factor(gestation$sex)
 head(gestation)
\end{Sinput}
\begin{Soutput}
      sex age weight id
1.boy boy  40   2968  1
2.boy boy  38   2795  2
3.boy boy  40   3163  3
4.boy boy  35   2925  4
5.boy boy  36   2625  5
6.boy boy  37   2847  6
\end{Soutput}
\end{Schunk}


\item
First, plot the weight against gestational age, differentiating between boys and girls.
Then add a smooth line for both sexes.(You can use the \verb+lowess+-function.)

\item
Formulate two models.

M1:
A model with different intercepts but common slope for regressions
of weight on the gestational age.

M2:
A model that additionally differentiates the two slopes for the sexes.

%\[
%M_1:  \quad E(y_i)=\mu_i=\mu + \alpha_k + \beta x_i,
%\quad i=1,\dots,N, \quad k=\textrm{boy,girl}
%\]
%
%\[
%M_2: \quad  E(y_i)=\mu_i=\mu + \alpha_k + \beta x_i + \beta_k x_i,
%\quad i=1,\dots,N, \quad  k=\textrm{boy,girl}
%\]

Fit these models using the \verb+glm+-function.

\item
In model M1 find and interpret the estimates of the model parameters.
Use  the \verb+esticon+-function of the package \verb+doBy+ to
find the estimate of the intercept for girls.

Here we note that a simpler way to obtain this intercept is to declare
the model M1 by
\begin{Schunk}
\begin{Sinput}
 m1.alter <- glm(weight ~ sex + age - 1, data = gestation)
 coef(summary(m1.alter))[,c(1,2)]
\end{Sinput}
\end{Schunk}

Also find and interpret the estimates of the model parameters in model M2.

\item
Use the function \verb+predict+ to calculate the fitted values of M1.
Then plot these fitted lines together with the observations.

\item
In the model M1 find confidence intervals for the parameters.

\item
We want to test whether it is necessary to have a slope for each sex.
Use the function \verb+anova+ to compare the two models M1 and M2.

\item
Use the function \verb+drop1+ on \verb+m1+. What does the result tell you?

\item
Plots for model checking can be obtained.
\begin{Schunk}
\begin{Sinput}
 m1 <- glm(weight~sex+age-1,data=gestation)
 par(mfrow=c(2,2))
 plot(m1,pch=c(16,1)[gestation$sex],cex=2)
\end{Sinput}
\end{Schunk}
Does the model fit data?
\end{enumerate}



\section{Sexism Score}
\begin{topics}
  Linear normal model; two--way anova; LSMEANS
\end{topics}

A study was conducted to compare the sexist attitudes of students
     at various types of colleges in the US. The colleges-types are:
     mixed (gender) college with at least 75\% male students, mixed
     college with less than 75\% male students, and single sex college.
     For each gender, random samples of each 10 undergraduate students
     were selected from each of the three types of colleges. Each
     student filled in a questionnaire, from which a score for 'degree
     of sexism'-defined as the extent to which a student considered
     males and females to have different life roles-was determined





\begin{enumerate}
\item
Reading the data
\begin{Schunk}
\begin{Sinput}
 data(sexism,package='LiSciData')
 sexism <- transform(sexism, score = sexism)
 sexism <- transform(sexism, type = factor(type))
 sexism <- transform(sexism, gender.type = interaction(gender,type))
 head(sexism)
\end{Sinput}
\begin{Soutput}
  sexism type gender score gender.type
1     50    1   male    50      male.1
2     35    1   male    35      male.1
3     37    1   male    37      male.1
4     32    1   male    32      male.1
5     46    1   male    46      male.1
6     38    1   male    38      male.1
\end{Soutput}
\end{Schunk}


\item
Plot the data to visualize the effect of gender and type on the sexism score.
(The function \verb+stripchart+ can be used.)

\item
%We will work with the model having six different means one for each gender and type.
Formulate a model with a separate mean for each combination of gender and type. Call the model M1.
%\[
%M_1:  \quad E(y_i)=\mu_i=\mu + \alpha_j + \beta_k + \gamma_{jk} ,
% \quad k=1,2,3, \quad j=\textrm{boy, girl}
%\]
Fit the model M1 using the \verb+glm+-function and find the estimates of
 the model parameters.

\item
Use the function \verb+predict+ to calculate the fitted values and plot these fitted values.
The function \verb+coplot+ can be used here. If \verb+m1+ is the object holding your
model fit of M1 the following will work
\begin{Schunk}
\begin{Sinput}
 sexism <- transform(sexism,fit=predict(m1))
 coplot(fit ~ type | gender, data = sexism ,panel = panel.smooth, pch = c(18))
\end{Sinput}
\end{Schunk}
Does the plot indicate an interaction between sex and type?

Another possibility for such a conditional plot is via trellis-graphics implemented in the \verb+lattice+ package
\begin{Schunk}
\begin{Sinput}
 library(lattice)
 print(
  xyplot(fit~type|gender,data=sexism,type='l')
  )
\end{Sinput}
\end{Schunk}
or
\begin{Schunk}
\begin{Sinput}
 print(
  xyplot(fit~type,groups=gender,data=sexism,type='l',auto.key=TRUE)
  )
\end{Sinput}
\end{Schunk}
\item
Now consider also the model with no interaction between gender and type
\[
M_2:  \quad E(y_{jk})=\mu_{jk}=\mu + \alpha_j + \beta_k,
 \quad k=1,2,3, \quad j=\textrm{boy, girl}
\]
Fit this model using the \verb+glm+-function and test the hypothesis of
no interaction using the function \verb+anova+.
\item
Use the model M1 to obtain an estimate for the average sexism score  for males across the types. Weight the three levels of type equally
and use the \verb+esticon function+.
The resulting mean is called the LSMEANS (or population average)
for males across type.
\item
Similarly one can calculate the LSMEAN for females across type.
One should consider these LSMEANS as possibly interesting predictions of the
general level of sexism for males and females.
But because there is an interaction between gender and type, it is 
not an interesting scientific hypothesis, that these LSMEANS should be the same.
Nevertheless, it is mathematically possible to test the hypothesis.

In R you can do this with the \verb+drop1+ function, but for the
contrast of the factors in the model you must use the \verb+contrast.sum+
contrast, not the \verb+contr.treatment+ we normally use.
\begin{Schunk}
\begin{Sinput}
 old<-options()$contrasts
 options(contrasts=c('contr.sum','contr.poly'))
 m1L<- glm(score ~ gender + type+ gender:type,data=sexism)
 drop1(m1L,.~.,test='F')
\end{Sinput}
\begin{Soutput}
Single term deletions

Model:
score ~ gender + type + gender:type
            Df Deviance    AIC F value     Pr(F)    
<none>           1365.3 371.76                      
gender       1   1593.5 379.03  9.0237  0.004034 ** 
type         2   2022.7 391.34 13.0013 2.459e-05 ***
gender:type  2   1624.6 378.19  5.1279  0.009141 ** 
---
Signif. codes:  0 '***' 0.001 '**' 0.01 '*' 0.05 '.' 0.1 ' ' 1 
\end{Soutput}
\end{Schunk}
The row with \verb+gender+ gives the test that the averages for males and 
females (equally weighted across type-levels) are the same.
This hypothesis should by no means be considered as testing the 
main-effects of gender accounted for type, as it is sometimes expressed. 
Because there is interaction between gender and type there is no point in speaking of a main effect.

Finally we reset the contrasts
\begin{Schunk}
\begin{Sinput}
 options(old)
\end{Sinput}
\end{Schunk}
\end{enumerate}

\section{Bacteria}

\begin{topics}
  Linear normal model; log-transformation.
\end{topics}

In an experiment on nutrients on bacterial growth,
bacteria were grown in different media with the nutrients sucrose and leucine added.
After fours days the numbers of bacteria, \verb+density+, were counted.
\begin{Schunk}
\begin{Sinput}
 data(bactsucrose, package="LiSciData")
 head(bactsucrose)
\end{Sinput}
\begin{Soutput}
  day sucrose leucine  density
1   1       1       1 2.47e+07
2   1       1       2 3.81e+07
3   1       1       3 3.05e+08
4   1       2       1 1.22e+07
5   1       2       2 8.93e+07
6   1       2       3 1.54e+09
\end{Soutput}
\end{Schunk}

\begin{enumerate}
\item Assume the responses to be normally distributed and use an
  identity link.  Assume the predictors \verb+day+, \verb+sucrose+ and
  \verb+leucine+ as factors.  Fit a model which is additive in
  \verb+day+ and with an interaction between the other two factors.

\item
Look at the residuals. What do you observe?
\item

  Fit the model, where you assume that the transformed observations
\verb+log(density)+ are normally distributed.
Look at the residuals.

\end{enumerate}

\end{document}

















