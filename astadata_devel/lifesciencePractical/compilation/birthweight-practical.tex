\documentclass[12pt]{article}

\usepackage{a4wide,hyperref}
\usepackage[T1]{fontenc}
\usepackage{url,a4}
\usepackage{boxedminipage,color,xcolor}
\usepackage{shortvrb}
\usepackage{framed}
\usepackage{comment}

\usepackage[inline,nomargin,draft]{fixme}
\usepackage{listings}
\usepackage{bm}
\usepackage{framed}
\usepackage{comment}
\usepackage{amsmath}

\RequirePackage{color,fancyvrb,amsmath,amsfonts}

\MakeShortVerb�

%%
%% Easy matrices:
%%
\newcommand{\matrxr}[2][rrrrrrrrrrrrrrrrrrrrrrrrrrrrrrrrrrrrr]
{\left[
    \begin{array}{#1}
      #2 \\
    \end{array}
  \right]}
% Usage: $\matrxr{-1&-2\\3&4}$

\newcommand{\matrxc}[2][cccccccccccccccccccccccccccccccccccc]
{\left[
    \begin{array}{#1}
      #2 \\
    \end{array}
  \right]}
% Usage: $\matrxl{-1&-2\\3&4}$

\newcommand{\matrxl}[2][lllllllllllllllllllllllllllllllllllll]
{\left[
    \begin{array}{#1}
      #2 \\
    \end{array}
  \right]}
% Usage: $\matrxl{-1&-2\\3&4}$


\def\R{\texttt{R}}
\def\code#1{\texttt{#1}}
\def\ok{\texttt{OK}}
\def\fix{\texttt{FIX}}

\def\red#1{{\bf \textcolor{red}{#1}}}
\def\blue#1{{\bf \textcolor{blue}{#1}}}
\def\purple#1{{\bf \textcolor{purple}{#1}}}

\def\com#1{{\em{\color{red} #1}}}
\def\comi#1{{\bf\sc{\color{darkblue} #1}}\index{#1}}
\def\comic#1{{\bf\sc{\color{darkblue} #1}}\index{#1}}

\def\simn{\sim\mathcal N}
\def\simx{\sim\chi^2}
\def\xx{\chi^2}
\def\norm{\mathcal N}


\DeclareMathOperator{\RR}{\mathbb{R}}
\DeclareMathOperator{\EE}{\mathbb{E}}
\DeclareMathOperator{\var}{\mathbb{V}ar}
\DeclareMathOperator{\cov}{\mathbb{C}ov}
%\DeclareMathOperator{\norm}{N}
\DeclareMathOperator{\spanx}{span}
\DeclareMathOperator{\corr}{Corr}
\DeclareMathOperator{\deter}{det}
\DeclareMathOperator{\trace}{tr}
\DeclareMathOperator{\diag}{diag}
\DeclareMathOperator{\logit}{logit}
\DeclareMathOperator{\odds}{odds}

\def\vnorm#1{\left|\left|#1\right|\right|} 
\def\innerp#1{\left\langle #1 \right\rangle}
\def\inv{^{-1}}
\def\transp{^{\top}}
\definecolor{darkblue}{rgb}{0,0,.5}

%%UHH commands added by UHH
%%some new enviroments
\newtheorem{theorem}{Theorem}[section]
\newenvironment{example}{\medskip\noindent\refstepcounter{theorem}{\bf
Example \thetheorem\ }}{\hfill $\Box$ }
\newenvironment{theoreme}{\medskip\noindent\refstepcounter{theorem}{\bf
Theorem \thetheorem\ }}{}

\definecolor{shadecolor}{gray}{0.91}
\definecolor{howR}{gray}{0.91}
%\specialcomment{howR}{\begin{shaded}\small R code: }{\end{shaded}}
%\specialcomment{howR}{\begin{shaded}\vspace{-4mm}\small}{\vspace{-3mm}\end{shaded}}
%\specialcomment{sblock}{\begin{shaded}\vspace{-4mm}\small}{\vspace{-3mm}\end{shaded}}

\specialcomment{solution}
{\begin{shaded} Solution: }{\end{shaded}}

\specialcomment{topics}
{\begin{shaded} TOPICS: }{\end{shaded}}



\newenvironment{howR}{}{}

\newenvironment{examplea}
                {\begin{example}  \em \openup-1pt }
                {\hfill $\Box$  %\vspace{5mm}
                \end{example}}



%%\excludecomment{topics}
%%\excludecomment{solution}

\usepackage{Sweave}
\begin{document}


\RecustomVerbatimEnvironment{Sinput}{Verbatim}%
    {fontsize=\scriptsize,frame=single,framerule=1pt,
      rulecolor=\color{red},
      fillcolor=\color{yellow}
    }
\RecustomVerbatimEnvironment{Soutput}{Verbatim}%
    {fontsize=\scriptsize, frame=single,framerule=0.1pt}




% \renewenvironment{Schunk}{\linespread{.85}\scriptsize}{}
    
%% Efter preamble
% \definecolor{myGray}{rgb}{0.95,0.95,0.95}

% \makeatletter
% \renewenvironment{Schunk}{
%   \begin{lrbox}{\@tempboxa}
%     \begin{boxedminipage}
%       {\columnwidth}\scriptsize}
%     {\end{boxedminipage}
%   \end{lrbox}%
%   \colorbox{myGray}{\usebox{\@tempboxa}}
% }
% \makeatother


\parindent0pt
\parskip5pt


{Miniproject: Birthweight of babies}

{S�ren H�jsgaard \hfill Created: \today}

\hrule


\setkeys{Gin}{width=4in,height=3in}


\section{Birthweight of babies}
\label{sec:birthweight-babies}


\begin{topics} 
Linear normal model
\end{topics}

The \verb+births+ data in the \verb+Epi+ package contains recordings
of birtweights of babies in a London hospital together with recordings
of several possible explanatory variables.

The purpose of this exercise is to establish a model for understanding
how the birthweight is related to the explanatory variables. 

\begin{Schunk}
\begin{Sinput}
 library(Epi)
 data(births)
 head(births)
\end{Sinput}
\begin{Soutput}
  id bweight lowbw gestwks preterm matage hyp sex
1  1    2974     0   38.52       0     34   0   2
2  2    3270     0      NA      NA     30   0   1
3  3    2620     0   38.15       0     35   0   2
4  4    3751     0   39.80       0     31   0   1
5  5    3200     0   38.89       0     33   1   1
6  6    3673     0   40.97       0     33   0   2
\end{Soutput}
\begin{Sinput}
 dim(births)
\end{Sinput}
\begin{Soutput}
[1] 500   8
\end{Soutput}
\begin{Sinput}
 sapply(births, class)
\end{Sinput}
\begin{Soutput}
       id   bweight     lowbw   gestwks   preterm    matage       hyp       sex 
"numeric" "numeric" "numeric" "numeric" "numeric" "numeric" "numeric" "numeric" 
\end{Soutput}
\end{Schunk}


\begin{enumerate}
\item
  \begin{enumerate}
  \item Plot the birth weights of babies and observe the
    distribution. Comments?

\begin{solution} 

\begin{Schunk}
\begin{Sinput}
 hist(births$bweight)
\end{Sinput}
\end{Schunk}
\includegraphics{fig/PRACS-004}
\end{solution}    

\item See how birth weight of babies are related to gestation
    period, and maternal age by scatter plot. Differentiate metween males and
    females in the plot. (Hint: �xyplot()� could be your friend.) Comments?

  \item Make a scatterplot of all pair of variables. (Hint: �pairs()�
    could be your friend). Comments?
  \end{enumerate}

\item
\begin{enumerate}
\item Assume that number of boys follows a 
  $bin(N=500,\theta)$--distribution.

\item Estimate $\theta$ from observed
  data. Estimate the variance of $\hat\theta$ from data. 

\item What is the probability of observing a male child?

\item What is the probability of observing at least 26 babies of the
  same sex in 44 births? (Use above estimated value of $\theta$.)
\end{enumerate}

\item Fit a linear normal model $M_1$ for $bweight \sim
  gestwks + matage$ to the \verb+births+ data. 

  \begin{enumerate}
  \item Write down in precise mathematical terms what your model is.
  \item Write down in your own words what the assumptions behind a
    linear normal model are. Investigate the residuals of $M_1$.
    Comment on whether you find that these assumptions are likely to
    be satisfied for these data.
    
  \item Write down in your own words what a least squares fit of a
    model to data is.
    
  \item From model $M_1$ report the estimated parameters and their
    standard deviations. 
    
  \item In model $M_1$, calculate the values of fitted values (use
    function \verb+predict()+) and plot the fitted line over the
    observations.
  
  \item Test the effect of each term in the model $M_1$ using the
    function \verb+drop1()+.
    
  \item Predict the birth weights at weeks $10,15,20,25,30,35,40,45,50$
    for women of age $20$, $30$ and $40$. Include confidence intervals
    in your prediction. Present the results as a table and as
    illustrative graphics. Comment on these predictions.  
  \end{enumerate}
  
\item Fit an interaction model $M_2$ for $bweight \sim
  gestwks+matage+gestwks*matage$ to the \verb+births+ data. 

\item Write down $M_2$ in precise mathematical terms.

\item Investigate whether $M_2$ fits well to data.

\item Test whether the additive model is more appropriate or the
  interactive model. 

\item Fit a model $M_3$ that differentiate the two slope for sexes.

\item Write down $M_3$ in precise mathematical terms.

\item Investigate whether $M_3$ fits well to data.
  
\item Investigate whether it is necessary to have two slopes for each sex
  by comparing the models $M_1$ and $M_3$.

\end{enumerate}






\end{document}

















