%% This file has been modified by igattr
%% Do not edit manually
% \VignetteIndexEntry{Users guide to geepack}
% \VignetteKeyword{Generalized Estimating Equation}
% \VignetteKeyword{Working correlation matrix}

\documentclass{article}
\usepackage{boxedminipage,color,a4,shortvrb,hyperref}
\usepackage[latin1]{inputenc}
\usepackage[T1]{fontenc}
\MakeShortVerb|

\def\pkg#1{{\bf #1}}





\title{On the usage of the \texttt{geepack} }
\author{S{\o}ren H{\o}jsgaard and Ulrich Halekoh}
\date{\pkg{geepack} version 1.2-0 as of 2014-09-13}


\usepackage{Sweave}
\begin{document}

\parindent0pt\parskip4pt

%% Efter preamble
\definecolor{myGray}{rgb}{0.95,0.95,0.95}
\makeatletter
\renewenvironment{Schunk}{
  \begin{lrbox}{\@tempboxa}
    \begin{boxedminipage}
      {\columnwidth}\scriptsize}
    {\end{boxedminipage}
  \end{lrbox}%
  \colorbox{myGray}{\usebox{\@tempboxa}}
}
\makeatother

\maketitle

\section{Introduction}
\label{sec:intro}


The primary reference for the |geepack| package is the Halekoh, U.,
H�jsgaard, S., Yan, J. (2006) -- paper in Journal of Statistical
Software, see

\begin{Schunk}
\begin{Sinput}
> library(geepack)
> citation("geepack")
\end{Sinput}
\begin{Soutput}
To cite geepack in publications use:

  H�jsgaard, S., Halekoh, U. & Yan J. (2006) The R Package geepack for Generalized
  Estimating Equations Journal of Statistical Software, 15, 2, pp1--11

  Yan, J. & Fine, J.P. (2004) Estimating Equations for Association Structures
  Statistics in Medicine, 23, pp859--880.

  Yan, J (2002) geepack: Yet Another Package for Generalized Estimating Equations
  R-News, 2/3, pp12-14.
\end{Soutput}
\end{Schunk}

If you use |geepack| in your own work, please do cite the above
reference.


This note contains a few extra examples. We illustrate the usage of a
the |waves| argument and the |zcor| argument together with a fixed
working correlation matrix for the |geeglm()| function. To illustrate
these features we simulate some data suitable for a regression model.

\begin{Schunk}
\begin{Sinput}
> library(geepack)
> timeorder <- rep(1:5, 6)
> tvar      <- timeorder + rnorm(length(timeorder))
> idvar <- rep(1:6, each=5)
> uuu   <- rep(rnorm(6), each=5)
> yvar  <- 1 + 2*tvar + uuu + rnorm(length(tvar))
> simdat <- data.frame(idvar, timeorder, tvar, yvar)
> head(simdat,12)
\end{Sinput}
\begin{Soutput}
   idvar timeorder        tvar      yvar
1      1         1  2.78691314  5.422171
2      1         2  2.49785048  4.829722
3      1         3  1.03338284  3.312259
4      1         4  4.70135590  9.053832
5      1         5  4.52720859 10.447779
6      2         1 -0.06782371 -1.087285
7      2         2  1.78202509  4.745779
8      2         3  1.97399555  4.668961
9      2         4  3.27110877  7.355274
10     2         5  4.37496073  9.726676
11     3         1 -0.68669331 -1.342365
12     3         2  2.83778704  5.875711
\end{Soutput}
\end{Schunk}

Notice that clusters of data appear together in |simdat| and that
observations are ordered (according to |timeorder|) within clusters.

We can fit a model with an AR(1) error structure as

\begin{Schunk}
\begin{Sinput}
> mod1 <- geeglm(yvar~tvar, id=idvar, data=simdat, corstr="ar1")
> mod1
\end{Sinput}
\begin{Soutput}
Call:
geeglm(formula = yvar ~ tvar, data = simdat, id = idvar, corstr = "ar1")

Coefficients:
(Intercept)        tvar 
  0.4399188   2.1106751 

Degrees of Freedom: 30 Total (i.e. Null);  28 Residual

Scale Link:                   identity
Estimated Scale Parameters:  [1] 1.136851

Correlation:  Structure = ar1    Link = identity 
Estimated Correlation Parameters:
    alpha 
0.5346614 

Number of clusters:   6   Maximum cluster size: 5 
\end{Soutput}
\end{Schunk}

This works because observations are ordered according to time within
each subject in the dataset.




\section{Using the \texttt{waves} argument}
\label{sec:xxx}


If observatios were not ordered according to cluster and time within
cluster we would get the
wrong result:

\begin{Schunk}
\begin{Sinput}
> set.seed(123)
> ## library(doBy)
> simdatPerm <- simdat[sample(nrow(simdat)),]
> ## simdatPerm <- orderBy(~idvar, simdatPerm)
> simdatPerm <- simdatPerm[order(simdatPerm$idvar),]
> head(simdatPerm)
\end{Sinput}
\begin{Soutput}
  idvar timeorder     tvar      yvar
2     1         2 2.497850  4.829722
4     1         4 4.701356  9.053832
1     1         1 2.786913  5.422171
3     1         3 1.033383  3.312259
5     1         5 4.527209 10.447779
9     2         4 3.271109  7.355274
\end{Soutput}
\end{Schunk}

Notice that in |simdatPerm| data is ordered according to subject but
the time ordering within subject is random.

Fitting the model as
before gives

\begin{Schunk}
\begin{Sinput}
> mod2 <- geeglm(yvar~tvar, id=idvar, data=simdatPerm, corstr="ar1")
> mod2
\end{Sinput}
\begin{Soutput}
Call:
geeglm(formula = yvar ~ tvar, data = simdatPerm, id = idvar, 
    corstr = "ar1")

Coefficients:
(Intercept)        tvar 
  0.1969892   2.2155856 

Degrees of Freedom: 30 Total (i.e. Null);  28 Residual

Scale Link:                   identity
Estimated Scale Parameters:  [1] 1.123093

Correlation:  Structure = ar1    Link = identity 
Estimated Correlation Parameters:
  alpha 
0.51013 

Number of clusters:   6   Maximum cluster size: 5 
\end{Soutput}
\end{Schunk}

Likewise if clusters do not appear contigously in data we also get the
wrong result (the clusters are not recognized):

\begin{Schunk}
\begin{Sinput}
> ## simdatPerm2 <- orderBy(~timeorder, data=simdat)
> simdatPerm2 <- simdat[order(simdat$timeorder),]
> geeglm(yvar~tvar, id=idvar, data=simdatPerm2, corstr="ar1")
\end{Sinput}
\begin{Soutput}
Call:
geeglm(formula = yvar ~ tvar, data = simdatPerm2, id = idvar, 
    corstr = "ar1")

Coefficients:
(Intercept)        tvar 
  0.2589139   2.1828479 

Degrees of Freedom: 30 Total (i.e. Null);  28 Residual

Scale Link:                   identity
Estimated Scale Parameters:  [1] 1.118158

Correlation:  Structure = ar1    Link = identity 
Estimated Correlation Parameters:
alpha 
    0 

Number of clusters:   30   Maximum cluster size: 1 
\end{Soutput}
\end{Schunk}








To obtain the right result we must give the |waves| argument:

\begin{Schunk}
\begin{Sinput}
> wav <- simdatPerm$timeorder
> wav
\end{Sinput}
\begin{Soutput}
 [1] 2 4 1 3 5 4 5 2 1 3 2 3 4 5 1 5 4 2 1 3 3 4 5 1 2 2 5 4 1 3
\end{Soutput}
\begin{Sinput}
> mod3 <- geeglm(yvar~tvar, id=idvar, data=simdatPerm, corstr="ar1", waves=wav)
> mod3
\end{Sinput}
\begin{Soutput}
Call:
geeglm(formula = yvar ~ tvar, data = simdatPerm, id = idvar, 
    waves = wav, corstr = "ar1")

Coefficients:
(Intercept)        tvar 
  0.4399188   2.1106751 

Degrees of Freedom: 30 Total (i.e. Null);  28 Residual

Scale Link:                   identity
Estimated Scale Parameters:  [1] 1.136851

Correlation:  Structure = ar1    Link = identity 
Estimated Correlation Parameters:
    alpha 
0.5346614 

Number of clusters:   6   Maximum cluster size: 5 
\end{Soutput}
\end{Schunk}

\section{Using a fixed correlation matrix and the \texttt{zcor} argument}
\label{sec:xxx}

Suppose we want to use a fixed working correlation matrix:

\begin{Schunk}
\begin{Sinput}
> cor.fixed <- matrix(c(1    , 0.5  , 0.25,  0.125, 0.125,
+                       0.5  , 1    , 0.25,  0.125, 0.125,
+                       0.25 , 0.25 , 1   ,  0.5  , 0.125,
+                       0.125, 0.125, 0.5  , 1    , 0.125,
+                       0.125, 0.125, 0.125, 0.125, 1     ), 5, 5)
> cor.fixed
\end{Sinput}
\begin{Soutput}
      [,1]  [,2]  [,3]  [,4]  [,5]
[1,] 1.000 0.500 0.250 0.125 0.125
[2,] 0.500 1.000 0.250 0.125 0.125
[3,] 0.250 0.250 1.000 0.500 0.125
[4,] 0.125 0.125 0.500 1.000 0.125
[5,] 0.125 0.125 0.125 0.125 1.000
\end{Soutput}
\end{Schunk}

Such a working correlation matrix has to be passed to |geeglm()| as a
vector in the |zcor| argument. This vector can be created using the
|fixed2Zcor()| function:

\begin{Schunk}
\begin{Sinput}
> zcor <- fixed2Zcor(cor.fixed, id=simdatPerm$idvar, waves=simdatPerm$timeorder)
> zcor
\end{Sinput}
\begin{Soutput}
 [1] 0.125 0.500 0.250 0.125 0.125 0.500 0.125 0.250 0.125 0.125 0.125 0.125 0.125 0.500 0.125
[16] 0.125 0.125 0.500 0.250 0.250 0.250 0.125 0.125 0.500 0.500 0.125 0.250 0.125 0.125 0.125
[31] 0.125 0.125 0.125 0.125 0.125 0.125 0.500 0.500 0.250 0.250 0.500 0.125 0.250 0.250 0.125
[46] 0.125 0.125 0.125 0.125 0.500 0.125 0.125 0.500 0.250 0.125 0.125 0.125 0.125 0.500 0.250
\end{Soutput}
\end{Schunk}

Notice that |zcor| contains correlations between measurements within
the same cluster. Hence if a cluster contains only one observation,
then there will be generated no entry in |zcor| for that cluster. Now
we can fit the model with:

\begin{Schunk}
\begin{Sinput}
> mod4 <- geeglm(yvar~tvar, id=idvar, data=simdatPerm, corstr="fixed", zcor=zcor)
> mod4
\end{Sinput}
\begin{Soutput}
Call:
geeglm(formula = yvar ~ tvar, data = simdatPerm, id = idvar, 
    zcor = zcor, corstr = "fixed")

Coefficients:
(Intercept)        tvar 
  0.4692344   2.1148665 

Degrees of Freedom: 30 Total (i.e. Null);  28 Residual

Scale Link:                   identity
Estimated Scale Parameters:  [1] 1.133532

Correlation:  Structure = fixed    Link = identity 
Estimated Correlation Parameters:
alpha:1 
      1 

Number of clusters:   6   Maximum cluster size: 5 
\end{Soutput}
\end{Schunk}






\end{document}
