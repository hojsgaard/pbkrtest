%%%
%%% Automatically generated file from  PigGrowth.Rnw - DO NOT EDIT MANUALLY
%%%
%\newslide
\section{Pig growth -- \texttt{dietox}}
\label{sec:dietox}

The data used here are described by \cite{lauridsen:99} and contains growth data
for a pig feeding experiment. Data is available as the \texttt{dietox} data set
in the \texttt{doBy} package for \R. 

One of the questions asked in connection with the experiment was whether 
copper added to pig feed increase/decrease growth. 
Copper (hereafter abbreviated Cu) was used in three levels Cu=1: No copper,
Cu=2: xxx mg/kg and Cu=3: yyy mg/kg. Here we shall analyze data as if they were
layed out as a factorial experiment (even though the design was a (almost)
balanced incomplete block design -- because there is an issue of a litter
effect). 
The weight of slaughter pigs were measured weekly over a 12 week period. 


\section{Loading data}

Data can be loaded as:
\begin{Schunk}
\begin{Sinput}
> library(doBy)
\end{Sinput}
\begin{Soutput}
Loading required package: ash 
\end{Soutput}
\begin{Sinput}
> data(dietox)
> dietox[1:5, ]
\end{Sinput}
\begin{Soutput}
    Weight      Feed Time  Pig Evit Cu Litter
1 26.50000        NA    1 4601    1  1      1
2 27.59999  5.200005    2 4601    1  1      1
3 36.50000 17.600000    3 4601    1  1      1
4 40.29999 28.500000    4 4601    1  1      1
5 49.09998 45.200001    5 4601    1  1      1
\end{Soutput}
\end{Schunk}

Cu is coded with levels 1,2,3 meaning that \R\ will regard Cu as a numeric
variable, which it is not. To turn Cu into a factor we do:
\begin{Schunk}
\begin{Sinput}
> dietox$Cu <- as.factor(dietox$Cu)
\end{Sinput}
\end{Schunk}

If instead data was saved as a comma--searated file (a .csv file) it could be
loaded as
\begin{Schunk}
\begin{Sinput}
> dietox <- read.csv("dietox.csv")
\end{Sinput}
\end{Schunk}

\newslide
\subsection{Looking at data}



\shfig{dietox01}{XXX}{height=4cm}

The weight as function of time is shown in  Figure~\ref{fig:dietox01}. This plot
is produced using the \texttt{plotBy()} function in the \texttt{doBy} package as follows:


First, make space for 1 row and 3 columns of plots:
\begin{Schunk}
\begin{Sinput}
> par(mfrow = c(1, 3))
\end{Sinput}
\end{Schunk}

Then call the \texttt{plotBy()} function:
\begin{Schunk}
\begin{Sinput}
> plotBy(Weight ~ Time, subject = Pig, group = Cu, title = "Cu=", 
+     data = dietox, col = 1:100, lines = T)
\end{Sinput}
\end{Schunk}

\newslide
 Figure~\ref{fig:dietox01} suggests
\begin{itemize}
\item Approximately linear growth curves
\item Some tendency for variance to increase with mean
\end{itemize}

\newslide

A next step could be to calculate the mean and variance for each combination of
Cu and Time using the \texttt{summaryBy()} function:
\begin{Schunk}
\begin{Sinput}
> m.dietox <- summaryBy(Weight ~ Cu + Time, data = dietox, FUN = c(mean, 
+     var))
> m.dietox[1:5, ]
\end{Sinput}
\begin{Soutput}
  Cu Time mean.Weight var.Weight
1  1    1    25.34782  12.264414
2  2    1    25.50000   9.656658
3  3    1    26.14999  18.799118
4  1    2    29.48695  15.213879
5  2    2    29.33599  16.070716
\end{Soutput}
\end{Schunk}
\newslide
-- and to plot these (see Figure~\ref{fig:dietox01mean}).
\begin{Schunk}
\begin{Sinput}
> par(mfrow = c(1, 1))
> plotBy(mean.Weight ~ Time, subject = Cu, data = m.dietox, lines = T, 
+     col = c("black", "red", "green"), silent = F)
\end{Sinput}
\begin{Soutput}
  symbol colour group subject line
1      1  black  .by.       1    1
2      1    red  .by.       2    1
3      1  green  .by.       3    1
\end{Soutput}
\end{Schunk}

\shfig{dietox01mean}{XXX}{height=6cm,width=7cm}

Figure~\ref{fig:dietox01mean} suggests that 
\begin{itemize}
\item Growth is not quite linear. The curves are curved (slightly S--shaped)!
\item If there is a treatment effect, then it is small!
\end{itemize}

Next we can plot the log variance against the log mean and fit a straight line
to the data (see Figure~\ref{fig:dietox-010}):
\begin{Schunk}
\begin{Sinput}
> plot(log(var.Weight) ~ log(mean.Weight), data = m.dietox)
> lm1 <- lm(log(var.Weight) ~ log(mean.Weight), data = m.dietox)
> abline(lm1, lwd = 2, col = "red")
> lm1
\end{Sinput}
\begin{Soutput}
Call:
lm(formula = log(var.Weight) ~ log(mean.Weight), data = m.dietox)

Coefficients:
     (Intercept)  log(mean.Weight)  
          -1.134             1.212  
\end{Soutput}
\end{Schunk}
The plot suggests that $\log Var(y) \approx a + b \log E(y)$ and hence
$$
Var(y) \approx e^a \cdot E(y)^b
$$
The slope $b$ is about one suggesting that the variance is approximately
proportional to the mean.

\shfig{dietox-010}{XXX}{height=6cm,width=7cm}

\newslide

\subsection{Onto fitting some models}


 @ 
 <<>>=
 lm1 <- lm(mean.Weight~Cu*Time, data=m.dietox)
 lm2 <- lm(mean.Weight~Cu*(Time+I(Time^2)), data=m.dietox)
 lm3 <- lm(mean.Weight~Cu*poly(Time,3), data=m.dietox)
 @ %def 


% par(mfrow=c(1,3)
% plotBy(fitted(lm1)~Time, subject=Cu, data=m.dietox,lines=T,col=1:3)
% plotBy(fitted(lm2)~Time, subject=Cu, data=m.dietox,lines=T,col=1:3)
% plotBy(fitted(lm3)~Time, subject=Cu, data=m.dietox,lines=T,col=1:3)



\newslide

\subsubsection{Linear regression}

Let $c$: Cu, $p$: pig (within treatment), $t$: time.

Simple regression model:
$$
y_{cpt} = \alpha+ \beta t+ \alpha_c + \beta_c t + e_{cpt}; \ \ e_{cpt}\sim N(0,\sigma^2)
$$
Written shortly as:
$$
[y] = 1 + \underline{time} + Cu + Cu * \underline{time} + [e]
$$


\newslide
\subsubsection{In \SAS\ and \R}

\begin{verbatim}
proc mixed data=dietox noinfo noclprint;
  class cu pig;
  model weight = time cu cu * time / solution htype=1;
run;
\end{verbatim}


\begin{Schunk}
\begin{Sinput}
> fm0 <- lm(Weight ~ Time + Cu + Cu * Time, data = dietox)
\end{Sinput}
\end{Schunk}


To plot the residuals and the fitted values, one can do:
\begin{Schunk}
\begin{Sinput}
> par(mfrow = c(2, 3))
> plotBy(resid(fm0) ~ Time, subject = Pig, group = Cu, data = dietox, 
+     lines = T, col = 1:100)
> plotBy(fitted(fm0) ~ Time, subject = Pig, group = Cu, data = dietox, 
+     lines = T, col = 1:3)
\end{Sinput}
\end{Schunk}

\newslide
\subsubsection{Figures}
\shfig{dietox02}{XXX}{}

If model is appropriate, residuals should fluctuate randomly around zero. That
is clearly not so. 

\newslide
\subsubsection{Adding a random intercept term}

Pigs who start above (below) average tend to keep that position throughout the
experiment. 

This suggests to add a pig--specific (random) intercept term:

$$
y_{cpt} = \alpha+ \beta t+ \alpha_c + \beta_c t +{\color{red}U_{cp}}+ e_{cpt}; \ \ 
%e_{cpt}\sim N(0,\sigma_e^2), U_{cp}\sim N(0,\sigma_U^2), 
$$
Written shortly as:
$$
[y] = 1 + \underline{time} + Cu + Cu * \underline{time} + {\color{red}[Cu*Pig]}+[e]
$$




\newslide
\subsubsection{In \SAS\ and \R}

\begin{verbatim}
proc mixed data=dietox noinfo noclprint;
  class cu pig;
  model weight = time cu cu * time / solution htype=1;
  random int / subject=cu*pig;
run;
\end{verbatim}

\begin{Schunk}
\begin{Sinput}
> library(nlme)
> fm1 <- lme(Weight ~ Time + Cu + Cu * Time, data = dietox, random = ~1 | 
+     Pig)
\end{Sinput}
\end{Schunk}


\newslide

\newslide
\subsubsection{Figures}

\shfig{dietox03}{XXX}{}

Residuals still problematic: Residuals on some pigs steadily increasing, others
steadily decreasing.


\newslide
\subsubsection{Adding a random slope term}

To account for this phenomenon, one can add a pig--specific (random) slope term:
$$
y_{cpt} = \alpha+ \beta t+ \alpha_c + \beta_c t +U_{cp}+ {\color{red}W_{cp} t} + e_{cpt}; \ \ 
%e_{cpt}\sim N(0,\sigma_e^2), U_{cp}\sim N(0,\sigma_U^2), W_{cp}\sim N(0,\sigma_W^2), 
$$
Written shortly as:
$$
[y] = 1 + \underline{time} + Cu + Cu * \underline{time} + [CU*Pig] + 
{\color{red}[CU*Pig]*\underline{time}}+[e]
$$

Such a model is called a \com{random regression model}.

\newslide
\subsubsection{In \SAS\ and \R}

\begin{verbatim}
data dietox; set dietox; 
  timec = time;

proc mixed data=dietox noinfo noclprint;
  class cu pig timec;
  model weight = time cu cu * time / solution htype=1;
  random int time/ subject=cu*pig;
run;
\end{verbatim}

\begin{Schunk}
\begin{Sinput}
> fm2 <- lme(Weight ~ Cu * Time, data = dietox, random = ~1 + Time | 
+     Pig)
\end{Sinput}
\end{Schunk}
\newslide

