\documentclass[12pt]{article}

\usepackage{a4wide,hyperref}
\usepackage[T1]{fontenc}
\usepackage{url,a4}
\usepackage{boxedminipage,color,xcolor}
\usepackage{shortvrb}
\usepackage{framed}
\usepackage{comment}

\usepackage[inline,nomargin,draft]{fixme}
\usepackage{listings}
\usepackage{bm}
\usepackage{framed}
\usepackage{comment}
\usepackage{amsmath}

\RequirePackage{color,fancyvrb,amsmath,amsfonts}

\MakeShortVerb�

%%
%% Easy matrices:
%%
\newcommand{\matrxr}[2][rrrrrrrrrrrrrrrrrrrrrrrrrrrrrrrrrrrrr]
{\left[
    \begin{array}{#1}
      #2 \\
    \end{array}
  \right]}
% Usage: $\matrxr{-1&-2\\3&4}$

\newcommand{\matrxc}[2][cccccccccccccccccccccccccccccccccccc]
{\left[
    \begin{array}{#1}
      #2 \\
    \end{array}
  \right]}
% Usage: $\matrxl{-1&-2\\3&4}$

\newcommand{\matrxl}[2][lllllllllllllllllllllllllllllllllllll]
{\left[
    \begin{array}{#1}
      #2 \\
    \end{array}
  \right]}
% Usage: $\matrxl{-1&-2\\3&4}$


\def\R{\texttt{R}}
\def\code#1{\texttt{#1}}
\def\ok{\texttt{OK}}
\def\fix{\texttt{FIX}}

\def\red#1{{\bf \textcolor{red}{#1}}}
\def\blue#1{{\bf \textcolor{blue}{#1}}}
\def\purple#1{{\bf \textcolor{purple}{#1}}}

\def\com#1{{\em{\color{red} #1}}}
\def\comi#1{{\bf\sc{\color{darkblue} #1}}\index{#1}}
\def\comic#1{{\bf\sc{\color{darkblue} #1}}\index{#1}}

\def\simn{\sim\mathcal N}
\def\simx{\sim\chi^2}
\def\xx{\chi^2}
\def\norm{\mathcal N}


\DeclareMathOperator{\RR}{\mathbb{R}}
\DeclareMathOperator{\EE}{\mathbb{E}}
\DeclareMathOperator{\var}{\mathbb{V}ar}
\DeclareMathOperator{\cov}{\mathbb{C}ov}
%\DeclareMathOperator{\norm}{N}
\DeclareMathOperator{\spanx}{span}
\DeclareMathOperator{\corr}{Corr}
\DeclareMathOperator{\deter}{det}
\DeclareMathOperator{\trace}{tr}
\DeclareMathOperator{\diag}{diag}
\DeclareMathOperator{\logit}{logit}
\DeclareMathOperator{\odds}{odds}

\def\vnorm#1{\left|\left|#1\right|\right|} 
\def\innerp#1{\left\langle #1 \right\rangle}
\def\inv{^{-1}}
\def\transp{^{\top}}
\definecolor{darkblue}{rgb}{0,0,.5}

%%UHH commands added by UHH
%%some new enviroments
\newtheorem{theorem}{Theorem}[section]
\newenvironment{example}{\medskip\noindent\refstepcounter{theorem}{\bf
Example \thetheorem\ }}{\hfill $\Box$ }
\newenvironment{theoreme}{\medskip\noindent\refstepcounter{theorem}{\bf
Theorem \thetheorem\ }}{}

\definecolor{shadecolor}{gray}{0.91}
\definecolor{howR}{gray}{0.91}
%\specialcomment{howR}{\begin{shaded}\small R code: }{\end{shaded}}
%\specialcomment{howR}{\begin{shaded}\vspace{-4mm}\small}{\vspace{-3mm}\end{shaded}}
%\specialcomment{sblock}{\begin{shaded}\vspace{-4mm}\small}{\vspace{-3mm}\end{shaded}}

\specialcomment{solution}
{\begin{shaded} Solution: }{\end{shaded}}

\specialcomment{topics}
{\begin{shaded} TOPICS: }{\end{shaded}}



\newenvironment{howR}{}{}

\newenvironment{examplea}
                {\begin{example}  \em \openup-1pt }
                {\hfill $\Box$  %\vspace{5mm}
                \end{example}}



%%\excludecomment{topics}
%%\excludecomment{solution}

\usepackage{Sweave}
\begin{document}


\RecustomVerbatimEnvironment{Sinput}{Verbatim}%
    {fontsize=\scriptsize,frame=single,framerule=1pt,
      rulecolor=\color{red},
      fillcolor=\color{yellow}
    }
\RecustomVerbatimEnvironment{Soutput}{Verbatim}%
    {fontsize=\scriptsize, frame=single,framerule=0.1pt}




% \renewenvironment{Schunk}{\linespread{.85}\scriptsize}{}
    
%% Efter preamble
% \definecolor{myGray}{rgb}{0.95,0.95,0.95}

% \makeatletter
% \renewenvironment{Schunk}{
%   \begin{lrbox}{\@tempboxa}
%     \begin{boxedminipage}
%       {\columnwidth}\scriptsize}
%     {\end{boxedminipage}
%   \end{lrbox}%
%   \colorbox{myGray}{\usebox{\@tempboxa}}
% }
% \makeatother


\parindent0pt
\parskip5pt


{Miniproject: Logistic regression - Poisson regression - Gamma regression - GLM}

{Ulrich Halekoh and S�ren H�jsgaard \hfill Created: \today}


\hrule

\tableofcontents


\setkeys{Gin}{width=4in,height=3in}

\begin{topics}
Logistic regression  
\end{topics}

\section{Shuttle}
On January, 28, 1986, the shuttle Challenger (flight 61-I) 
exploded shortly after the start. Later it was found 
that 'a combustion gas leak' caused the catastrophe. 
The leakage may by attributed to the failure of one or several of six 
O-rings, which task was to seal the field joints of the motor.

The night before the event engineers discussed the problem 
in a three hour telephone conference. 
In the conference a plot of the failure of O-rings in former missions 
were discussed, but only data for flights with failures were used, 
the flights without failures were considered not to contribute valuable information.
Later the Rogers Commission noted that this was a failure of a 
proper data analysis.

\begin{Schunk}
\begin{Sinput}
 library(LiSciData)
 data(shuttleOrings)
\end{Sinput}
\end{Schunk}

\begin{enumerate}
\item
Plot the number of damages against the temperature. What seems to be 
the conclusion on the effect of temperature, if one disregards the flights where
no damages were observed?
\item
Fit a logistic regression model using \verb+temp+ and \verb+pres+ as covariates.
\item
Use a model where the temperature is the only covariate.
Predict the probability of the damage of at least one O-ring
at the temperature of 31 Fahrenheit (=(31-32)*5/9  ${}^0$C=-0.6 ${}^0$C.)
(Remember, if $\pi$ is the probability of a ring to fail, the probability of at least one ring to fail is 
$1-(1-\pi)^6$.)
Provide  a confidence interval.
\item
Fit models with the probit and the cloglog-link.
Predict again the above probability.
\end{enumerate}

\section{Mendel's law and primulae}
\begin{topics}
Test for binomial proportions  
\end{topics}

A total of 560 plants were classified by type of their leafs (flat or crimped) and the type of their eye
(normal or Primrose Queen). 
(In the picture
\url{	http://www.primulaworld.com/PWWeb/gallery/slides/sinensisTM3.html} the eye is the yellow
area round the mouth of the corolla tube). 
According to simple Mendelian law one would expect
a 3:1 ratio for the dominant to the recessive characteristic for each part of the plant.
For leafs one would expect  420 plants with 
flat leaves  and  140 plants with crimped leaves. For the eye one would would expect 420 plants normal and 120 plants
Primrose Queen.

\begin{table}[h]
\begin{tabular}{c|cc|c}
& \multicolumn{2}{c}{Leaves} & \\ \hline
Eye & Flat & Crimped & Total\\ \hline
Normal  & 328 & 77 &  405 \\
Primrose Queen & 122 & 33 & 155 \\ \hline
Total & 450 & 110 & 560 \\ \hline
\end{tabular}    
\end{table}

Concentrate first on the leaves where 450 are flat and 110 crimped. Describe the distribution of the observations, 
formulate the hypothesis and test it. 
% (Take a look at the one-sample testing problem in the 
% \verb+example-logistic.pdf+ document.)
Repeat the analysis for the eyes.

\section{A simple example on different presentation of tabulated data}
In a study on the ratio of born boys to girls one observed 10 families with 1 child
and 17 families with 2 children and recorded the number of boys.
Answer the question whether boys are equally frequent then girls in a family. 

How would you test the supposition?
The experimenter tabulated the data in three different ways. 
Which tabulation do you think is most convenient for your analysis?

\begin{table}[h]
\centering
\caption{Number of families with 1 or 2 children classified after their number of boys.}
\label{tab:boys1}
\begin{tabular}{lrrrr}\hline
& \multicolumn{3}{c}{Number of boys} & Total\\ 
number children & 0 & 1 & 2 \\ \hline
1               & 9 & 1 & - & 10\\
2               & 2 & 10 & 5  & 17\\ \hline
\end{tabular}
\end{table}
\begin{table}[h]
\centering
\caption{Number of families with 1 or 2 children classified after their number of boys and girls.}
\label{tab:boys2}
\begin{tabular}{cccr}\hline
 boys & girls & number children & number families \\
0 & 1 & 1 &   9 \\
0 & 2 &  2&  2 \\
1 & 0 &  1&  1 \\
1 & 1 &  2 & 10 \\
2 & 0 &  2& 5 \\ \hline
\end{tabular}
\end{table}

\begin{table}[h]
\centering
\caption{Number of families with 1 or 2 children classified after their number of boys and girls.}
\label{tab:boys3}
\begin{tabular}{cccc}\hline
number boys  & \multicolumn{3}{c}{number girls}\\ \hline 
      &  0 & 1 & 2 \\
0     & -  & 9 & 2 \\
1     &  1 & 10 & - \\
2     & 2  & - & - \\ \hline
\end{tabular} 
\end{table}



\begin{solution}


We use Table \ref{tab:boys2}.
We assume that in each family $i=1,\dots,27$ of size $n_i$ the number of boys is binomially distributed 
with probability $\pi$, 
\[
Y_{i}  \sim bin(n_i,\pi)
\]
The number of families give  the frequency how often  a certain number of boys is
observed in a specific family size. We use this as a frequency in the model fit.
\begin{Schunk}
\begin{Sinput}
 dboys<-data.frame(boys=c(0,0,1,1,2),girls=c(1,2,0,1,0),nfam=c(9,2,1,10,5))
\end{Sinput}
\end{Schunk}

Fit
\begin{Schunk}
\begin{Sinput}
 m<-glm(cbind(boys,girls)~1,data=dboys,family=binomial,weights=nfam)
\end{Sinput}
\end{Schunk}

We generate now a data set, where each family is individually represented, e.g.
the 2 families with 0 boy and 1 girl are represented by 2 observations (0,1) and (0,1).
The data set will have 10+17=27 rows, one for each family. 
\begin{Schunk}
\begin{Sinput}
 explode<-function(dat,number){
  #explodes data by copying each row in data with the entry i  number
  # number is a column of dat!
  numb<-dat[,number]
  dat<-dat[! (is.na(numb) | numb==0),]
  numb<-dat[,number]
  dat$inde<-1:nrow(dat)
  numb<-dat[,number]
  indec<-data.frame(inde=rep(dat$inde,numb))
  datexploded<-merge(dat,indec,by='inde')
  }
 dboysE<-explode(dboys,'nfam')
\end{Sinput}
\end{Schunk}
An alternative solution is therefore
\begin{Schunk}
\begin{Sinput}
 m.a<-glm(cbind(boys,girls)~1,data=dboysE,family=binomial)
\end{Sinput}
\end{Schunk}
\end{solution}

\section{Low birth weight in infants}
%
\begin{enumerate}
\item
The data set \verb+birthwt+ contains information about 189 births at a US hospital.
\begin{Schunk}
\begin{Sinput}
 library(MASS)
 data(birthwt)
\end{Sinput}
\end{Schunk}
\item
Use the help page for the data frame to find out what variables the data set contain.
\begin{Schunk}
\begin{Sinput}
 help(birthwt)
\end{Sinput}
\end{Schunk}
\item
Run the example code of the data frame.
This creates the data frame \verb+bwt+ that aggregate levels of
some of the variables of \verb+birthwt+. 
\begin{Schunk}
\begin{Sinput}
 example(birthwt)
\end{Sinput}
\end{Schunk}

\item
We treat 'low' as the response-variable and wish to find the effect of the other variable on
the weight of the children. What type of model will we use?

\item
Examine the effect of the independent variables on the birthweight.
Which can be left out your model.

\item
Make a residual analysis of your fitted model.

\end{enumerate}

\section{Damage of carrots by carrot larvae}
Wheatley and Freeman (1982) analysed an experiment on the impact of two different insecticides, applied in different soil depths, on the damage of carrots by carrot fly larvae. Additionally to the two insecticides two control experiments without applied insecticides were performed. The experiment was repeated three times. 

From each plot on which a treatment had been applied a number of carrots were chosen, washed and classified as damaged or not. 


Reading  data and printing the data
\begin{Schunk}
\begin{Sinput}
 data(carrotfly,package='LiSciData')
\end{Sinput}
\end{Schunk}
\verb+dami+ are the number of damaged carrots in replicate \verb+i+ and \verb+exami+ are the corresponding 
analysed carrots.

\begin{enumerate}
\item
\begin{itemize}
\item
The data are given in wide format 
(all observations for one \verb+insecticide+ $\times$ \verb+depth+ 
combinations are in one row.
Transform the data such that in each row we have the two observations (number of damaged and examined  carrots)
for each replicate. Take care to create a variable \verb+replicate+ that identifies the replicates.
(The example code of the carrotfly-data may be of help (you need to load the package \verb+LiSciData+).
\begin{solution}
\begin{Schunk}
\begin{Sinput}
 library(LiSciData)
 example(carrotfly)
\end{Sinput}
\begin{Soutput}
crrtfl data(carrotfly)

crrtfl #reshape the data into long fromat
crrtfl carrot<-reshape(carrotfly,direction='long',varying=list(c('dam1','dam2','dam3'),
crrtfl                             c('exam1','exam2','exam3')),
crrtfl                       v.names=c('dam','exam'),
crrtfl                       times =c(1,2,3),timevar='replicate')

crrtfl #removing the id variable
crrtfl carrot<-subset(carrot,select=-id)
\end{Soutput}
\end{Schunk}
\end{solution}
\item
Make the variables \verb+replicate+ a factor.
\begin{solution}
\begin{Schunk}
\begin{Sinput}
 carrot<-transform(carrot,replicate=factor(replicate))
\end{Sinput}
\end{Schunk}
\end{solution}
\item
Create a factor variable \verb+alltreat+  has  as levels the combinations of the levels of \verb+insecticide+ and
\verb+depth+. \verb+alltreat+ represents thus a factor variable with 11 levels representing 11 considering all combinations of insecticide and depth as 11 different treatments.  Use the function \verb+interaction+ to perform the task.
The function \verb+interaction+ generates a factor with  levels that consists of all possible combinations of the levels of  \verb+depth+ and \verb+insecticide+ . It will therefore also create level-combinations that are not presented in the data, e.g. the combination \verb+insecticide='control', depth=1+. To get rid of these combinations use the \verb+drop=T+
argument.
\begin{solution}
\begin{Schunk}
\begin{Sinput}
 carrot<-transform(carrot,alltreat=interaction(depth,insecticide,drop=T))
\end{Sinput}
\end{Schunk}
\end{solution}
\end{itemize}
\item
Formulate a model and fit a model, where you analyse the impact of replicate and the 11 different treatments on the damage of
carrots.
\begin{solution}
We assume that the data are binomially distributed.
\begin{Schunk}
\begin{Sinput}
 g<-glm(cbind(dam,exam-dam)~replicate+alltreat,data=carrot,family=binomial)
\end{Sinput}
\end{Schunk}
\end{solution}
\item
Are there signs of overdispersion?
\begin{solution}
\begin{Schunk}
\begin{Sinput}
 X2<-sum(residuals(g,type='pearson')^2)/g$df.residual
 X2
\end{Sinput}
\begin{Soutput}
[1] 5.362905
\end{Soutput}
\end{Schunk}
\end{solution}
\item
Test whether the treatments have an impact on the number of damaged roots.
\begin{solution}
\begin{Schunk}
\begin{Sinput}
 g.quasi<-glm(cbind(dam,exam-dam)~replicate+alltreat,data=carrot,family=quasibinomial)
 anova(g.quasi,test='F')
\end{Sinput}
\begin{Soutput}
Analysis of Deviance Table

Model: quasibinomial, link: logit

Response: cbind(dam, exam - dam)

Terms added sequentially (first to last)


          Df Deviance Resid. Df Resid. Dev      F    Pr(>F)    
NULL                         35    1824.41                     
replicate  2    64.69        33    1759.73  6.031  0.007839 ** 
alltreat  10  1636.75        23     122.97 30.520 6.417e-11 ***
---
Signif. codes:  0 '***' 0.001 '**' 0.01 '*' 0.05 '.' 0.1 ' ' 1 
\end{Soutput}
\end{Schunk}
or
\begin{Schunk}
\begin{Sinput}
 g.quasi.0<-glm(cbind(dam,exam-dam)~replicate,data=carrot,family=quasibinomial)
 anova(g.quasi,g.quasi.0,test='F')
\end{Sinput}
\begin{Soutput}
Analysis of Deviance Table

Model 1: cbind(dam, exam - dam) ~ replicate + alltreat
Model 2: cbind(dam, exam - dam) ~ replicate
  Resid. Df Resid. Dev  Df Deviance     F    Pr(>F)    
1        23     122.97                                 
2        33    1759.73 -10  -1636.8 30.52 6.417e-11 ***
---
Signif. codes:  0 '***' 0.001 '**' 0.01 '*' 0.05 '.' 0.1 ' ' 1 
\end{Soutput}
\end{Schunk}
\end{solution}
\item
Test whether  the effect of the insecticides are different for the depth 1.
\item
Test the null-hypothesis: The effects of the insecticides are not different on any level of
depth?

\begin{solution}

This hypothesis is formulated by several contrasts to be zero at the same time.
If the model is
\[
logit(\pi)= \mu + \alpha_{id} + \beta_r
\]
then the hypothesis of no insecticide effect is
\[
\alpha_{diazonon,d}-\alpha_{disulfoton,d}=0
\]
for all   $d=1,2,5,5,10,25$. (Why is $d=0$ not included?)
You can use the \verb+esticon+ function of \verb+doBy+ package to test this hypothesis:
 The parameter estimates are
\begin{Schunk}
\begin{Sinput}
 coef(g.quasi)
\end{Sinput}
\begin{Soutput}
           (Intercept)             replicate2             replicate3 
             1.6341642              0.5031748              0.5863955 
    alltreat1.diazinon   alltreat2.5.diazinon     alltreat5.diazinon 
            -1.1202164             -2.1065815             -3.6135831 
   alltreat10.diazinon    alltreat25.diazinon   alltreat1.disulfoton 
            -4.0302763             -1.9878067             -1.4707183 
alltreat2.5.disulfoton   alltreat5.disulfoton  alltreat10.disulfoton 
            -1.6703448             -2.2139604             -2.8947688 
 alltreat25.disulfoton 
            -1.1735287 
\end{Soutput}
\end{Schunk}
the contrast vector for the difference 
\[
\alpha_{diazonon,d}-\alpha_{disulfoton,d}=0
\]
would be
\begin{Schunk}
\begin{Sinput}
 lamb1<-lamb2<-lamb5<-lamb10<-lamb25<-0*coef(g.quasi)
 lamb1[c('alltreat1.diazinon','alltreat1.disulfoton')]<-c(-1,1)
 lamb2[c('alltreat2.5.diazinon','alltreat2.5.disulfoton')]<-c(-1,1)
 lamb5[c('alltreat5.diazinon','alltreat5.disulfoton')]<-c(-1,1)
 lamb10[c('alltreat10.diazinon','alltreat10.disulfoton')]<-c(-1,1)
 lamb25[c('alltreat25.diazinon','alltreat25.disulfoton')]<-c(-1,1)
 lamb<-rbind(lamb1,lamb2,lamb5,lamb10,lamb25)
\end{Sinput}
\end{Schunk}
and using the esticon function with the matrix $lamb$ as contrast matrix:
\begin{Schunk}
\begin{Sinput}
 esticon(g.quasi,cm=lamb,joint.test=T)
\end{Sinput}
\begin{Soutput}
   X2.stat DF   Pr(>|X^2|)
1 37.95923  5 3.845064e-07
\end{Soutput}
\end{Schunk}
\end{solution}
\item
Assume now that the transformed data $\text{logit}(dam/exam)$ 
are normally distributed.
Fit a model with the same covariates as above and compare the parameter estimates and their standard errors.
\end{enumerate}


\section{Esophageal cancer}


\begin{enumerate}

\item
The data set \verb+esoph+ contains records from a case-control study of esophageal cancer.
For 88  combinations of age , alcohol consumption and tobacco consumption the number of cases
and controls are given.
\begin{Schunk}
\begin{Sinput}
 data(esoph)
\end{Sinput}
\end{Schunk}

\item
Make some descriptive plots of data.

\item
Formulate a model for your data.

\item
Fit the model with interactions between all three predictors.

\item
Eliminate factors to simplify the model as far as is possible.

\item
All three factors are ordered. Convert the factors to numerical 
variables using the \verb+recodevar+ of the package \verb+doBy+.

\item
Can the model be simplified using the numerical representation of the predictors?

\item
Analyse the residuals of your final model.
\end{enumerate}
\section{Discoveries}

\begin{enumerate}
\item
The data \verb+discoveries+ lists the number of great inventions and scientific discoveries in each of the years between 1860 to 1959.
\begin{Schunk}
\begin{Sinput}
 data(discoveries)
\end{Sinput}
\end{Schunk}

\item
Plot the data.

\item
Add a smooth line to the plot.

\item
How would you model data. Write down the systematic and the random part of your model.

\item
Has the discovery rate remained constant over time?

\end{enumerate}


\section{Bacteria}
We reanalyze data we used already in the last project day.
In an experiment on nutrients on bacterial growth,
bacteria were grown in different media with the nutrients sucrose and leucine added.
After fours days the numbers of bacteria, \verb+density+, were counted.
\begin{Schunk}
\begin{Sinput}
 library(LiSciData)
 data(bactsucrose)
\end{Sinput}
\end{Schunk}

\begin{enumerate}
\item
Assume the responses \verb+density+ to be normally distributed.
Assume the predictors \verb+day+, \verb+sucrose+ and \verb+leucine+ as factors.
Consider  a model which is  additive in \verb+day+ and with an interaction between the other two
factors.  
Which link function would the Box-Cox approach suggest?
\item
Fit the corresponding model  using the identified link assuming the data to be normally distributed.
\item
We want to identify an appropriate variance function.
Take the absolute residuals \verb+abs(res)+ (on the response scale) from the above model 
and the predicted values \verb+fit+ (on the response scale) and fit the linear model
\[
E(\log(|res|))= \phi + \lambda \log (fit)
\]
Twice the slope of the fitted equation will give the exponent $\lambda$ in the variance function
\[
V(\mu)= \phi \mu^\lambda
\]
\item
Use the estimated $2 \cdot \lambda$ to propose a different distribution for the observations. Fit this new model.
\item
Check whether you still need the interaction between sucrose and leucine.
\item
Fit two models with the \verb+log+ link assuming either normally or gamma distributed data.
Use AIC and cross-validation to decide which model to choose.

\end{enumerate}


\begin{solution}
\begin{Schunk}
\begin{Sinput}
 bactsucrose<-transform(bactsucrose,y=density/max(density),day=factor(day),sucrose=factor(sucrose),leucine=factor(leucine))
 library(MASS)
 boxcox(y~day+sucrose+leucine+sucrose:leucine,data=bactsucrose)
 m.normal.log<- glm(density~day+sucrose+leucine+sucrose:leucine,data=bactsucrose,family=gaussian(link=log))
 res<-residuals(m.normal.log,type='response')
 fit<-predict(m.normal.log,type='response')
 g<-glm(log(abs(res))~log(fit))
 m.normal.log<- glm(density~day+sucrose+leucine+sucrose:leucine,data=bactsucrose,family=gaussian(link=log))
 m.gamma.log<-glm(density~day+sucrose+leucine+sucrose:leucine,data=bactsucrose,family=Gamma(link=log))
 library(boot)
 AIC(m.normal.log)
 AIC(m.gamma.log)
 cv.glm(bactsucrose,m.normal.log)$delta
 cv.glm(bactsucrose,m.gamma.log)$delta
\end{Sinput}
\end{Schunk}

\end{solution}

 
\section{Criminal teens and Poisson distribution}
Sometimes the Poisson distribution can explain a phenomenon just as a random phenomenon,
making more intricate argumentation superfluous.
In an article from 17. December 2007, Politiken wrote 
{\it 
{\bf Violent teens live in small  villages  - not the large city}
It was beautiful in the countryside. In old days. Today the small villages as
Fars�, Bredebro, Aarup and Holsted are on the top if the number of
 sentences for violence  for  persons aged between 15 and 17  is counted.
If the number of young persons that are involved in 
violence is related to the number of teens in their community,
then the largest towns are ranked very low with Copenhagen at number 81
and Aarhus as number 125 of the 272 communities in 2006.
An explanation could be that the small communities have the cheapest accommodations.
}
A sociologist confronted with these data commented, that the findings were surprising.

We want to find out whether the data provide evidence for  
something surprising going on or 
whether a simple assumption of equal rates based on  Poisson counts would explain the data.

\begin{enumerate}
\item
First we make a little theoretical exercise.
Assume we have 3 councils with the following population of teens:
\begin{Schunk}
\begin{Sinput}
 pop<-rep(c(10,100,1000),each=c(1,1,1))
\end{Sinput}
\end{Schunk}
We assume further that each community has the same problem with violent teens,
i.e. we assume that all towns have the same rate of violent teens.
Let us assume the violence-rate is 3\%, e.g. 3 violent per 100 teens.
We assume that the number of violent teens is Poisson distributed and
generate for each town the number of violent teens according to this assumption
A single replication would be coded as:
\begin{Schunk}
\begin{Sinput}
 b<-rpois(length(pop),lambda=0.03*pop)
\end{Sinput}
\end{Schunk}
We repeat this 1000 times
\begin{Schunk}
\begin{Sinput}
 set.seed(89)
 b<-matrix(NA,length(pop),1000)
 for (i in 1:1000) {
  b[,i]<-rpois(length(pop),lambda=0.03*pop)
  }
\end{Sinput}
\end{Schunk}
Now we calculate for each town the  proportion of violent teens for each of our replication
\begin{Schunk}
\begin{Sinput}
 b<-b/pop
\end{Sinput}
\end{Schunk}
We plot the distribution of the proportion for each town
\begin{Schunk}
\begin{Sinput}
 par(mfrow=c(2,2))
  hist(b[1,],xlim=c(0,0.25),ylim=c(0,65),main='village',probability=TRUE)
  hist(b[2,],xlim=c(0,0.25),ylim=c(0,65),main='town',probability=TRUE)
  hist(b[3,],xlim=c(0,0.25),ylim=c(0,65),main='large city',probability=TRUE)	
\end{Sinput}
\end{Schunk}
\includegraphics{fig/PRACS-031}
 Looking at the plots, which type of city (i.e. village, town, large city) do you think will most probably
be 
\begin{enumerate} 
\item
 among those with the smallest proportions of violent teens,
\item
among those with the largest proportions of violent teens,
\item
among those with a median amount of violent teens?
\end{enumerate}
Relate your conclusion to that of the article that the violent live in small but not in large cities.

\item
We analyze now the data. We consider only the non-aggregated data, i.e. we do not use the 
data for the 'Amt's and for 'Hele landet'.
Additionally we exclude all data with a ratio of 0 because for these data we cannot calculate the number of
teens aged between 15 and 17 in a city.

Preparation of the data, the rows for 'Amt' and 'Hele-landet' are deleted and the number of
teens aged 15 to 17 in each town is calculated.
\begin{Schunk}
\begin{Sinput}
 v<-get(data(poissonviolence,package='LiSciData'))
 v$pop<-with(v,number/ratio*1000)
 is.amt<-grep('Amt',v$kommune)
 v<-v[-is.amt,]
 v<-subset(v, !(kommune %in% c('Hele_landet')))
\end{Sinput}
\end{Schunk}

\begin{enumerate}
\item
Fit now two alternative models to the data
\begin{enumerate}
\item
A simple Poisson model to the data, assuming a common violence rate for each community.
Consider the necessity to account for overdispersion.
\item
A model, where the violence rate is proportional to the
size of the town
\end{enumerate}
\item
Are the two models statistically different?
\item
What is the likely  consequence that we left out the towns where the
ratio of violent teens had been estimated to zero?
\end{enumerate} 


\begin{solution}
A proposal solution to the model fitting

\begin{Schunk}
\begin{Sinput}
 M0<-glm(number~offset(log(pop)),data=v,family=poisson)
 X2pearson<-function(m) sum(residuals(m,type='pearson')^2)/m$df.residual
 X2pearson(M0)
\end{Sinput}
\begin{Soutput}
[1] 1.774907
\end{Soutput}
\begin{Sinput}
 M1<-glm(number~offset(log(pop)),data=v,family=quasipoisson)
 M2<-glm(number~log(pop)+offset(log(pop)),data=v,family=quasipoisson)
\end{Sinput}
\end{Schunk}
\end{solution}

\end{enumerate}

\section{Survial data}
The data in set \verb+whitebloodcell+  of the package \verb+LiSciData+ contain the 
survival time (in weeks) of patients after the diagnosis of
leukemia. Also the logarithm to the base 10 of the
counts of the white blood cells at the time of diagnoses
is given.

Survival times can sometimes be assumed to be exponentially
distributed.

The exponential distribution is a special case of the gamma
distribution with the dispersion parameter (the reciprocal of the
shape parameter) equal to 1.
The density if the Gamma distribution is
\[
f(y)= \frac{1}{s^a \Gamma(a)} y^{a-1} e^{-\frac{y}{s}}
\]
with $s>0$ the scale parameter, $a>0$ the shape parameter.

Setting $a=1$ and defining the rate $\lambda=1/s$ we get the most common
form of the density for the exponential distribution
\[
f(y)= \lambda e^{-\lambda y}.
\]

\begin{enumerate}
\item
What is the expectation and the variance of an exponential
distribution. (HINT: Refer to the lecture notes for the Gamma distribution)

\begin{solution}
The Gamma distribution has expectation and variance
\[
\EE(Y)=a\cdot s  \quad \var(Y)=a\cdot s^2
\]
With $a=1$ and $s=\frac{1}{\lambda}$
we have for an exponentially distributed random variable $Y$
\[
\EE(Y)=1/\lambda  \quad  \var(Y)= 1/\lambda^2
\]
\end{solution}


\item
Draw the density of the exponential distribution with
the expectation of 4.7. (HINT: use the R-function \verb+dexp+).

\begin{solution}
\begin{Schunk}
\begin{Sinput}
 x<-seq(0,20,l=50)
\end{Sinput}
\end{Schunk}
\begin{Schunk}
\begin{Sinput}
 plot(x,dexp(x,rate=1/4.7),type='l')
\end{Sinput}
\end{Schunk}
\includegraphics{fig/PRACS-035}
\end{solution}


\item
Assume that the expected survival times may be described by
\[
\EE(Y_i|x_i)= \exp(\alpha + \gamma x_i)
\]
where $x_i$ are the $\log_{10}$ counts of the white blood cells.

Fit an appropriate GLM (write down the formula for the linear
 predictor)
 and give an estimate for the parameters and
their standard errors.

Be aware of the following: The estimation of the parameters of the
linear predictor of a GLM is independent of the estimate for
the dispersion. This is reflected in R by the fact that a \verb+glm+
object has no dispersion attribute. But if you you use the
\verb+summary+
function, the dispersion parameter is either estimated by default via
the Pearson's $X^2$ or you may specify it.

Because we assume that our data are exponentially distributed, you must choose 
the dispersion equal to 1 (remember that the dispersion is the reciprocal of the shape parameter: $\phi=1/a$)!

\begin{solution}
The model is linear on the log-scale, we use therefore the
\verb+log+-link:
\[
\log(\EE(Y_i|x_i))= \alpha + \gamma x_i
\]

\begin{Schunk}
\begin{Sinput}
 library(LiSciData)
 data(whitebloodcell)
 g<-glm(time~logcount,data=whitebloodcell,family=Gamma(link=log))
\end{Sinput}
\end{Schunk}
 
ATTENTION: We must explicitely specify the \verb+log+-link for the
Gamma family, because R assumes by default the canonical link for the
Gamma distribution, i.e. the inverse function. 

\begin{Schunk}
\begin{Sinput}
 summary(g,dispersion=1)$coef
\end{Sinput}
\begin{Soutput}
             Estimate Std. Error   z value     Pr(>|z|)
(Intercept)  8.343253   1.608815  5.185960 2.149046e-07
logcount    -1.080825   0.389811 -2.772689 5.559521e-03
\end{Soutput}
\end{Schunk}
\end{solution}


\item
What is the expected life time  for patients with 
a $\log_{10}$ cell count of 4.5? (HINT: Use the \verb+predict+ function).


\begin{solution}
\begin{Schunk}
\begin{Sinput}
 predict(g,newdata=data.frame(logcount=4.5),type='response')
\end{Sinput}
\begin{Soutput}
       1 
32.44487 
\end{Soutput}
\end{Schunk}
\end{solution}


\item
Provide a confidence interval for this expected
lifetime. (HINT: Use the \verb+se+ argument of the \verb+predict+ function
and set the \verb+dispersion+ argument to 1).

\begin{solution}
\begin{Schunk}
\begin{Sinput}
 e<-predict(g,newdata=data.frame(logcount=4.5),dispersion=1,se=TRUE)
\end{Sinput}
\end{Schunk}


\begin{Schunk}
\begin{Sinput}
 ci.log<-e$fit+ 1.96* e$se.fit *c(-1,1)
\end{Sinput}
\begin{Soutput}
[1] 2.906002 4.053082
\end{Soutput}
\begin{Sinput}
 ci<-exp(ci.log)
\end{Sinput}
\begin{Soutput}
[1] 18.28355 57.57466
\end{Soutput}
\end{Schunk}
If we repeat the calculation of the confidecne interval, but
incorporate the dispersion into its calculation, we get a  narrower interval
(Note that I do NOT set the value of dispersion)
\begin{Schunk}
\begin{Sinput}
 e<-predict(g,newdata=data.frame(logcount=4.5),se=TRUE)
 ci.log<-e$fit+ 1.96* e$se.fit *c(-1,1)
 ci<-exp(ci.log)
\end{Sinput}
\end{Schunk}
\begin{Schunk}
\begin{Sinput}
 ci<-exp(ci.log)
\end{Sinput}
\begin{Soutput}
[1] 18.57472 56.67214
\end{Soutput}
\end{Schunk}

\end{solution}
\end{enumerate}

\section{Insurance claims}

Install the CRAN-package \verb+faraway+.  In the
\verb+faraway+-package you find the data set \verb+motorins+ which
contain the number of car insurance claims for 1797 groups of cars in
Sweden in 1977.

\begin{Schunk}
\begin{Sinput}
 data(motorins,package='faraway')
\end{Sinput}
\end{Schunk}


We want examine the effect of Zone, Make and  on the number of claims.

\begin{enumerate}
\item
Formulate a model to solve the problem.
Which variable could be used as an offset in this model?
Check your model.

\begin{Schunk}
\begin{Sinput}
 g.quasi<-glm(cbind(dam,exam-dam)~replicate+alltreat,data=carrot,family=quasibinomial)
 carrot$p<-carrot$dam/carrot$exam
 g<-glm(log(p/(1-p))~replicate+alltreat,data=carrot,family=gaussian)
\end{Sinput}
\end{Schunk}

\item
Compare your previous analysis to an analysis where you assume that
the data a log-normally distributed.
\begin{enumerate}
\item
Fit a normal linear model for the
log-transformed data (i.e. assume that the survival times
are log-normally distributed)


\begin{solution}

\begin{Schunk}
\begin{Sinput}
 g.lognormal<-lm(log(time)~logcount,data=whitebloodcell)
\end{Sinput}
\end{Schunk}
\end{solution}

\item
Based on your model, estimate the mean life time of patients
with 
a $\log_{10}$ cell count of 4.5?.
(HINT: predict first the expected $\log$-survival time, and transform
this to the original scale using the formula given in the lecture notes.)

What is your conclusion from comparing the two ways of analysis for these data?


\begin{solution}
\begin{Schunk}
\begin{Sinput}
 exp(
  predict(g.lognormal,newdata=data.frame(logcount=4.5))+
  0.5*summary(g.lognormal)$sigma^2)
\end{Sinput}
\begin{Soutput}
       1 
28.29349 
\end{Soutput}
\end{Schunk}
\end{solution}
\end{enumerate}


\end{enumerate}


\section{Weight loss under training}
\begin{topics}
  Generalized linear models; gee
\end{topics}

Twenty men participated in a 'waist loss' program where there initial weight and their weight after 
completion of the program were measured. The question is, whether the program had an effect.
\begin{Schunk}
\begin{Sinput}
 library(LiSciData)
 data(waistloss)
\end{Sinput}
\end{Schunk}

\begin{enumerate}
\item
Plot the weights against the before/ after factor combing the points for each man.
\begin{solution}
\begin{Schunk}
\begin{Sinput}
 library(lattice)
 waistL<-reshape(waistloss,direction='long',varying=list(c('before','after')),v.names='weight',time=c('before','after'),
  timevar='time')
 waistL$timenum<-with(waistL,ifelse(time=='before',-1,1))
 xyplot(weight~timenum,groups=id,type='l',data=waistL)
\end{Sinput}
\end{Schunk}
\end{solution}
\item 
Formulate and fit a model where you assume that all observations are independent.
\item
Formulate and fit an appropriate model to the data taking the correlation between the repeated measurements into account.
\item
Do you think that the working correlation \verb+exchangeable+ and \verb+ar1+ could yield different results?
\item 
Compare the estimates and standard errors of the two previous model fits.
\item
Analyze the difference of the weights for each man. Compare the results to your previous analysis.
The test you perform here is also known as a 'paired t-test'. (You can try to use the R-function
\verb+t.test+. The result should be the same as for the analysis of the differences).

\begin{solution}
\begin{Schunk}
\begin{Sinput}
 library(geepack)
 waistL<-with(waistL,waistL[order(id,time),])
 m.un<-glm(weight~time,data=waistL)
 m.gee<-geeglm(weight~time,data=waistL,id=id,corstr='exchangeable')
 waistloss$diff<-with(waistloss,before-after)
 m.diff<-glm(diff~1,data=waistloss)
\end{Sinput}
\end{Schunk}
\end{solution}

\end{enumerate}
\section{Sitka Spruce trees}
\begin{topics}
  Generalized linear models
\end{topics}
In a study on the effect of ozone on the growth of trees, 54 sitka 
spruce trees
 were grown in an atmosphere enriched with ozone, and 25 trees were grown under normal conditions.
The 79 trees of the same age were 
randomly assigned to the two groups at the beginning of the growth experiment.
 

In 1989, the second year of growth, the size of the trees was measured
 at roughly
     monthly intervals.  

The data set \verb+sitka89+ in our data library \verb+LiSciData+ 
contains the size of the trees measured as $\log(h*d^2)$ 
where $h$ is the height of a tree and $d$ its diameter, 
 the time of measurement in days 
after January, 1st, 1988, an identification number  for each tree and
the indicator of the treatment.




\begin{enumerate}
\item
Plot the growth curves for all trees and the mean of the growth curves for the ozone and the control group
\item
Propose and fit a model assuming that all observations are independent.
Use the time-variable as a factor.
\item 
Fit a model taking into account the repeated measurements on the same tree.
\item
Compare the parameter estimates and their standard errors of the two fits.
\item
Test the null-hypothesis that there is no effects of ozone.
\item
Provide the effect-estimates for each tree.
\item
Make some residual plots.
\end{enumerate}
\section{Influence of method of anaesthetic gas application.}
\begin{topics}
  Generalized linear Models
\end{topics}

In a study the effect of  two different airway-masks (classical FM and experimental LMA) on the post-operative
experience of sore throat was analyzed. 
Data from surgeries with 10 consultants (anaethesists) were collected 
on different patients.  

The data are available as  as
\begin{Schunk}
\begin{Sinput}
 data(soreThroat,package='LiSciData')
\end{Sinput}
\end{Schunk}

\begin{enumerate}
\item
Plot the proportion of patients experiencing sore throat against the type of airway mask.
\item
Formulate and fit a model to the data assuming independence of the observations.
\item
Formulate and fit a random intercept model taking account of the correlation between observations from the same consultant.\item
 Provide parameter estimates and their confidence intervals.
\item
Test the null- hypothesis that the two  airway masks are  equivalent.
\item
Give the estimates for the consultant effects.
\end{enumerate}


\section{Damage of carrots by carrot fly larvae}

\begin{topics}
  Generalized linear models; link function; variance function
\end{topics}


In a cross-sectional study the relation between plasma retinol and person 
characteristics and dietary factors were collected. Low level of plasma retinol
is suspected ti be related to increase risk of cancer.


The data \verb+plasmaRetinol+ (removing the one observations with zero
plasma concentration): 
\begin{Schunk}
\begin{Sinput}
 v<-get(data(plasmaRetinol,package='LiSciData'))
 v<-subset(v,betaplasma>0)
\end{Sinput}
\end{Schunk}

\begin{Schunk}
\begin{Sinput}
 v<-transform(v,sex=factor(sex,labels=c('M','F')),
  smokstat=factor(smokstat,labels=c('nev','for','cur')),
  vituse=factor(vituse))
 names(v)[names(v)=='beteadiet']<-'betadiet'
\end{Sinput}
\end{Schunk}
\begin{enumerate}
\item
Fit the following three models to the response \verb+betaplasma+
 and the independent factors
\verb+vituse+,\verb+sexp+,\verb+age+,\verb+kcal+,\verb+fat+,\verb+fat+,
\verb+alcohol+,\verb+cholestorol+\verb+betadiet+.
Decide which variables should be factors and which variables should be
covariates.
\begin{enumerate}
\item
response is normally distributed, link is the identity
\item
response is normally distributed, link is the logarithm,
\item
response is Gamma distributed, link is the logarithm
\end{enumerate}
Which model seems to fit best?

\begin{Schunk}
\begin{Sinput}
 m.normal.id<-glm(betaplasma~vituse+sex +age + kcal+fat+fiber+alcohol+cholesterol+betadiet,data=v)	
 #
 m.normal.log<-glm(betaplasma~vituse+sex +age + kcal+fat+fiber+alcohol+cholesterol+betadiet,data=v,family=gaussian(link=log))	
 #
 m.gamma.log<-glm(betaplasma~vituse+sex +age + kcal+fat+fiber+I(fiber^2)+alcohol+cholesterol+betadiet,data=v,family=Gamma(link=log))
\end{Sinput}
\end{Schunk}

\item
In the third model, make q-q-plots of the  Pearson and the deviance residuals, What do you observe?

\item
Fit a generalized additive model (using the package \verb+mgcv+), using smooth terms for all covariates.
Would you need to enhance your model?
\begin{Schunk}
\begin{Sinput}
 library(mgcv)
 mm<-gam(betaplasma~vituse+sex +s(age) + s(kcal)+s(fat)+s(fiber)+s(alcohol)+
  s(cholesterol)+s(betadiet),
         data=v)	
\end{Sinput}
\end{Schunk}
\end{enumerate}
\section{Fruitfly}
\begin{topics}
  Generalized linear models; link function; variance function
\end{topics}


In study on the impact of sexual activity on lift-time of male fruit-flies,
three groups with different degrees of sexual activity were measured.
Additionally the thorax-length of males was measured because it is known to have an impact on longevity.

\begin{Schunk}
\begin{Sinput}
 data(fruitfly,package='faraway')
 sapply(fruitfly,class)
\end{Sinput}
\begin{Soutput}
   thorax longevity  activity 
"numeric" "integer"  "factor" 
\end{Soutput}
\end{Schunk}



Formulate
and fit an appropriate model to the data.
\begin{enumerate}
\item
Start with a simple normal modal using the identity link
\item
use a model where the variance function is quadratic in the mean.
\item
use the preceding model bu use now a log-link.
(Because \verb+longevity+ is a measure of lifetime, 
one often analysis the logarithm of such positive data).
\end{enumerate}

Use residual analysis and the AIC-criterion to find an appropriate model.

What would the conclusion be if one would not use the thorax-length in the analysis?

 
\section{Bacteria}
In an experiment on nutrients on bacterial growth,
bacteria were grown in different media with the nutrients sucrose and leucione add3ed.
After fours days the number of bacteria \verb+density+were counted.
\begin{Schunk}
\begin{Sinput}
 v<-get(data(bactsucrose,package='LiSciData'))
 v$y<-with(v,density/quantile(density,0.9))
\end{Sinput}
\end{Schunk}

Fit following models
\begin{enumerate}
\item
Assume the responses to be normally distributed and use a identity link.
Fit the maximal model.
\item
Take the absolute residuals \verb+abs(res)=+ from the above model and the predicted values \verb+fit+ and fit the linear model
\begin{verbatim}
lm(log(abs(res))~log(fit))
\end{verbatim}
Twice the slope of the fitted equation will give the exponent $\lambda$ in the variance function
\[
V(\mu)= \phi \mu^\lambda
\]
Use the estimated $\lambda$ to improve your model-fit.
\item
Check whether you still need the interaction between sucrose and leucine.
\item
Try in your model the \verb+log+ and the \verb+identity+-link.
Use AIC and cross-validation to decide which model to choose.
\begin{Schunk}
\begin{Sinput}
 m.gamma.log<-glm(density~day+sucrose+leucine,data=v,family=Gamma(link=log))
 m.gamma.id<-glm(density~day+sucrose+leucine,data=v,family=Gamma(link=identity))
\end{Sinput}
\end{Schunk}


\begin{Schunk}
\begin{Sinput}
 library(boot)
 extractAIC(m.gamma.id)
\end{Sinput}
\begin{Soutput}
[1]    4.000 2074.988
\end{Soutput}
\begin{Sinput}
 extractAIC(m.gamma.log)
\end{Sinput}
\begin{Soutput}
[1]    4.000 2043.355
\end{Soutput}
\begin{Sinput}
 cv.glm(v,m.gamma.id)$delta
\end{Sinput}
\begin{Soutput}
[1] 1.013426e+19 1.010748e+19
\end{Soutput}
\begin{Sinput}
 cv.glm(v,m.gamma.log)$delta
\end{Sinput}
\begin{Soutput}
[1] 6.007113e+18 5.904827e+18
\end{Soutput}
\end{Schunk}
\end{enumerate}

\section{Hodkin}
\begin{topics}
  Generalized linear models; link function; variance function
\end{topics}

\begin{Schunk}
\begin{Sinput}
 library(LiSciData)
 data(hodkin)
 v<-reshape(hodkin,direction='long',varying=list(c('hodk','nonhodk')),v.names='number',timevar='hodkin',times=c('yes','non'))
 v$hodkin<-factor(v$hodkin)
\end{Sinput}
\end{Schunk}

\begin{enumerate}
\item
Formulate and fit an appropriate model.Check for overdispersion.
Does the q-q-plot of the standardized residuals tell you something 
about overdispersion?
\begin{Schunk}
\begin{Sinput}
 m<-glm(number~hodkin,data=v,family=poisson)
 m.ov<-glm(number~hodkin,data=v,family=quasipoisson)
\end{Sinput}
\end{Schunk}
\item
Fit a model assuming the negative binomial distribution. Use the
\verb+glm.nb+ function of the package \verb+MASS+
\begin{Schunk}
\begin{Sinput}
 library(MASS)
 m.neg<-glm.nb(number~hodkin,data=v)
\end{Sinput}
\end{Schunk}
\item
Fit a model assuming the data to be normally distributed and
us the log-ling.
 Compare the coefficients, the predicted values and the AIC's of
all your fitted models.
\item
Fitting the model with the normal assumption,  is there a difference 
in AIC using the log or the identity link? 
\end{enumerate}



\end{document}

















