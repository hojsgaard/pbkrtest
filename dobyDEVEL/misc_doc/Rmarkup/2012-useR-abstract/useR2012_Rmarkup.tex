\documentclass[11pt, a4paper]{article}
\usepackage{amsfonts, amsmath, hanging, hyperref, parskip, times}
\usepackage[numbers]{natbib}
\usepackage[pdftex]{graphicx}
\hypersetup{
  colorlinks,
  linkcolor=blue,
  urlcolor=blue,
  citecolor=blue
}

\let\section=\subsubsection
\newcommand{\pkg}[1]{{\normalfont\fontseries{b}\selectfont #1}} 
\let\proglang=\textit
\let\code=\texttt 
\renewcommand{\title}[1]{\begin{center}{\bf \LARGE #1}\end{center}}
\newcommand{\affiliations}{\footnotesize\centering}
\newcommand{\keywords}{\paragraph{Keywords:}}

\setlength{\topmargin}{-15mm}
\setlength{\oddsidemargin}{-2mm}
\setlength{\textwidth}{165mm}
\setlength{\textheight}{250mm}

\begin{document}
\pagestyle{empty}

\title{\code{Rmarkup} -- a very simple tool for literate programming}

\begin{center}
  {\bf S{\o}ren H{\o}jsgaard$^{1,^\star}$}
\end{center}

\begin{affiliations}
1. Department of Mathematical Sciences, Aalborg University, Denmark
$^\star$Contact author: \href{mailto:sorenh@math.aau.dk}{sorenh@math.aau.dk}\\
\end{affiliations}

\keywords markup, script file %, First, Second, $\ldots$ up to 5 keywords

\vskip 0.8cm

The \code{Rmarkup} function in the \pkg{doBy} package does the
following: Descriptive text and \proglang{R} code is put into a source
document. The target document created by weaving will contain the
descriptive text and program code together with graphics and results
from the computations.

The format of the source document is a plain text file containing
\proglang{R} and descriptive text (in lines starting with \#\#).
\code{Rmarkup} allows some markup facilities for the text. These are
inspired by \code{txt2tags} markups (see \url{http://txt2tags.org/}).
For example, {\bf boldface}; {\it italics}; \underline{underline} and
\code{monospace} is produced with \verb'**'boldface\verb'**';
\verb'//'italics\verb'//'; \verb'__'underline\verb'__' and
\verb'&&'monospace\verb'&&'. Moreover, different levels of headings
are produced with 
\verb'= Title level 1 ='; 
\verb'== Title level 2=='; and so on. 
The target document in an HTML
document containing the descriptive text (with possible markups), and
program code together with graphics and results from the computations.
\code{Rmarkup} is implemented by using the
\code{RweaveHTML} driver in the \pkg{R2HTML}, \citep{R2HTML}.


A natural question is what \code{Rmarkup} offers that can not be
accomplished using tools like \code{Sweave} \citep{Sweave} and
\pkg{odfWeave} \citep{odfWeave}. In terms of functionality,
\code{Rmarkup} is nowhere nearly as advanced as  \code{Sweave} and
\pkg{odfWeave}. In terms of simplicity of installation and use,
\code{Rmarkup} has advantages: 
\code{Rmarkup} grew out of teaching \proglang{R} to graduate students
and others in the life sciences, e.g.\ students with a background in
agronomy, biology, food science, veterinary science etc.  Such
students typically use Microsoft Office in their work and such
students are generally hesitant to having to install to much software
on their computer. We have found that requesting the students to
install \proglang{R} itself and possibly also a suitable editor (we
have recommended \proglang{Notepad++} for Windows users) is about as
much as we can ask of the students. (Needless to say that it is
pointless to ask such students to learn \LaTeX.)
\code{Rmarkup} provides a tool for
such \proglang{R} users.

We have found an additional value of \code{Rmarkup}: It is common to
have a plain text file as a sandbox for playing around with e.g.\ data
manipulation tasks. One may subsequently choose to form a \LaTeX file
with these data manipulation steps and a textual description of steps
using \code{Sweave}. In practice this means that one has a script file
and a \LaTeX file describing essentially the same tasks, and hence
there is a risk of things being out of sync. Using \code{Rmarkup} in
connection with e.g.\ data manipulation tasks, there is only a simple
script file in play and it is a manageable task to add useful
comments and some simple text markup to such a file. This means that
the task of documenting the work (in particular the more tedious parts
of the work) is more likely to be done.




% Some suggestions and rules: if you mention a programming language like
% \proglang{R}, typeset the language name with the {\tt \textbackslash
%   proglang\{\}} macro. If you mention an \proglang{R} function
% \code{foo}, typeset the function name with the with the
% {\tt\textbackslash code\{\}} macro. If you mention an \proglang{R}
% package \pkg{fooPkg}, typeset the package name with the
% {\tt\textbackslash pkg\{\}} macro. Textual ({\it e.g.}, \citet{ref1}
% jumped over the fence.) and parenthetical ({\it e.g.}, The fence was
% jumped \citep{ref1}.) citations may appear within the
% abstract. Itemized lists are allowed in abstracts, but may be wasteful
% of space, which is {\it strictly limited}. Avoid itemized lists if
% possible, but gracefully. {\bf Abstracts should not exceed one US
%   letter (8.5 x 11 inches) page}. The page should not be numbered.

%% The \proglang, \code, and \pkg macros may be reused


%% references: 
%%\nocite{ref1,ref2,ref3}
\bibliographystyle{chicago}
\bibliography{useR2012_Rmarkup}

%% references can alternatively be entered by hand
%\subsubsection*{References}

%\begin{hangparas}{.25in}{1}
%AuthorA (2007). Title of a web resource, \url{http://url/of/resource/}.

%AuthorC (2008a). Article example in proceedings. In \textit{useR! 2008, The R
%User Conference, (Dortmund, Germany)}, pp. 31--37.

%AuthorC (2008b). Title of an article. \textit{Journal name 6}, 13--17.
%\end{hangparas}

\end{document}
