\documentclass{article}

\special{html: 
  <link title="main" media="all" rel="stylesheet" href="http://people.math.aau.dk/~sorenh/css/shdcss.css"
 type="text/css" />
}


\special{html: 
  <img style="width: 100px; height: 76px;" src="logo.jpg" alt="">
}

\def\doby{\href{http://cran.r-project.org/web/packages/doBy/index.html}{\texttt{doBy}}}
\def\doBy{\doby}
\def\R{\href{http://www.R-project.org}{\texttt{R}}}
\def\htmlrep{\texttt{Rmarkup()}}
\def\code#1{\texttt{#1}}

\begin{document}

\section*{The \doby\ package}

\begin{itemize}
\item The \doby\ package is (yet another) \R\ package with utility
functions.
\item \doby\ originally grew out of a need to calculate
groupwise summary statistics, which explains the origin of the name:
\begin{quote}
``\underline{do} something when data is grouped \underline{By} some
variables''.  
\end{quote}
\item The facilities for groupwise computations in \doby\ were relatively popular in the years 2005-2010
  or so. Today these facilities are less used because there are many
  (mostly better) alternatives. For example, \R' s own
  \code{aggregate()} function is an alternative to \doby 's
  \code{summaryBy()} function. Also the
  \href{https://cran.r-project.org/web/packages/dplyr/index.html}{dplyr} package
  provides nice altertatives to the facilities for groupwise computations in \doby.

\item However, \doby\ contains many other utilities.
\item Among these are
  facilities for computing LS-means (least squares means, also known
  as marginal means or population means). 
\item Development versions of \doby\ may be available
  \href{./devel}{here}
\end{itemize}

% \paragraph{Development versions}
% \label{sec:development-versions}



\paragraph{Vignettes}
\label{sec:vignettes}

\begin{itemize}
\item 
  \href{http://cran.r-project.org/web/packages/doBy/vignettes/LSmeans.pdf}{doBy: LSmeans and other linear estimates}
\item 
  \href{http://cran.r-project.org/web/packages/doBy/vignettes/doBy.pdf}{doBy: Groupwise computations and other utilities}
\end{itemize}


% \paragraph{Miscellaneous}
% \label{sec:miscellaneous}
% Markup of R-script: 
% Around 2010 the \htmlrep\ function was introduced as a tool for
% turning an \R\ script file into an HTML document. The \htmlrep\
% function has been removed from \doby\ because it is now obsolete. A
% better alternative is to use the \code{markdown} and \code{knitr} packages. An
% example is provided in the script file
% \href{./markdown/Puromycin-markdown.txt}{Puromycin-markdown.txt}. The
% HTML document created from the script is
% \href{./markdown/Puromycin.html}{here}. The program
% \code{wkhtmltopdf} does a good job in turning HTML into pdf; the
% result can be seen   \href{./markdown/Puromycin.pdf}{here}.


\paragraph{FAQ (Frequently Asked Questions)}
\label{sec:faq-frequently-asked}

\begin{description}

\item[Q:] Is there a paper describing the \doby\ package?

\item[A:] Not for the moment.

\end{description}

S�ren H�jsgaard

sorenh [at] math [dot] aau [dot] dk

\special{html: 
<!-- Site Meter -->
<script type="text/javascript" src="http://s24.sitemeter.com/js/counter.js?site=s24sorenh">
</script>
<noscript>
<a href="http://s24.sitemeter.com/stats.asp?site=s24sorenh" target="_top">
<img src="http://s24.sitemeter.com/meter.asp?site=s24sorenh" alt="Site Meter" border="0"/></a>
</noscript>
<!-- Copyright (c)2009 Site Meter -->
}

\end{document}
