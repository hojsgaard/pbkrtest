%% This file has been modified by igattr
%% Do not edit manually
\documentclass[10pt]{article}
\usepackage[T1]{fontenc}
\usepackage{a4wide,SHmathnot}
\title{Estimability}
\author{S�ren H�jsgaard}
\usepackage[small,compact]{titlesec}
\usepackage{Sweave}
\begin{document}
\renewenvironment{Schunk}{\linespread{.85}\small}{}

\maketitle
\parindent0pt\parskip5pt

Consider an $n$ dimensional random vector $y$ for which $\EE(y)=\mu=X\beta$
ans $\cov(y)=V$. Here $X$ is an $n\times p$ with $n>p$ matrix which does not
necessarily have full rank.
A least squares estimate for $\beta$ is
\begin{displaymath}
  \hat \beta = (X\transp V\inv X)^- X\transp V\inv y
\end{displaymath}
where $A^-$ is a generalized inverse of $A$.
We are interested in contrasts of the form $c=\lambda\transp\beta$. An
estimate of such a contrast is $\hat c = \lambda\transp\hat\beta$.
Generalized inverses are not unique and therefore $\hat \beta$ is not
uniquely estimated, and hence the contrasts are not uniquely
estimated. On the other hand the fitted values $\hat y = X \hat \beta$
are uniquely indentified.

However, for some $\lambda$s we always get the same estimated
contrast. Such contrasts are said to be estimable. These contrasts can
be described as follows:
We
can only learn about $\beta$ through $X\beta$ so the only thing we can
say something about is linear combinations $\rho\transp X\beta$. Hence
we can only say something about $\lambda\transp\beta$ if there exists
$\rho$ such that $\lambda\transp\beta=\rho\transp X \beta$, i.e., if
$\lambda=X\transp\rho$, that is, if $\lambda$ is in the column space
$C(X\transp)$ of $X\transp$. That is, if $\lambda$ is perpendicular to
all vectors in the null space $N(X)$ of $X$. To check
this, we find a basis $B$ for $N(X) \subset R^p$ which can be done using SVD:

Example:
\begin{Schunk}
\begin{Sinput}
> ff <- factor(rep(1:3, each=2))
> X  <- cbind(rep(1,6),model.matrix(~0+ff)); X
\end{Sinput}
\begin{Soutput}
    ff1 ff2 ff3
1 1   1   0   0
2 1   1   0   0
3 1   0   1   0
4 1   0   1   0
5 1   0   0   1
6 1   0   0   1
\end{Soutput}
\begin{Sinput}
> y  <- 1:6
\end{Sinput}
\end{Schunk}

\begin{Schunk}
\begin{Sinput}
> S <- svd(X)
> lapply(S, zapsmall)
\end{Sinput}
\begin{Soutput}
$d
[1] 2.828427 1.414214 1.414214 0.000000

$u
           [,1]       [,2]       [,3]       [,4]
[1,] -0.4082483  0.5534366  0.1644422 -0.7071068
[2,] -0.4082483  0.5534366  0.1644422  0.7071068
[3,] -0.4082483 -0.1343072 -0.5615113  0.0000000
[4,] -0.4082483 -0.1343072 -0.5615113  0.0000000
[5,] -0.4082483 -0.4191294  0.3970691  0.0000000
[6,] -0.4082483 -0.4191294  0.3970691  0.0000000

$v
           [,1]       [,2]       [,3] [,4]
[1,] -0.8660254  0.0000000  0.0000000  0.5
[2,] -0.2886751  0.7826776  0.2325563 -0.5
[3,] -0.2886751 -0.1899391 -0.7940968 -0.5
[4,] -0.2886751 -0.5927385  0.5615405 -0.5
\end{Soutput}
\end{Schunk}

The basis of $N(X)$ is
\begin{Schunk}
\begin{Sinput}
> B <- S$v[, S$d/max(S$d) < 1e-10, drop=F]
> (B <- as.numeric(B/max(B)))
\end{Sinput}
\begin{Soutput}
[1]  1 -1 -1 -1
\end{Soutput}
\end{Schunk}

Hence valid vectors $\lambda$ must satisfy that $
  \lambda_1 - \lambda_2 - \lambda_3 - \lambda_4  = 0.$

\begin{Schunk}
\begin{Sinput}
> lam <- c(1, 1, 0, 0)
> sum( lam*B ) # Orthogonal
\end{Sinput}
\begin{Soutput}
[1] 2.220446e-16
\end{Soutput}
\end{Schunk}

\begin{Schunk}
\begin{Sinput}
> (b.hat <- as.numeric(MASS::ginv(t(X)%*%X) %*% t(X) %*% y))
\end{Sinput}
\begin{Soutput}
[1]  2.625 -1.125  0.875  2.875
\end{Soutput}
\begin{Sinput}
> sum( lam * b.hat )
\end{Sinput}
\begin{Soutput}
[1] 1.5
\end{Soutput}
\end{Schunk}

Take another basis for the mean space:
\begin{Schunk}
\begin{Sinput}
> X2 <- X
> X2[,3]<-0
> (b.hat2 <- as.numeric(MASS::ginv(t(X2)%*%X2) %*% t(X2) %*% y))
\end{Sinput}
\begin{Soutput}
[1]  3.5 -2.0  0.0  2.0
\end{Soutput}
\begin{Sinput}
> sum( lam * b.hat2 )
\end{Sinput}
\begin{Soutput}
[1] 1.5
\end{Soutput}
\end{Schunk}

On the other other hand, the following does not produce a unique estimate:
\begin{Schunk}
\begin{Sinput}
> lam2 <- c(1, 0, 0, 0)
> sum( lam2 * b.hat )
\end{Sinput}
\begin{Soutput}
[1] 2.625
\end{Soutput}
\begin{Sinput}
> sum( lam2 * b.hat2 )
\end{Sinput}
\begin{Soutput}
[1] 3.5
\end{Soutput}
\end{Schunk}



\end{document}
