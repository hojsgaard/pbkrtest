% Options for packages loaded elsewhere
\PassOptionsToPackage{unicode}{hyperref}
\PassOptionsToPackage{hyphens}{url}
%
\documentclass[
  ignorenonframetext,
]{beamer}
\usepackage{pgfpages}
\setbeamertemplate{caption}[numbered]
\setbeamertemplate{caption label separator}{: }
\setbeamercolor{caption name}{fg=normal text.fg}
\beamertemplatenavigationsymbolsempty
% Prevent slide breaks in the middle of a paragraph
\widowpenalties 1 10000
\raggedbottom
\setbeamertemplate{part page}{
  \centering
  \begin{beamercolorbox}[sep=16pt,center]{part title}
    \usebeamerfont{part title}\insertpart\par
  \end{beamercolorbox}
}
\setbeamertemplate{section page}{
  \centering
  \begin{beamercolorbox}[sep=12pt,center]{part title}
    \usebeamerfont{section title}\insertsection\par
  \end{beamercolorbox}
}
\setbeamertemplate{subsection page}{
  \centering
  \begin{beamercolorbox}[sep=8pt,center]{part title}
    \usebeamerfont{subsection title}\insertsubsection\par
  \end{beamercolorbox}
}
\AtBeginPart{
  \frame{\partpage}
}
\AtBeginSection{
  \ifbibliography
  \else
    \frame{\sectionpage}
  \fi
}
\AtBeginSubsection{
  \frame{\subsectionpage}
}

\usepackage{amsmath,amssymb}
\usepackage{lmodern}
\usepackage{iftex}
\ifPDFTeX
  \usepackage[T1]{fontenc}
  \usepackage[utf8]{inputenc}
  \usepackage{textcomp} % provide euro and other symbols
\else % if luatex or xetex
  \usepackage{unicode-math}
  \defaultfontfeatures{Scale=MatchLowercase}
  \defaultfontfeatures[\rmfamily]{Ligatures=TeX,Scale=1}
\fi
% Use upquote if available, for straight quotes in verbatim environments
\IfFileExists{upquote.sty}{\usepackage{upquote}}{}
\IfFileExists{microtype.sty}{% use microtype if available
  \usepackage[]{microtype}
  \UseMicrotypeSet[protrusion]{basicmath} % disable protrusion for tt fonts
}{}
\makeatletter
\@ifundefined{KOMAClassName}{% if non-KOMA class
  \IfFileExists{parskip.sty}{%
    \usepackage{parskip}
  }{% else
    \setlength{\parindent}{0pt}
    \setlength{\parskip}{6pt plus 2pt minus 1pt}}
}{% if KOMA class
  \KOMAoptions{parskip=half}}
\makeatother
\usepackage{xcolor}
\newif\ifbibliography
\setlength{\emergencystretch}{3em} % prevent overfull lines
\setcounter{secnumdepth}{3}

\usepackage{color}
\usepackage{fancyvrb}
\newcommand{\VerbBar}{|}
\newcommand{\VERB}{\Verb[commandchars=\\\{\}]}
\DefineVerbatimEnvironment{Highlighting}{Verbatim}{commandchars=\\\{\}}
% Add ',fontsize=\small' for more characters per line
\usepackage{framed}
\definecolor{shadecolor}{RGB}{241,243,245}
\newenvironment{Shaded}{\begin{snugshade}}{\end{snugshade}}
\newcommand{\AlertTok}[1]{\textcolor[rgb]{0.68,0.00,0.00}{#1}}
\newcommand{\AnnotationTok}[1]{\textcolor[rgb]{0.37,0.37,0.37}{#1}}
\newcommand{\AttributeTok}[1]{\textcolor[rgb]{0.40,0.45,0.13}{#1}}
\newcommand{\BaseNTok}[1]{\textcolor[rgb]{0.68,0.00,0.00}{#1}}
\newcommand{\BuiltInTok}[1]{\textcolor[rgb]{0.00,0.23,0.31}{#1}}
\newcommand{\CharTok}[1]{\textcolor[rgb]{0.13,0.47,0.30}{#1}}
\newcommand{\CommentTok}[1]{\textcolor[rgb]{0.37,0.37,0.37}{#1}}
\newcommand{\CommentVarTok}[1]{\textcolor[rgb]{0.37,0.37,0.37}{\textit{#1}}}
\newcommand{\ConstantTok}[1]{\textcolor[rgb]{0.56,0.35,0.01}{#1}}
\newcommand{\ControlFlowTok}[1]{\textcolor[rgb]{0.00,0.23,0.31}{#1}}
\newcommand{\DataTypeTok}[1]{\textcolor[rgb]{0.68,0.00,0.00}{#1}}
\newcommand{\DecValTok}[1]{\textcolor[rgb]{0.68,0.00,0.00}{#1}}
\newcommand{\DocumentationTok}[1]{\textcolor[rgb]{0.37,0.37,0.37}{\textit{#1}}}
\newcommand{\ErrorTok}[1]{\textcolor[rgb]{0.68,0.00,0.00}{#1}}
\newcommand{\ExtensionTok}[1]{\textcolor[rgb]{0.00,0.23,0.31}{#1}}
\newcommand{\FloatTok}[1]{\textcolor[rgb]{0.68,0.00,0.00}{#1}}
\newcommand{\FunctionTok}[1]{\textcolor[rgb]{0.28,0.35,0.67}{#1}}
\newcommand{\ImportTok}[1]{\textcolor[rgb]{0.00,0.46,0.62}{#1}}
\newcommand{\InformationTok}[1]{\textcolor[rgb]{0.37,0.37,0.37}{#1}}
\newcommand{\KeywordTok}[1]{\textcolor[rgb]{0.00,0.23,0.31}{#1}}
\newcommand{\NormalTok}[1]{\textcolor[rgb]{0.00,0.23,0.31}{#1}}
\newcommand{\OperatorTok}[1]{\textcolor[rgb]{0.37,0.37,0.37}{#1}}
\newcommand{\OtherTok}[1]{\textcolor[rgb]{0.00,0.23,0.31}{#1}}
\newcommand{\PreprocessorTok}[1]{\textcolor[rgb]{0.68,0.00,0.00}{#1}}
\newcommand{\RegionMarkerTok}[1]{\textcolor[rgb]{0.00,0.23,0.31}{#1}}
\newcommand{\SpecialCharTok}[1]{\textcolor[rgb]{0.37,0.37,0.37}{#1}}
\newcommand{\SpecialStringTok}[1]{\textcolor[rgb]{0.13,0.47,0.30}{#1}}
\newcommand{\StringTok}[1]{\textcolor[rgb]{0.13,0.47,0.30}{#1}}
\newcommand{\VariableTok}[1]{\textcolor[rgb]{0.07,0.07,0.07}{#1}}
\newcommand{\VerbatimStringTok}[1]{\textcolor[rgb]{0.13,0.47,0.30}{#1}}
\newcommand{\WarningTok}[1]{\textcolor[rgb]{0.37,0.37,0.37}{\textit{#1}}}

\providecommand{\tightlist}{%
  \setlength{\itemsep}{0pt}\setlength{\parskip}{0pt}}\usepackage{longtable,booktabs,array}
\usepackage{calc} % for calculating minipage widths
\usepackage{caption}
% Make caption package work with longtable
\makeatletter
\def\fnum@table{\tablename~\thetable}
\makeatother
\usepackage{graphicx}
\makeatletter
\def\maxwidth{\ifdim\Gin@nat@width>\linewidth\linewidth\else\Gin@nat@width\fi}
\def\maxheight{\ifdim\Gin@nat@height>\textheight\textheight\else\Gin@nat@height\fi}
\makeatother
% Scale images if necessary, so that they will not overflow the page
% margins by default, and it is still possible to overwrite the defaults
% using explicit options in \includegraphics[width, height, ...]{}
\setkeys{Gin}{width=\maxwidth,height=\maxheight,keepaspectratio}
% Set default figure placement to htbp
\makeatletter
\def\fps@figure{htbp}
\makeatother

\makeatletter
\makeatother
\makeatletter
\makeatother
\makeatletter
\@ifpackageloaded{caption}{}{\usepackage{caption}}
\AtBeginDocument{%
\ifdefined\contentsname
  \renewcommand*\contentsname{Table of contents}
\else
  \newcommand\contentsname{Table of contents}
\fi
\ifdefined\listfigurename
  \renewcommand*\listfigurename{List of Figures}
\else
  \newcommand\listfigurename{List of Figures}
\fi
\ifdefined\listtablename
  \renewcommand*\listtablename{List of Tables}
\else
  \newcommand\listtablename{List of Tables}
\fi
\ifdefined\figurename
  \renewcommand*\figurename{Figure}
\else
  \newcommand\figurename{Figure}
\fi
\ifdefined\tablename
  \renewcommand*\tablename{Table}
\else
  \newcommand\tablename{Table}
\fi
}
\@ifpackageloaded{float}{}{\usepackage{float}}
\floatstyle{ruled}
\@ifundefined{c@chapter}{\newfloat{codelisting}{h}{lop}}{\newfloat{codelisting}{h}{lop}[chapter]}
\floatname{codelisting}{Listing}
\newcommand*\listoflistings{\listof{codelisting}{List of Listings}}
\makeatother
\makeatletter
\@ifpackageloaded{caption}{}{\usepackage{caption}}
\@ifpackageloaded{subcaption}{}{\usepackage{subcaption}}
\makeatother
\makeatletter
\@ifpackageloaded{tcolorbox}{}{\usepackage[many]{tcolorbox}}
\makeatother
\makeatletter
\@ifundefined{shadecolor}{\definecolor{shadecolor}{rgb}{.97, .97, .97}}
\makeatother
\makeatletter
\makeatother
\ifLuaTeX
  \usepackage{selnolig}  % disable illegal ligatures
\fi
\IfFileExists{bookmark.sty}{\usepackage{bookmark}}{\usepackage{hyperref}}
\IfFileExists{xurl.sty}{\usepackage{xurl}}{} % add URL line breaks if available
\urlstyle{same} % disable monospaced font for URLs
\hypersetup{
  pdftitle={Computer algebra systems in R},
  pdfauthor={Mikkel Meyer Andersen and  Søren Højsgaard},
  hidelinks,
  pdfcreator={LaTeX via pandoc}}

\title{Computer algebra systems in \texttt{R}}
\subtitle{COMPSTAT 2023 London, UK}
\author{Mikkel Meyer Andersen and Søren Højsgaard}
\date{8/24/23}

\begin{document}
\frame{\titlepage}
\ifdefined\Shaded\renewenvironment{Shaded}{\begin{tcolorbox}[enhanced, sharp corners, frame hidden, interior hidden, boxrule=0pt, breakable, borderline west={3pt}{0pt}{shadecolor}]}{\end{tcolorbox}}\fi

\renewcommand*\contentsname{Table of contents}
\begin{frame}[allowframebreaks]
  \frametitle{Table of contents}
  \tableofcontents[hideallsubsections]
\end{frame}
\hypertarget{take-home-message}{%
\section{Take-home message}\label{take-home-message}}

\begin{frame}{Take-home message}
\begin{itemize}
\item
  The caracas package for R provides computer algebra / symbolic math

  \begin{itemize}
  \tightlist
  \item
    At your fingertips\ldots{}
  \item
    within R\ldots{}
  \item
    using R syntax
  \item
    Calculus: derivatives, integrals, sums etc.
  \item
    Linear algebra
  \item
    Solving equations
  \end{itemize}
\item
  The caracas package can easily be extended
\item
  Easy transition from symbolic expression to numerical expressions
\item
  Easy generation of math expressions for documents (used in this
  presentation).
\item
  See \url{https://r-cas.github.io/caracas/} for vignettes and other
  info.
\end{itemize}
\end{frame}

\hypertarget{caracas}{%
\section{\texorpdfstring{\texttt{caracas}}{caracas}}\label{caracas}}

\begin{frame}[fragile]{\texttt{caracas}}
\begin{itemize}
\item
  Initiated in 2019 by \textbf{Søren Højsgaard} and \textbf{Mikkel Meyer
  Andersen}
\item
  Supported by a grant from the R Consortium
\item
  Based on \href{https://www.sympy.org/}{SymPy} (large computer algebra
  library for Python), using \texttt{reticulate} package in R.
\item
  `\emph{cara}': face in Spanish (Castellano) / `\emph{cas}': computer
  algebra system
\item
  Links

  \begin{itemize}
  \tightlist
  \item
    Stable version: \url{https://CRAN.R-project.org/package=caracas}
  \item
    Development version: \url{https://github.com/r-cas/caracas/}
  \item
    Online documentation: \url{http://r-cas.github.io/caracas/}
  \end{itemize}
\end{itemize}
\end{frame}

\hypertarget{getting-started}{%
\section{Getting started}\label{getting-started}}

\begin{frame}[fragile]{Installation}
\protect\hypertarget{installation}{}
\begin{Shaded}
\begin{Highlighting}[]
\CommentTok{\#devtools::install\_github("r{-}cas/caracas")}
\FunctionTok{install.packages}\NormalTok{(}\StringTok{"caracas"}\NormalTok{)}
\end{Highlighting}
\end{Shaded}

\begin{Shaded}
\begin{Highlighting}[]
\FunctionTok{library}\NormalTok{(caracas)}
\FunctionTok{packageVersion}\NormalTok{(}\StringTok{"caracas"}\NormalTok{)}
\end{Highlighting}
\end{Shaded}

\begin{verbatim}
[1] '2.0.1.9001'
\end{verbatim}
\end{frame}

\begin{frame}[fragile]{Symbols}
\protect\hypertarget{symbols}{}
\begin{Shaded}
\begin{Highlighting}[]
\FunctionTok{def\_sym}\NormalTok{(x, y)}
\NormalTok{p }\OtherTok{\textless{}{-}}\NormalTok{ x}\SpecialCharTok{\^{}}\DecValTok{2} \SpecialCharTok{+} \DecValTok{3}\SpecialCharTok{*}\NormalTok{x }\SpecialCharTok{+} \DecValTok{4}\SpecialCharTok{*}\NormalTok{y }\SpecialCharTok{+}\NormalTok{ y}\SpecialCharTok{\^{}}\DecValTok{4}
\NormalTok{p}
\end{Highlighting}
\end{Shaded}

\begin{verbatim}
[c]:  2          4      
     x  + 3*x + y  + 4*y
\end{verbatim}

\begin{Shaded}
\begin{Highlighting}[]
\FunctionTok{str}\NormalTok{(x)}
\end{Highlighting}
\end{Shaded}

\begin{verbatim}
List of 1
 $ pyobj:x
 - attr(*, "class")= chr "caracas_symbol"
\end{verbatim}

\begin{Shaded}
\begin{Highlighting}[]
\FunctionTok{str}\NormalTok{(p)}
\end{Highlighting}
\end{Shaded}

\begin{verbatim}
List of 1
 $ pyobj:x**2 + 3*x + y**4 + 4*y
 - attr(*, "class")= chr "caracas_symbol"
\end{verbatim}
\end{frame}

\begin{frame}[fragile]{Documents with mathematical contents}
\protect\hypertarget{documents-with-mathematical-contents}{}
Write the following in LaTeX:

\begin{Shaded}
\begin{Highlighting}[]
\SpecialCharTok{$}\ErrorTok{$}
\NormalTok{p }\OtherTok{=} \StringTok{\textasciigrave{}}\AttributeTok{r tex(p)}\StringTok{\textasciigrave{}}
\SpecialCharTok{$}\ErrorTok{$}
\end{Highlighting}
\end{Shaded}

Gives:

\[
p = x^{2} + 3 x + y^{4} + 4 y
\]

Used throughout this presentation :)
\end{frame}

\begin{frame}[fragile]{From symbols to R expressions and numerical
evaluations}
\protect\hypertarget{from-symbols-to-r-expressions-and-numerical-evaluations}{}
\begin{Shaded}
\begin{Highlighting}[]
\NormalTok{p\_ }\OtherTok{\textless{}{-}} \FunctionTok{as\_expr}\NormalTok{(p); p\_}
\end{Highlighting}
\end{Shaded}

\begin{verbatim}
expression(x^2 + 3 * x + y^4 + 4 * y)
\end{verbatim}

\begin{Shaded}
\begin{Highlighting}[]
\FunctionTok{eval}\NormalTok{(p\_, }\FunctionTok{list}\NormalTok{(}\AttributeTok{x =} \DecValTok{1}\NormalTok{, }\AttributeTok{y =} \DecValTok{1}\NormalTok{))}
\end{Highlighting}
\end{Shaded}

\begin{verbatim}
[1] 9
\end{verbatim}

\begin{Shaded}
\begin{Highlighting}[]
\NormalTok{p\_fn }\OtherTok{\textless{}{-}} \FunctionTok{as\_func}\NormalTok{(p); p\_fn}
\end{Highlighting}
\end{Shaded}

\begin{verbatim}
function (x, y) 
{
    x^2 + 3 * x + y^4 + 4 * y
}
<environment: 0x55d59e8d59b0>
\end{verbatim}

\begin{Shaded}
\begin{Highlighting}[]
\FunctionTok{p\_fn}\NormalTok{(}\AttributeTok{x =} \DecValTok{1}\NormalTok{, }\AttributeTok{y =} \DecValTok{1}\NormalTok{)}
\end{Highlighting}
\end{Shaded}

\begin{verbatim}
[1] 9
\end{verbatim}
\end{frame}

\hypertarget{mathematical-examples}{%
\section{Mathematical examples}\label{mathematical-examples}}

\begin{frame}[fragile]{Linear algebra}
\protect\hypertarget{linear-algebra}{}
\begin{verbatim}
as_sym() # Converts R object to caracas symbol
\end{verbatim}

\begin{Shaded}
\begin{Highlighting}[]
\NormalTok{A }\OtherTok{\textless{}{-}} \FunctionTok{matrix\_}\NormalTok{(}\FunctionTok{c}\NormalTok{(}\DecValTok{2}\NormalTok{, }\DecValTok{1}\NormalTok{, }\DecValTok{4}\NormalTok{, }\StringTok{"x"}\NormalTok{), }\DecValTok{2}\NormalTok{, }\DecValTok{2}\NormalTok{)}
\DocumentationTok{\#\# A \textless{}{-} as\_sym(matrix(c(2, 1, 4, "x"), 2, 2)) \#\# Same}
\NormalTok{A}
\end{Highlighting}
\end{Shaded}

\[\left[\begin{matrix}2 & 4\\1 & x\end{matrix}\right]\]

\begin{Shaded}
\begin{Highlighting}[]
\FunctionTok{t}\NormalTok{(A)}
\end{Highlighting}
\end{Shaded}

\[\left[\begin{matrix}2 & 1\\4 & x\end{matrix}\right]\]

\begin{Shaded}
\begin{Highlighting}[]
\FunctionTok{det}\NormalTok{(A)}
\end{Highlighting}
\end{Shaded}

\[2 x - 4\]
\end{frame}

\begin{frame}[fragile]
\begin{Shaded}
\begin{Highlighting}[]
\NormalTok{A[}\DecValTok{2}\NormalTok{,]}
\end{Highlighting}
\end{Shaded}

\[\left[\begin{matrix}1\\x\end{matrix}\right]\]

\begin{Shaded}
\begin{Highlighting}[]
\NormalTok{A }\SpecialCharTok{\%*\%}\NormalTok{ A[}\DecValTok{2}\NormalTok{,]}
\end{Highlighting}
\end{Shaded}

\[\left[\begin{matrix}4 x + 2\\x^{2} + 1\end{matrix}\right]\]

\begin{Shaded}
\begin{Highlighting}[]
\NormalTok{Ai  }\OtherTok{\textless{}{-}} \FunctionTok{inv}\NormalTok{(A) }\SpecialCharTok{|\textgreater{}} \FunctionTok{simplify}\NormalTok{() }
\NormalTok{Ai}
\end{Highlighting}
\end{Shaded}

\[\left[\begin{matrix}\frac{x}{2 \left(x - 2\right)} & - \frac{2}{x - 2}\\- \frac{1}{2 x - 4} & \frac{1}{x - 2}\end{matrix}\right]\]
\end{frame}

\begin{frame}[fragile]{Solving equations}
\protect\hypertarget{solving-equations}{}
\begin{verbatim}
# Solve Ax = b; also inv(A) for inverse of A
solve_lin(A, b)
# Solve lhs = rhs for vars; rhs omitted finds roots
solve_sys(lhs, rhs, vars) 
\end{verbatim}

\begin{Shaded}
\begin{Highlighting}[]
\FunctionTok{def\_sym}\NormalTok{(x, y)}
\NormalTok{lhs }\OtherTok{\textless{}{-}} \FunctionTok{cbind}\NormalTok{(}\DecValTok{3} \SpecialCharTok{*}\NormalTok{ x }\SpecialCharTok{*}\NormalTok{ y }\SpecialCharTok{{-}}\NormalTok{ y, x)}
\NormalTok{rhs }\OtherTok{\textless{}{-}} \FunctionTok{cbind}\NormalTok{(}\SpecialCharTok{{-}}\DecValTok{5} \SpecialCharTok{*}\NormalTok{ x, y }\SpecialCharTok{+} \DecValTok{4}\NormalTok{)}
\end{Highlighting}
\end{Shaded}

\[
\left[\begin{matrix}3 x y - y\\x\end{matrix}\right] = \left[\begin{matrix}- 5 x\\y + 4\end{matrix}\right] 
\]
\end{frame}

\begin{frame}[fragile]
\begin{Shaded}
\begin{Highlighting}[]
\NormalTok{sol }\OtherTok{\textless{}{-}} \FunctionTok{solve\_sys}\NormalTok{(lhs, rhs, }\FunctionTok{list}\NormalTok{(x, y))}
\NormalTok{sol}
\end{Highlighting}
\end{Shaded}

\begin{verbatim}
Solution 1:
  x =  2/3 
  y =  -10/3 
Solution 2:
  x =  2 
  y =  -2 
\end{verbatim}
\end{frame}

\begin{frame}[fragile]{Derivatives - gradient and Hessian}
\protect\hypertarget{derivatives---gradient-and-hessian}{}
\begin{Shaded}
\begin{Highlighting}[]
\NormalTok{gp }\OtherTok{\textless{}{-}} \FunctionTok{der}\NormalTok{(p, }\FunctionTok{c}\NormalTok{(x, y))}
\NormalTok{gp}
\end{Highlighting}
\end{Shaded}

\[\left[\begin{matrix}2 x + 3 & 4 y^{3} + 4\end{matrix}\right]\]

\begin{Shaded}
\begin{Highlighting}[]
\NormalTok{H }\OtherTok{\textless{}{-}} \FunctionTok{der2}\NormalTok{(p, }\FunctionTok{c}\NormalTok{(x, y)) }\CommentTok{\# Hessian}
\end{Highlighting}
\end{Shaded}

\[
H = \left[\begin{matrix}2 & 0\\0 & 12 y^{2}\end{matrix}\right]
\]
\end{frame}

\begin{frame}[fragile]{Sums}
\protect\hypertarget{sums}{}
\begin{Shaded}
\begin{Highlighting}[]
\FunctionTok{sum\_}\NormalTok{(expr, var, [from, to], }\AttributeTok{doit =} \ConstantTok{TRUE}\NormalTok{)}
\end{Highlighting}
\end{Shaded}

Find \(\sum_{k=0}^n k^2\).

\begin{Shaded}
\begin{Highlighting}[]
\FunctionTok{def\_sym}\NormalTok{(k)}
\NormalTok{s1 }\OtherTok{\textless{}{-}} \FunctionTok{sum\_}\NormalTok{(k}\SpecialCharTok{\^{}}\DecValTok{2}\NormalTok{, k, }\DecValTok{0}\NormalTok{, }\StringTok{"n"}\NormalTok{, }\AttributeTok{doit =} \ConstantTok{FALSE}\NormalTok{)}
\NormalTok{s2 }\OtherTok{\textless{}{-}} \FunctionTok{doit}\NormalTok{(s1)}
\NormalTok{s3 }\OtherTok{\textless{}{-}}\NormalTok{ s2 }\SpecialCharTok{|\textgreater{}} \FunctionTok{simplify}\NormalTok{()}
\end{Highlighting}
\end{Shaded}

\[
s1 = \sum_{k=0}^{n} k^{2}; \quad s2 = \frac{n^{3}}{3} + \frac{n^{2}}{2} + \frac{n}{6}; \quad s3 = \frac{n \left(2 n^{2} + 3 n + 1\right)}{6}
\]
\end{frame}

\begin{frame}[fragile]{Integration}
\protect\hypertarget{integration}{}
\begin{verbatim}
int(expr, var, [from, to], doit = TRUE)
\end{verbatim}

Upper half of unit circle: \(y=\sqrt{1-x^2}\) for \(-1 \le x \le 1\).

\begin{Shaded}
\begin{Highlighting}[]
\FunctionTok{def\_sym}\NormalTok{(x, y)}
\NormalTok{y }\OtherTok{\textless{}{-}} \FunctionTok{sqrt}\NormalTok{(}\DecValTok{1} \SpecialCharTok{{-}}\NormalTok{ x}\SpecialCharTok{\^{}}\DecValTok{2}\NormalTok{)}
\NormalTok{s1 }\OtherTok{\textless{}{-}} \FunctionTok{int}\NormalTok{(y, x)}
\NormalTok{s2 }\OtherTok{\textless{}{-}} \FunctionTok{int}\NormalTok{(y, x, }\SpecialCharTok{{-}}\DecValTok{1}\NormalTok{, }\DecValTok{1}\NormalTok{)}
\end{Highlighting}
\end{Shaded}

\[
y=\sqrt{1 - x^{2}}; \: s1 = \frac{x \sqrt{1 - x^{2}}}{2} + \frac{\operatorname{asin}{\left(x \right)}}{2}; \: s2=\frac{\pi}{2}
\]
\end{frame}

\hypertarget{variance-of-the-average-of-correlated-data}{%
\section{Variance of the average of correlated
data}\label{variance-of-the-average-of-correlated-data}}

\begin{frame}{Variance of the average of correlated data}
Consider random variables \(x_1,\dots, x_n\) where
\(\mathbf{Var}(x_i)=v\) and \(\mathbf{Cov}(x_i, x_j)=v r\) for
\(i\not = j\), where \(0 \le |r| \le1\). For \(n=3\), the covariance
matrix of \((x_1,\dots, x_n)\) is therefore

\begin{equation}
  \label{eq:1}
  V = v R = v \left[\begin{matrix}1 & r & r\\r & 1 & r\\r & r & 1\end{matrix}\right].
\end{equation}

Let \(\bar x = \sum_i x_i / n\) denote the average.

\begin{itemize}
\item
  What is \(\mathbf{Var}(\bar x)\), when \(n\) goes to infinity for
  fixed \(r\)?
\item
  What is \(\mathbf{Var}(\bar x)\), when \(r\) goes \(0\) and \(1\) for
  fixed \(n\)?
\item
  How many independent observations do \(n\) correlated observations
  correspond to (in terms of the same variance of the averages)?
\end{itemize}
\end{frame}

\begin{frame}[fragile]
We need the variance of a sum \(x. = \sum_i x_i\) which is \begin{align}
\mathbf{Var}(x.) &= \sum_i \mathbf{Var}(x_i) + 2 \sum_{ij:i<j}
\mathbf{Cov}(x_i, x_j) \\
&= v(n + 2 \sum_{i=1}^{n-1}\sum_{j=i+1}^{n} r)
\end{align}

(i.e., the sum of the elements of the covariance matrix). We can do this
in \texttt{caracas} as follows:

\begin{Shaded}
\begin{Highlighting}[]
\FunctionTok{def\_sym}\NormalTok{(v, r, n, j, i)}
\NormalTok{s1 }\OtherTok{\textless{}{-}} \FunctionTok{sum\_}\NormalTok{(r, j, i}\SpecialCharTok{+}\DecValTok{1}\NormalTok{, n)}
\NormalTok{s2 }\OtherTok{\textless{}{-}} \FunctionTok{sum\_}\NormalTok{(s1, i, }\DecValTok{1}\NormalTok{, n}\DecValTok{{-}1}\NormalTok{)}
\NormalTok{var\_sum }\OtherTok{\textless{}{-}}\NormalTok{ v}\SpecialCharTok{*}\NormalTok{(n }\SpecialCharTok{+} \DecValTok{2} \SpecialCharTok{*}\NormalTok{ s2) }\SpecialCharTok{|\textgreater{}} \FunctionTok{simplify}\NormalTok{()}
\NormalTok{var\_avg }\OtherTok{\textless{}{-}}\NormalTok{ var\_sum }\SpecialCharTok{/}\NormalTok{ n}\SpecialCharTok{\^{}}\DecValTok{2}
\end{Highlighting}
\end{Shaded}

\begin{align}
\texttt{s1} &= r \left(- i + n\right); \quad
\texttt{s2} = n r \left(n - 1\right) - r \left(\frac{n^{2}}{2} - \frac{n}{2}\right) 
\end{align}
\end{frame}

\begin{frame}[fragile]
\begin{align*}
\mathbf{Var}(x.) &= n v \left(r \left(n - 1\right) + 1\right),
\quad
\mathbf{Var}(\bar x) &= \frac{v \left(r \left(n - 1\right) + 1\right)}{n}.
\end{align*}

From hereof, we can study the limiting behavior of the variance
\(\mathbf{Var}(\bar x)\) in different situations:

\begin{Shaded}
\begin{Highlighting}[]
\NormalTok{l\_1 }\OtherTok{\textless{}{-}} \FunctionTok{lim}\NormalTok{(var\_avg, n, }\ConstantTok{Inf}\NormalTok{)        }\DocumentationTok{\#\# n {-}\textgreater{} infinity}
\NormalTok{l\_2 }\OtherTok{\textless{}{-}} \FunctionTok{lim}\NormalTok{(var\_avg, r, }\DecValTok{0}\NormalTok{, }\AttributeTok{dir=}\StringTok{\textquotesingle{}+\textquotesingle{}}\NormalTok{) }\DocumentationTok{\#\# r {-}\textgreater{} 0}
\NormalTok{l\_3 }\OtherTok{\textless{}{-}} \FunctionTok{lim}\NormalTok{(var\_avg, r, }\DecValTok{1}\NormalTok{, }\AttributeTok{dir=}\StringTok{\textquotesingle{}{-}\textquotesingle{}}\NormalTok{) }\DocumentationTok{\#\# r {-}\textgreater{} 1}
\end{Highlighting}
\end{Shaded}

\[
l_1 = r v, \quad
l_2 = \frac{v}{n}, \quad
l_3 = v, \quad
\]
\end{frame}

\begin{frame}[fragile]
For a given correlation \(r\), investigate how many independent
variables \(k_n\) the \(n\) correlated variables correspond to (in the
sense of the same variance of the average).

Moreover, study how \(k_n\) behaves as function of \(n\) when
\(n \rightarrow \infty\). That is we must

\begin{enumerate}
\item
  solve \(v (1 + (n-1)r)/n = v/k_n\) for \(k_n\) and
\item
  find \(\lim_{n\rightarrow\infty} k\):
\end{enumerate}

\begin{Shaded}
\begin{Highlighting}[]
\FunctionTok{def\_sym}\NormalTok{(k\_n)}
\NormalTok{k\_n }\OtherTok{\textless{}{-}} \FunctionTok{solve\_sys}\NormalTok{(var\_avg }\SpecialCharTok{{-}}\NormalTok{ v }\SpecialCharTok{/}\NormalTok{ k\_n, k\_n)[[}\DecValTok{1}\NormalTok{]]}\SpecialCharTok{$}\NormalTok{k\_n}
\NormalTok{l\_k }\OtherTok{\textless{}{-}} \FunctionTok{lim}\NormalTok{(k\_n, n, }\ConstantTok{Inf}\NormalTok{)}
\end{Highlighting}
\end{Shaded}

The findings above are: \[
k_n = \frac{n}{n r - r + 1}, \quad
l_k = \frac{1}{r} .
\]
\end{frame}

\begin{frame}[fragile]
It is illustrative to supplement the symbolic computations above with
numerical evaluations.

\begin{Shaded}
\begin{Highlighting}[]
\NormalTok{dat }\OtherTok{\textless{}{-}} \FunctionTok{expand.grid}\NormalTok{(}\AttributeTok{r=}\FunctionTok{c}\NormalTok{(.}\DecValTok{1}\NormalTok{, .}\DecValTok{2}\NormalTok{, .}\DecValTok{5}\NormalTok{), }\AttributeTok{n=}\FunctionTok{c}\NormalTok{(}\DecValTok{10}\NormalTok{, }\DecValTok{50}\NormalTok{))}
\NormalTok{k\_fun }\OtherTok{\textless{}{-}} \FunctionTok{as\_func}\NormalTok{(k\_n)}
\NormalTok{dat}\SpecialCharTok{$}\NormalTok{k\_n }\OtherTok{\textless{}{-}} \FunctionTok{k\_fun}\NormalTok{(}\AttributeTok{r=}\NormalTok{dat}\SpecialCharTok{$}\NormalTok{r, }\AttributeTok{n=}\NormalTok{dat}\SpecialCharTok{$}\NormalTok{n)}
\NormalTok{dat}\SpecialCharTok{$}\NormalTok{l\_k }\OtherTok{\textless{}{-}} \DecValTok{1}\SpecialCharTok{/}\NormalTok{dat}\SpecialCharTok{$}\NormalTok{r}
\NormalTok{dat}
\end{Highlighting}
\end{Shaded}

\begin{verbatim}
    r  n  k_n l_k
1 0.1 10 5.26  10
2 0.2 10 3.57   5
3 0.5 10 1.82   2
4 0.1 50 8.47  10
5 0.2 50 4.63   5
6 0.5 50 1.96   2
\end{verbatim}

Shows that even a moderate correlation reduces the effective sample size
substantially
\end{frame}

\hypertarget{extending-caracas}{%
\section{Extending caracas}\label{extending-caracas}}

\begin{frame}[fragile]{Extending caracas}
Only small part of Sympy is interfaced from \texttt{caracas} but it is
easy to extend \texttt{caracas}. For example: polynomial division

\begin{Shaded}
\begin{Highlighting}[]
\FunctionTok{def\_sym}\NormalTok{(x)}
\NormalTok{f }\OtherTok{=} \DecValTok{5} \SpecialCharTok{*}\NormalTok{ x}\SpecialCharTok{\^{}}\DecValTok{2} \SpecialCharTok{+} \DecValTok{10} \SpecialCharTok{*}\NormalTok{ x }\SpecialCharTok{+} \DecValTok{3}
\NormalTok{g }\OtherTok{=} \DecValTok{2} \SpecialCharTok{*}\NormalTok{ x }\SpecialCharTok{+} \DecValTok{2}
\end{Highlighting}
\end{Shaded}

\[
f = 5 x^{2} + 10 x + 3; \; g=2 x + 2
\]

Find \(f / g\); that is find \(q\) and \(r\) such that \[
  f = q g + r
\]
\end{frame}

\begin{frame}[fragile]
The Sympy function for polynomial division is \texttt{div} and it can be
invoked via the \texttt{caracas} function \texttt{sympy\_func}.

\begin{Shaded}
\begin{Highlighting}[]
\NormalTok{v }\OtherTok{\textless{}{-}} \FunctionTok{sympy\_func}\NormalTok{(f, }\StringTok{"div"}\NormalTok{, g)}
\NormalTok{v}
\end{Highlighting}
\end{Shaded}

\begin{verbatim}
[[1]]
[c]: 5*x   5
     --- + -
      2    2

[[2]]
[c]: -2
\end{verbatim}

\begin{Shaded}
\begin{Highlighting}[]
\NormalTok{(v[[}\DecValTok{1}\NormalTok{]] }\SpecialCharTok{*}\NormalTok{ g }\SpecialCharTok{+}\NormalTok{ v[[}\DecValTok{2}\NormalTok{]]) }\SpecialCharTok{|\textgreater{}} \FunctionTok{simplify}\NormalTok{()}
\end{Highlighting}
\end{Shaded}

\begin{verbatim}
[c]:          2    
     5*(x + 1)  - 2
\end{verbatim}
\end{frame}

\hypertarget{wrapping-up}{%
\section{Wrapping up}\label{wrapping-up}}

\begin{frame}{Wrapping up}
\begin{itemize}
\item
  The caracas package for R provides computer algebra / symbolic math

  \begin{itemize}
  \tightlist
  \item
    At your fingertips within R using R syntax
  \item
    For example: derivatives, integration, sums, limits, linear algebra,
    solving equations
  \end{itemize}
\item
  Package can easily be extended
\item
  Easy transition from symbolic expression to numerical expressions
\item
  Easy generation of math expressions for documents (used in this
  presentation).
\item
  See \url{https://r-cas.github.io/caracas/} for vignettes and other
  info.
\item
  Thank you for your attention!
\end{itemize}
\end{frame}



\end{document}
