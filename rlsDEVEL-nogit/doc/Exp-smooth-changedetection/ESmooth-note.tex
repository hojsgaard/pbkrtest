%% This file has been modified by igattr
%% Do not edit manually
\documentclass[11pt]{article}
\usepackage[T1]{fontenc}
\usepackage{a4wide}
\usepackage{boxedminipage,color}

\title{Exponential smoothing and change point
  detection}
\author{S�ren H�jsgaard}
\def\eref#1{(\ref{eq:#1})}
\def\DS{S^{[2]}}
\def\TS{S^{[3]}}

\usepackage{Sweave}
\begin{document}
\maketitle
\parindent0pt\parskip5pt

 %s�t dir til "fig" og prefix til "bar" for figurer
\setkeys{Gin}{width=0.9\textwidth} %s�t figurst�rrelse i Sweave
\renewenvironment{Schunk}{\linespread{.85}\scriptsize}{}

%% Efter preamble
\definecolor{myGray}{rgb}{0.95,0.95,0.95}
\makeatletter
\renewenvironment{Schunk}{
  \begin{lrbox}{\@tempboxa}
    \begin{boxedminipage}
      {\columnwidth}\scriptsize}
    {\end{boxedminipage}
  \end{lrbox}%
  \colorbox{myGray}{\usebox{\@tempboxa}}
}
\makeatother

\section{Introduction}
\label{sec:intro}


\section{Simple exponential smoothing}
\label{sec:ses}

Given is a time series $\{x_t; t=0, 1,\dots,T\}$ observed at discrete
equidistant times.  Exponential smoothing or single exponential
smoothing (hereafter SES) is a dynamic smoothing of data which for
$0<\alpha<1$ is defined as
\begin{equation}
  \label{eq:es1}
  S_{t} 
  = \alpha x_{t} + (1-\alpha) S_{t-1}
  = S_{t-1} + \alpha (x_{t}-S_{t-1})
\end{equation}

Suppose that the underlying process is ``almost constant'', that is
$x_{t} = a$.  Then the forecast of $x$ at time $t+h$ based on data
upto (and including) time $t$ is
\begin{equation}
  \label{eq:es2}
  \hat x_{t+h|t} = S_{t}
\end{equation}

Write $\hat x_{t|t}=S_{t}$ as $\hat x_{t}$ which we
may think of as the current fitted value. 
Notice that
(\ref{eq:es1}) also reads $S_{t}=S_{t-1} + \alpha e_{t}$ where
$e_{t}=x_{t}-S_{t-1}=x_{t}-\hat x_{t|t-1}$ is the forecast error. 


We may
write $S_{t}$ given in (\ref{eq:es1})  as 
a linear function of data. Letting $\beta=1-\alpha$ we
get
\begin{equation}
  \label{eq:es3}
  S_{t} = \alpha \sum_{k=0}^{t-1} \beta^kx_{t-k} + \beta^t x(0)
\end{equation}

Hence, the influence of the initial value becomes negligible for large
$t$. Moreover, the sequence of weights $(\alpha, \alpha\beta,
\alpha\beta^2, \dots, \alpha\beta^{t-1})$ decrease exponentially and
will in the limit sum to one. Hence $S_{t}$ can be regarded as a
probability weighted average of data where recent observations carry
the highest weight.

In practice a value for $\alpha$ must be chosen. A data--driven way is
to choose $\alpha$ so as to minimize sum of squares of 1--step ahead
forecasts, that is: 
\begin{displaymath}
  \sum_{t=2}^T (x_{t}-\hat x_{t|t-1})^2
\end{displaymath}




\section{Double exponential smoothing}
\label{sec:des}


Double exponential smoothing (hereafter DES) is SES applied to $S_{t}$,
that is
\begin{equation}
  \label{eq:des1}
  \DS_{t} = \alpha S_{t} + (1-\alpha)\DS_{t-1}
\end{equation}

Suppose that the true process is ``almost linear'', that is
\begin{displaymath}
  x_{t} = a + b t
\end{displaymath}

Then $S_{t}$ will be 
\begin{displaymath}
 S_{t}= a + bt+ \frac{\beta}{\alpha}b +  O \left( 
    \left( 1-{\alpha} \right) ^{t} \right) 
\end{displaymath}

Then SES will be biased in the
sense that 
\begin{equation}
  \label{eq:des3}
  x_{t} - S_{t} = b \frac{\beta}{\alpha} \mbox{ for }  t\rightarrow \infty 
\end{equation}
or $x_{t}\approx S_{t}+ b \frac{\beta}{\alpha}$. 

By the same argument we get that smoothing $S_{t}$ again gives
\begin{displaymath}
\DS_{t} =  (a - 2b \frac \beta \alpha) + bt
\mbox{ for } t\rightarrow\infty   
\end{displaymath}

Hence $\DS_{t}-S_{t} \rightarrow b \frac \beta \alpha$.
From these considerations and \eref{des3} we therefore have 
\begin{eqnarray}
  \label{eq:ses4}
  b &=& \{S_{t}-\DS_{t}\}\frac\alpha\beta  \\
  x_{t} &=& S_{t}- b\beta/\alpha = S_{t}+(S_{t}-\DS_{t})=2S_{t}-\DS_{t}
\end{eqnarray}

Hence natural estimates of levels and slopes become
\begin{eqnarray}
  \label{eq:des6}
    \hat x_{t}  &=& S_{t}+\{S_{t}-\DS_{t}\}=2S_{t}-\DS_{t}\\
    \hat b_{t}  &=& \{S_{t}-\DS_{t}\}\frac \alpha\beta
\end{eqnarray}

Similarly, the forecasts become
\begin{equation}
  \label{eq:des7}
  \hat x(t+h|t) = \hat x_{t} + \hat b_{t} h
\end{equation}

\section{Triple exponential smoothing}
\label{sec:triple-expon-smooth}

Triple exponential smoothing (hereafter TES) is SES applied to $\DS_{t}$,
that is
\begin{equation}
  \label{eq:des1}
  \TS_{t} = \alpha \DS_{t} + (1-\alpha)\TS_{t-1}
\end{equation}

Suppose the original proces is quadratic
\begin{displaymath}
  x_{t} = a + b t + c t^2
\end{displaymath}

Then $S_{t}$ will be 
\begin{displaymath}
 S_{t}= 
 a + bt + ct^2 +
 2\,{\frac { \left( -1+{\alpha} \right) ct}{{\alpha}}}+{\frac {-b
{\alpha}-3\,c{\alpha}+2\,c+c{{\alpha}}^{2}+b{{\alpha}}^{2}
}{{{\alpha}}^{2}}}+O \left(  \left( 1-{\alpha} \right) ^{t}
 \right) 
\end{displaymath}
or
\begin{displaymath}
  x_{t}-S_{t}= 2\,{\frac { \left( -1+{\alpha} \right) ct}{{\alpha}}}+{\frac {-b
{\alpha}-3\,c{\alpha}+2\,c+c{{\alpha}}^{2}+b{{\alpha}}^{2}
}{{{\alpha}}^{2}}}+O \left(  \left( 1-{\alpha} \right) ^{t}
 \right) 
\end{displaymath}



% \section{Diagnosis}
% \label{sec:diagnosis}

% In the following we will denote fitted values from SES by $x_0()$ and
% from DES by $x_1()$ (refering to 0'th and 1'st degree polynomials). 

% A slow growth can in principle be detected by testing if $b=0$. If we
% assume the 


% \newpage

% @ 
% <<>>=
% source('../../EsmoothFUN.R')
% @ %def 


% \section{Examples}
% \label{sec:examples1}

% Consider the following scenarios: 1) steady state, 2) abrupt level
% change 3) an outlier and 4) a gradual change. Black lines are fitted
% values from SES, red lines are from DES. 


% @ 
% <<chunk0,eval=F,echo=F>>=
% plot(y,ylim=c(5,25),pch='.',cex=2)
% lines(mu, col='blue')
% lines(.ses(y,alpha=alpha),col=1)
% lines(.des(y,alpha=alpha),col=2)
% @ %def 

% @ 
% <<fig=T,height=4,echo=F>>=
% #set.seed(12345)
% eps <- rnorm(60, sd=1.5)
% vl <- function(){abline(v=ch,col='cyan')}
% hl <- function(){abline(h=0,col='pink')}
% alpha <- .15
% par(mfcol=c(2,2), mar=c(1, 4, 0, 1) + 0.1)
% ###########################################

% mu1 <- rep(10,60)
% y1  <- mu1 + eps
% ch1 <- NA
% ch  <- ch1
% mu <- mu1
% y <- y1

% <<chunk0>>

% mu2 <- rep(10,60)
% mu2[31] <- 20
% ch2 <- 31
% y2  <- mu2 + eps
% mu <- mu2
% y <- y2
% ch <- ch2
% <<chunk0>>

% mu3 <- c(rep(10,30), rep(20,30))
% ch3 <- 31
% y3  <- mu3 + eps
% mu <- mu3
% y <- y3
% ch <- ch3
% <<chunk0>>
  
% mu4 <- c(rep(10,20), 10+(1:20)/2, rep(20,20))
% ch4 <- c(21,41)
% y4  <- mu4 + eps
% mu <- mu4
% y <- y4
% ch<-ch4
% <<chunk0>>
% @ %def 

% \newpage


% \newpage

% From $\hat x_{t}$ and $\hat x(t|t-h)$ (for $h=1$) we 
% can perhaps distinguish between scenario 2) and 3): Look at the differences $\hat x_{t}-\hat
% x(t|t-h)$ (which is the difference in forecast errors): If that
% difference is practically zero at time $t+1$ then the extreme observation at
% time $t$ is an ``outlier''

% @ 
% <<forecerr,eval=F,echo=F>>=
% s1 <- .des(y,alpha=alpha)

% ferr<-as.data.frame(cbind(s1$y.obs-s1$y, predict(s1,hh=hh)$ferr))
% ferr$diff <- ferr[,1]-ferr[,2]
% se <- sd(ferr$diff[1:15],na.rm=T)
% dif <- ferr$diff
% plot(dif, pch='.',cex=3,ylim=c(-3,6),type='l'); vl();
% abline(h=c(-2*se,2*se))

% #dif <- s1$y.obs-s1$y - predict(s1,hh=4)$ferr
% #lines(dif,col=2,pch='.',cex=3)
% @ %def 


% @ 
% <<fig=T,height=4,echo=F>>=
% hh    <- 1
% par(mfcol=c(2,2), mar=c(1, 4, 0, 1) + 0.1)
% ###########################################

% y<-y1
% ch<-ch1
% <<forecerr>>
% y<-y2
% ch<-ch2
% <<forecerr>>
% y<-y3
% ch<-ch3
% <<forecerr>>
% y<-y4
% ch<-ch4
% <<forecerr>>
% @ %def 


% \newpage 

% If we assume data to be stationary at the beginning we can estimate
% mean and variance for the level and use this for subsequent tests for
% level change. 

% @ 
% <<levelc,eval=F,echo=F>>=
% s1 <- .des(y,alpha=alpha)

% m.y <- mean(s1$y[1:20])
% v.y <- var(s1$y[1:20])


% w  <- (s1$y - m.y)/sqrt(v.y)
% plot(w,pch='.',cex=3,ylim=c(-3,30))
% abline(h=c(-2*sqrt(v.y),2*sqrt(v.y)));vl()
% @ %def 


% @ 
% <<fig=T,height=4,echo=F>>=
% hh    <- 1
% par(mfcol=c(2,2), mar=c(1, 4, 0, 1) + 0.1)
% ###########################################

% y<-y1
% ch<-ch1
% <<levelc>>
% y<-y2
% ch<-ch2
% <<levelc>>
% y<-y3
% ch<-ch3
% <<levelc>>
% y<-y4
% ch<-ch4
% <<levelc>>
% @ %def 

% % w  <- (s1$y-m.y)^2/v.y
% % plot(1-pchisq(w,1),pch='.',cex=3)
% % abline(h=0.05);vl()










\appendix
\section{Linear filters and the z--transform}
\label{sec:lfzt}

We can look at \eref{es3} as a linear time invariant filter
\begin{equation}
  \label{eq:es4}
  S_{t} = \sum_{j=-\infty}^\infty h(j)x(t-j)
\end{equation}
where $h(j)=\alpha\beta^j$ for $j\ge 0$ and zero otherwise. Hence  $h(j)$ is
the weight given to $x(t-j)$ in forming $S_{t}$ or, equivalently, 
the effect of an observation $x_{t}$ at time $t$ will $j$ time steps
later be $x_{t}\alpha\beta^j$.

Let $\{h(j)\}$ for $j=-\infty \dots, \infty$
be a collection of weights and define
\begin{equation}
  \label{eq:lf1}
  y_{t} = \sum_{j=-\infty}^\infty h(j)x(t-j)
\end{equation}

Then (\ref{eq:lf1}) is a linear time invariant filter. As an example,
suppose that $h(0)$, $h(1)$ and $h(2)$ are non--zero and all other
weights are zero. Then we get
\begin{equation}
  \label{eq:lf2}
  y_{t} = h(0) x_{t} + h(1)x_{t-1} + h(2)x(t-2)
\end{equation}

Equation (\ref{eq:lf1}) is called a convolution of $x()$ and $h()$
and we write this briefly as 
\begin{equation}
  \label{eq:lf3}
  y_{t} = x_{t}*h_{t}
\end{equation}

We want to study what $y_{t}$ looks like for different choices of
$x_{t}$ and $h_{t}$. In doing so we use the z--transform: Let $f(n)$ be
a function defined for integer values of $n$. The z--transform is 
\begin{equation}
  \label{eq:lf4}
  F(z) = \sum_{n=-\infty}^\infty f(n)z^n
\end{equation}
for complex values $z$. Notice that we write the function in lower
case and the  z--transform of the function in upper case. Tables exist
for pairs of functions and their z--transform. 


We apply the z--transform to \eref{lf2}. Change variables from as
$m=t-j$ so that $j=t-m$ so that 
\begin{equation}
  \label{eq:lf5}
  y(n) = \sum_{m=-\infty}^\infty x(m) h(n-m)  
\end{equation}

Now apply the z--transform to $y(n)$ and we get (where all summations
are from $-\infty$ to $\infty$):
\begin{eqnarray}
  Y(z) 
  &=& \sum_n z^n \sum_m x(m) h(n-m) \nonumber \\
  &=& \sum_m \{z^m x(m) \sum_n z^{n-m}  h(n-m)\} \nonumber  \\
  &=& \sum_m \{z^m x(m) \sum_p z^{p}  h(p)\} \nonumber \\
  &=& H(z) X(z)
  \label{eq:lf6}  
\end{eqnarray}

Hence, all we need to do is to find $H()$ and $X()$ (typically by
looking into a table), carry out the
multiplication in \eref{lf6} to find $Y(z)$ and then find $y_{t}$ by
looking into the table again.  

There are linearities to be exploited in this connection: If
$f(n)=\sum_u \alpha_u f_u(n)$ then $F(z) =\sum_u \alpha_uF_u(z)$. Thus
if $x_{t}= \sum_u \alpha_u(t)$ and $h(j)=\sum_v \beta_v h_v(j)$ then
\begin{equation}
  \label{eq:lf7}
  Y(z) = \sum_{u,v} \alpha_u\beta_v X_u(z)H_v(z) = \sum_s \gamma_s
  Y_s(z),
\end{equation}
say. Exploiting the linearity again then gives
\begin{equation}
  \label{eq:lf8}
  y_{t} = \sum_s \gamma_s y_s(t)
\end{equation}

\end{document}

% \subsection{Steady--state}
% \label{sec:xxx}

% @ 
% <<fig=T,height=5,echo=F>>=
% source("EsmoothFUN.q")
% mu <- rep(10,60)
% y  <- mu + eps
% ch <- NA
% <<chunk1>>
% @ %def 

% \subsection{Outlier}
% \label{sec:xxx}

% @ 
% <<fig=T,height=5,echo=F>>=
% mu <- rep(10,60)
% mu[31] <- 20
% y  <- mu + eps
% ch <- 31
% <<chunk1>>
% @ %def 


% \subsection{Abrupt change}
% \label{sec:xxx}

% Now see what happens if the level suddenly changes. 

% @ 
% <<fig=T,height=5,echo=F>>=
% mu <- c(rep(10,30), rep(20,30))
% ch <- 31
% y  <- mu + eps
% <<chunk1>>
% @ %def 


% \subsection{Gradual change}
% \label{sec:xxx}

% Now study what happens when there is a gradual change:

% @ 
% <<fig=T,height=5,echo=F>>=
% mu <- c(rep(10,20), 10+(1:20)/2, rep(20,20))
% ch <- c(21,41)
% y  <- mu + eps
% <<chunk1>>
% @ %def 







% @ 
% <<chunk1,eval=F,keep.source=T,echo=F>>=
% par(mfcol=c(5,2), mar=c(1, 4, 0, 1) + 0.1)

% alpha <- .2
% hh    <- 5
% s0 <- .ses(y,alpha=alpha)
% s1 <- .des(y,alpha=alpha)

% www <- 1:20
% www <- c(1,4,7,11,15,18)
% m.y <- mean(s1$y[www])
% v.y <- var (s1$y[www])
% m.b <- mean(s1$b[www])
% v.b <- var (s1$b[www])

% plot(y, ylim=range(y,s0$y,s1$y));vl()
% lines(s0,col=1)
% lines(s1,col=2)

% rmat <- cbind(resid(s0), resid(s1))
% matplot(rmat, type='l'); vl(); hl()

% ferr<-as.data.frame(cbind(predict(s0,hh=hh)$ferr, predict(s1,hh=hh)$ferr))
% ferr$dif.ferr <- ferr[,2]-ferr[,1]
% matplot(ferr, type='l',lty=1); vl();hl()

% parmest <- do.call(cbind,s1)[,2:3]
% matplot(parmest,type='l',lty=1); vl()
% matplot(scale(parmest),type='l',lty=1); vl()

% parmest2 <- parmest
% parmest2[,1] <- parmest2[,1]/sqrt(v.y)
% parmest2[,2] <- parmest2[,2]/sqrt(v.b)
% matplot(parmest2,type='l',lty=1); vl()

% w.y <- (s1$y-m.y)^2/v.y
% w.b <- (s1$b-m.b)^2/v.b

% p.y <- pchisq(w.y,1)
% p.b <- pchisq(w.b,1)
% ps.y <- pchisq(.ses(w.y)$y,1)
% ps.b <- pchisq(.ses(w.b)$y,1)
% plot(p.y); vl()
% lines(ps.y)
% plot(p.b);vl()
% lines(ps.b)

% db <- c(NA, diff(s1$b))
% m.db <- mean(db[www],na.rm=T)
% v.db <- var(db[www],na.rm=T)
% plot(db,type='l'); vl();hl()
% abline(h=c(m.db, m.db-1*sqrt(v.db), m.db+1*sqrt(v.db)),col=c(4,3,3))
% @ %def 




% @ 
% <<forec,eval=F,echo=F>>=
% s0 <- .ses(y,alpha=alpha)
% s1 <- .des(y,alpha=alpha)
% plot(y,pch='.',cex=2);vl()
% lines(s1)
% lines(predict(s1,hh=hh),col=2)
% @ %def 


% @ 
% <<fig=T,height=4,echo=F>>=
% hh    <- 1
% par(mfcol=c(2,2), mar=c(1, 4, 0, 1) + 0.1)
% ###########################################

% y<-y1
% ch<-ch1
% <<forec>>
% y<-y2
% ch<-ch2
% <<forec>>
% y<-y3
% ch<-ch3
% <<forec>>
% y<-y4
% ch<-ch4
% <<forec>>
% @ %def 

