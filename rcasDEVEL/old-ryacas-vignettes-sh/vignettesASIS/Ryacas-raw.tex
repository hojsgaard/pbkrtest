\documentclass[]{article} 

	
\usepackage[utf8]{inputenc}
\usepackage{hyperref,a4,color,boxedminipage,Sweave}
%\VignetteIndexEntry{Introduction to Ryacas}
%\VignettePackage{Ryacas}
%\VignetteDepends{XML}

% Definitions
\newcommand{\slan}{{\tt S}}
\newcommand{\rlan}{{\tt R}}
\newcommand{\ryacas}{{\tt Ryacas}}
\newcommand{\yacas}{{\tt yacas}}
\newcommand{\code}[1]{{\tt #1}}
\def\yac{\texttt{Yacas}}
\def\R{\texttt{R}}
\def\sym{\texttt{Sym}}

\setlength{\parindent}{0in}
\setlength{\textwidth}{140mm}
\setlength{\oddsidemargin}{10mm}

\title{\ryacas{} -- an \rlan{} interface to the \yacas{} computer
  algebra system}
\author{Rob Goedman, \and
Gabor Grothendieck,  \and
S<f8>ren H<f8>jsgaard,  \and
Ayal Pinkus}


\begin{document}
\maketitle
\tableofcontents
\parskip5pt


\renewenvironment{Schunk}{\begin{center}
    \small
    \begin{boxedminipage}{0.95\textwidth}}{
    \end{boxedminipage}\end{center}}

%     \RecustomVerbatimEnvironment{Sinput}{Verbatim}%
%         {fontsize=\small,frame=single,framerule=1pt,
%           rulecolor=\color{red},   fillcolor=\color{yellow} }
%     \RecustomVerbatimEnvironment{Soutput}{Verbatim}%
%         {fontsize=\scriptsize, frame=single,framerule=0.1pt}


\section{Introduction} 
\label{sec:introduction}

\ryacas{} makes the \yacas{} computer algebra system available from
within \rlan. The name \yacas{} is short for ``Yet Another Computer Algebra
System''. The \yacas{} program 
is developed by Ayal Pinkhuis (who is also the maintainer)
and others, and is available at
\href{http://yacas.sourceforge.net}{http://yacas.sourceforge.net} for various
platforms. There is a comprehensive documentation (300+ pages) of
\yacas{} (also available at
\href{http://yacas.sourceforge.net}{http://yacas.sourceforge.net}) and the
documentation contains many examples.

% The examples given here are largely taken from the \yacas{}
% documentation (especially from the introductory chapter) but are
% organised differently.



\section{\R\ expressions,  \yacas\ expressions and \sym\ objects} 
\label{sec:hood}

The \ryacas{} package works by sending ``commands''
to \yacas{} which makes the calculations and returns the result
to \R{}. There are various different formats of the return value as well

\subsection{\R\ expressions}

A call to \yacas\ may be in the form of an \R\ expression which
involves valid R calls, symbols or constants (though not all valid \R\
expressions are valid). For example:
\begin{Schunk}
\begin{Sinput}
> exp1<- yacas(expression(Factor(x^2-1)))
\end{Sinput}
\begin{Soutput}
[1] "Starting Yacas!"
expression((x + 1) * (x - 1))
\end{Soutput}
\end{Schunk}

The result \code{exp1} is not an expression in the \R\ sense but an
object of class \code{"yacas"}. 
To evaluate the resulting expression numerically, we can do
\begin{Schunk}
\begin{Sinput}
> Eval(exp1, list(x=4))
\end{Sinput}
\begin{Soutput}
[1] 15
\end{Soutput}
\end{Schunk}

\subsection{\yacas\ expressions}

Some commands are not proper \R\ expressions. For example, 
typing 
\begin{verbatim}
  yacas(expression(D(x)Sin(x)))
\end{verbatim}
produces an error. 
For such cases we can make
a specification using the \yacas{} syntax:
\begin{Schunk}
\begin{Sinput}
> yacas("D(x)Sin(x)")
\end{Sinput}
\begin{Soutput}
expression(cos(x))
\end{Soutput}
\end{Schunk}


\subsection{\sym\ objects}

Probably the most elegant way of working with \yacas\ is by using
\sym\ objects.
A \sym\ object is a \yacas\ character string that has the "Sym" class.
One can combine \sym\ objects with other \sym\ objects as well as to
other \R\ objects using \code{+}, \code{-} and other similar \R\
operators.

The function \code{Sym(x)} coerces an object \code{x} to a \sym\ object by
first coercing it to character and then changing its class to "Sym":
\begin{Schunk}
\begin{Sinput}
> x<- Sym("x")
\end{Sinput}
\begin{Soutput}
expression(x)
\end{Soutput}
\end{Schunk}

Operations on \sym\ objects lead to new \sym\ objects:
\begin{Schunk}
\begin{Sinput}
> x+4
\end{Sinput}
\begin{Soutput}
expression(x + 4)
\end{Soutput}
\end{Schunk}

One can apply \code{sin}, \code{cos}, \code{tan}, \code{deriv}, \code{Integrate}
and other provided functions to \sym\ objects. For example:
\begin{Schunk}
\begin{Sinput}
> Integrate(sin(x), x)
\end{Sinput}
\begin{Soutput}
expression(-cos(x))
\end{Soutput}
\end{Schunk}
In this way the communication with \yacas\ is ``tacit''.

It is important to note the difference between the \R\ name \code{x}
and the symbol \code{"x"} as illustrated below:
\begin{Schunk}
\begin{Sinput}
> x<- Sym("xs")
\end{Sinput}
\begin{Soutput}
expression(xs)
\end{Soutput}
\begin{Sinput}
> x
\end{Sinput}
\begin{Soutput}
expression(xs)
\end{Soutput}
\begin{Sinput}
> x+4
\end{Sinput}
\begin{Soutput}
expression(xs + 4)
\end{Soutput}
\begin{Sinput}
> Eval(x+4, list(xs=5))
\end{Sinput}
\begin{Soutput}
[1] 9
\end{Soutput}
\end{Schunk}

The convention in the following is 1) that \sym\ objects match with
their names that they end with an 's', e.g.
\begin{Schunk}
\begin{Sinput}
> xs <- Sym('xs')
\end{Sinput}
\end{Schunk}


\section{A sample session} 
\label{sec:samplesession}

Algebraic calculations:
\begin{Schunk}
\begin{Sinput}
> yacas(expression((10 + 2) * 5 + 7^7))
\end{Sinput}
\begin{Soutput}
expression(823603)
\end{Soutput}
\begin{Sinput}
> yacas(expression(1/14+5/21* (30- 1+ 1/2)))
\end{Sinput}
\begin{Soutput}
expression(149/21)
\end{Soutput}
\end{Schunk}

\begin{Schunk}
\begin{Sinput}
> #(Sym(10) + Sym(2)) * Sym(5) + Sym(7) ^ Sym(7)
> Sym("10 * 2") * 5 + Sym(7) ^ 7
\end{Sinput}
\begin{Soutput}
expression(823643)
\end{Soutput}
\begin{Sinput}
> #(Sym(10) + 2) * 5 + Sym(7) ^ 7
> #Sym("(10+2)*5 + 7^7")
> Sym("1/14 + 5/21 * (30 - 1+1/2)")
\end{Sinput}
\begin{Soutput}
expression(149/21)
\end{Soutput}
\end{Schunk}


Numerical evaluations:
\begin{Schunk}
\begin{Sinput}
> yacas(expression(N(-12/2)))
\end{Sinput}
\begin{Soutput}
expression(-6)
\end{Soutput}
\end{Schunk}
\begin{Schunk}
\begin{Sinput}
> Sym("-12/2")
\end{Sinput}
\begin{Soutput}
expression(-6)
\end{Soutput}
\begin{Sinput}
> #Eval(Sym("-12/2"))
\end{Sinput}
\end{Schunk}


Symbolic expressions:
\begin{Schunk}
\begin{Sinput}
> yacas(expression(Factor(x^2-1)))
\end{Sinput}
\begin{Soutput}
expression((x + 1) * (x - 1))
\end{Soutput}
\begin{Sinput}
> exp1 <- expression(x^2 + 2 * x^2)
> exp2 <- expression(2 * exp0)
> exp3 <- expression(6 * pi * x)
> exp4 <- expression((exp1 * (1 - sin(exp3))) / exp2)
> yacas(exp4)
\end{Sinput}
\begin{Soutput}
expression(3 * x^2 * (1 - sin(6 * x * pi))/(2 * exp0))
\end{Soutput}
\end{Schunk}

\begin{Schunk}
\begin{Sinput}
> Factor(xs^2-1)
\end{Sinput}
\begin{Soutput}
expression((xs + 1) * (xs - 1))
\end{Soutput}
\begin{Sinput}
> exp1 <- xs^2 + 2 * xs^2
> exp0 <- Sym("exp0")
> exp2 <- 2 * Sym(exp0)
> exp3 <- 6 * Pi * xs
> exp4 <- exp1 * (1 - sin(exp3)) / exp2
> exp4
\end{Sinput}
\begin{Soutput}
expression(3 * xs^2 * (1 - sin(6 * xs * pi))/(2 * exp0))
\end{Soutput}
\end{Schunk}



Combining symbolic and numerical expressions:
\begin{Schunk}
\begin{Sinput}
> yacas(expression(N(Sin(1)^2 + Cos(x)^2)))
\end{Sinput}
\begin{Soutput}
expression(cos(x)^2 + 0.7080734182)
\end{Soutput}
\end{Schunk}

\begin{Schunk}
\begin{Sinput}
> N(sin(1)^2 + cos(xs)^2)
\end{Sinput}
\begin{Soutput}
expression(cos(xs)^2 + 0.708073418273571)
\end{Soutput}
\end{Schunk}


Differentiation:
\begin{Schunk}
\begin{Sinput}
> yacas("D(x)Sin(x)")
\end{Sinput}
\begin{Soutput}
expression(cos(x))
\end{Soutput}
\end{Schunk}
\begin{Schunk}
\begin{Sinput}
> deriv(sin(xs), xs)
\end{Sinput}
\begin{Soutput}
expression(cos(xs))
\end{Soutput}
\end{Schunk}

Integration: 

\begin{Schunk}
\begin{Sinput}
> yacas("Integrate(x,a,b)Sin(x)")
\end{Sinput}
\begin{Soutput}
expression(cos(a) - cos(b))
\end{Soutput}
\end{Schunk}

\begin{Schunk}
\begin{Sinput}
> as <- Sym("as")
> bs <- Sym("bs")
> Integrate(sin(xs), xs, as, bs)
\end{Sinput}
\begin{Soutput}
expression(cos(as) - cos(bs))
\end{Soutput}
\end{Schunk}


Expanding polynomials:
\begin{Schunk}
\begin{Sinput}
> yacas("Expand((1+x)^3)")
\end{Sinput}
\begin{Soutput}
expression(x^3 + 3 * x^2 + 3 * x + 1)
\end{Soutput}
\end{Schunk}

\begin{Schunk}
\begin{Sinput}
> Expand((1+xs)^3)
\end{Sinput}
\begin{Soutput}
expression(xs^3 + 3 * xs^2 + 3 * xs + 1)
\end{Soutput}
\end{Schunk}

Taylor expansion:
\begin{Schunk}
\begin{Sinput}
> yacas("texp := Taylor(x,0,3) Exp(x)")
\end{Sinput}
\begin{Soutput}
expression(x + x^2/2 + x^3/6 + 1)
\end{Soutput}
\end{Schunk}

\begin{Schunk}
\begin{Sinput}
> texp <- Taylor(exp(xs), xs, 0, 3)
\end{Sinput}
\begin{Soutput}
expression(xs + xs^2/2 + xs^3/6 + 1)
\end{Soutput}
\end{Schunk}


Printing the result in nice forms:
\begin{Schunk}
\begin{Sinput}
> yacas("PrettyForm(texp)") 
\end{Sinput}
\begin{Soutput}
     2    3    
    x    x     
x + -- + -- + 1
    2    6     
\end{Soutput}
\begin{Sinput}
> yacas("TeXForm(texp)", retclass = "unquote")
\end{Sinput}
\begin{Soutput}
$x + \frac{x ^{2}}{2}  + \frac{x ^{3}}{6}  + 1$
\end{Soutput}
\end{Schunk}

\begin{Schunk}
\begin{Sinput}
> PrettyForm(texp) 
\end{Sinput}
\begin{Soutput}
       2     3    
     xs    xs     
xs + --- + --- + 1
      2     6     
\end{Soutput}
\begin{Sinput}
> TeXForm(texp)
\end{Sinput}
\begin{Soutput}
expression("$xs + \frac{xs ^{2}}{2}  + \frac{xs ^{3}}{6}  + 1$")
\end{Soutput}
\end{Schunk}



\section{Simple \yac\ calculations}


\subsection{Setting and clearing a variable}

The function \code{Set()} and the operator \code{:=} can both be used to assign
values to global variables. 
\begin{Schunk}
\begin{Sinput}
> yacas("n := (10 + 2) * 5")
\end{Sinput}
\begin{Soutput}
expression(60)
\end{Soutput}
\begin{Sinput}
> yacas("n := n+n")
\end{Sinput}
\begin{Soutput}
expression(120)
\end{Soutput}
\begin{Sinput}
> #yacas("Set(z, Cos(a))")
> #yacas("z+z")
\end{Sinput}
\end{Schunk}

The same can be achieved with \sym\ objects: Consider:
\begin{Schunk}
\begin{Sinput}
> Set(ns, (10 + 2) * 5)
\end{Sinput}
\begin{Soutput}
expression(60)
\end{Soutput}
\end{Schunk}

Now \code{ns} exists as a variable in \yacas (and we
can make computations on this variable as above).
However we have no handle on
this variable in \R. Such a handle is obtained with
\begin{Schunk}
\begin{Sinput}
> ns <- Sym("ns")
\end{Sinput}
\end{Schunk}
Now the \R\ variable \code{ns} refers to the \yacas\ variable
\code{ns} and we can make  calculations directly from \R, e.g:
\begin{Schunk}
\begin{Sinput}
> Set(ns,123)
\end{Sinput}
\begin{Soutput}
expression(123)
\end{Soutput}
\begin{Sinput}
> ns
\end{Sinput}
\begin{Soutput}
expression(123)
\end{Soutput}
\begin{Sinput}
> ns^2
\end{Sinput}
\begin{Soutput}
expression(15129)
\end{Soutput}
\end{Schunk}

Likewise:
\begin{Schunk}
\begin{Sinput}
> as <- Sym("as")
> zs <- Sym("zs")
> Set(zs, cos(as))
\end{Sinput}
\begin{Soutput}
expression(cos(as))
\end{Soutput}
\begin{Sinput}
> zs + zs
\end{Sinput}
\begin{Soutput}
expression(2 * cos(as))
\end{Soutput}
\end{Schunk}


o clear a variable
binding execute \code{Clear()}:
\begin{Schunk}
\begin{Sinput}
> yacas(expression(n))
\end{Sinput}
\begin{Soutput}
expression(120)
\end{Soutput}
\begin{Sinput}
> yacas("Clear(n)")
\end{Sinput}
\begin{Soutput}
expression(TRUE)
\end{Soutput}
\begin{Sinput}
> yacas(expression(n))
\end{Sinput}
\begin{Soutput}
expression(n)
\end{Soutput}
\end{Schunk}

\begin{Schunk}
\begin{Sinput}
> Set(ns, 1)
\end{Sinput}
\begin{Soutput}
expression(1)
\end{Soutput}
\begin{Sinput}
> ns <- Sym("ns")
> ns
\end{Sinput}
\begin{Soutput}
expression(1)
\end{Soutput}
\begin{Sinput}
> Clear(ns)
\end{Sinput}
\begin{Soutput}
expression(TRUE)
\end{Soutput}
\begin{Sinput}
> ns
\end{Sinput}
\begin{Soutput}
expression(ns)
\end{Soutput}
\end{Schunk}


\subsection{Symbolic and numerical evaluations, precision}

Evaluations are generally exact:
\begin{Schunk}
\begin{Sinput}
> yacas("Exp(0)")
\end{Sinput}
\begin{Soutput}
expression(1)
\end{Soutput}
\begin{Sinput}
> yacas("Exp(1)")
\end{Sinput}
\begin{Soutput}
expression(exp(1))
\end{Soutput}
\begin{Sinput}
> yacas("Sin(Pi/4)")
\end{Sinput}
\begin{Soutput}
expression(root(1/2, 2))
\end{Soutput}
\begin{Sinput}
> yacas("355/113")
\end{Sinput}
\begin{Soutput}
expression(355/113)
\end{Soutput}
\end{Schunk}

\begin{Schunk}
\begin{Sinput}
> exp(Sym(0))
\end{Sinput}
\begin{Soutput}
expression(1)
\end{Soutput}
\begin{Sinput}
> exp(Sym(1))
\end{Sinput}
\begin{Soutput}
expression(exp(1))
\end{Soutput}
\begin{Sinput}
> sin(Pi/4)
\end{Sinput}
\begin{Soutput}
expression(root(1/2, 2))
\end{Soutput}
\begin{Sinput}
> Sym("355/113")
\end{Sinput}
\begin{Soutput}
expression(355/113)
\end{Soutput}
\end{Schunk}

 To obtain a numerical evaluation
(approximation), the \code{N()} function can be used:
\begin{Schunk}
\begin{Sinput}
> yacas("N(Exp(1))")
\end{Sinput}
\begin{Soutput}
expression(2.7182818284)
\end{Soutput}
\begin{Sinput}
> yacas("N(Sin(Pi/4))")
\end{Sinput}
\begin{Soutput}
expression(0.70710678118)
\end{Soutput}
\begin{Sinput}
> yacas("N(355/113)")
\end{Sinput}
\begin{Soutput}
expression(3.1415929203)
\end{Soutput}
\end{Schunk}

\begin{Schunk}
\begin{Sinput}
> N(exp(1))
\end{Sinput}
\begin{Soutput}
expression(2.71828182845905)
\end{Soutput}
\begin{Sinput}
> N(sin(Pi/4))
\end{Sinput}
\begin{Soutput}
expression(0.70710678118)
\end{Soutput}
\begin{Sinput}
> N(355/113)
\end{Sinput}
\begin{Soutput}
expression(3.14159292035398)
\end{Soutput}
\end{Schunk}

The \code{N()} function has an optional second argument, the required precision:
\begin{Schunk}
\begin{Sinput}
> yacas("N(355/133,20)")
\end{Sinput}
\begin{Soutput}
expression(2.66917293233083)
\end{Soutput}
\end{Schunk}

\begin{Schunk}
\begin{Sinput}
> N("355/113",20) 
\end{Sinput}
\begin{Soutput}
expression(3.14159292035398)
\end{Soutput}
\end{Schunk}


The command \code{Precision(n)}
can be used to specify that all floating point numbers should have a
fixed precision of n digits:
\begin{Schunk}
\begin{Sinput}
> yacas("Precision(5)")
\end{Sinput}
\begin{Soutput}
expression(TRUE)
\end{Soutput}
\begin{Sinput}
> yacas("N(355/113)")
\end{Sinput}
\begin{Soutput}
expression(3.14159)
\end{Soutput}
\end{Schunk}

\begin{Schunk}
\begin{Sinput}
> Precision(5)
\end{Sinput}
\begin{Soutput}
expression(TRUE)
\end{Soutput}
\begin{Sinput}
> N("355/113")
\end{Sinput}
\begin{Soutput}
expression(3.14159)
\end{Soutput}
\end{Schunk}


\subsection{Rational numbers}

Rational numbers will stay rational as long as the numerator and
denominator are integers:
\begin{Schunk}
\begin{Sinput}
> yacas(expression(55/10))
\end{Sinput}
\begin{Soutput}
expression(11/2)
\end{Soutput}
\end{Schunk}

\begin{Schunk}
\begin{Sinput}
> Sym("55 / 10")
\end{Sinput}
\begin{Soutput}
expression(11/2)
\end{Soutput}
\end{Schunk}


\subsection{Symbolic calculation}
\label{sec:symbolicCalculation}

Some exact manipulations :
\begin{Schunk}
\begin{Sinput}
> yacas("1/14+5/21*(30-(1+1/2)*5^2)")
\end{Sinput}
\begin{Soutput}
expression(-12/7)
\end{Soutput}
\begin{Sinput}
> yacas("0+x")
\end{Sinput}
\begin{Soutput}
expression(x)
\end{Soutput}
\begin{Sinput}
> yacas("x+1*y")
\end{Sinput}
\begin{Soutput}
expression(x + y)
\end{Soutput}
\begin{Sinput}
> yacas("Sin(ArcSin(alpha))+Tan(ArcTan(beta))")
\end{Sinput}
\begin{Soutput}
expression(alpha + beta)
\end{Soutput}
\end{Schunk}

\begin{Schunk}
\begin{Sinput}
> Sym("1/14+5/21*(1*30-(1+1/2)*5^2)")
\end{Sinput}
\begin{Soutput}
expression(-12/7)
\end{Soutput}
\begin{Sinput}
> xs <- Sym("xs")
> ys <- Sym("ys")
> 0+xs
\end{Sinput}
\begin{Soutput}
expression(xs)
\end{Soutput}
\begin{Sinput}
> xs+1*ys
\end{Sinput}
\begin{Soutput}
expression(xs + ys)
\end{Soutput}
\begin{Sinput}
> sin(asin(xs))+tan(atan(ys))
\end{Sinput}
\begin{Soutput}
expression(xs + ys)
\end{Soutput}
\end{Schunk}

\subsection{Complex numbers and the imaginary unit} 

The imaginary unit $i$ is denoted I and complex numbers can be entered
as either expressions involving I or explicitly Complex(a,b) for a+ib.
\begin{Schunk}
\begin{Sinput}
> yacas("I^2")
\end{Sinput}
\begin{Soutput}
expression(-1)
\end{Soutput}
\begin{Sinput}
> yacas("7+3*I")
\end{Sinput}
\begin{Soutput}
expression(complex_cartesian(7, 3))
\end{Soutput}
\begin{Sinput}
> yacas("Conjugate(%)")
\end{Sinput}
\begin{Soutput}
expression(complex_cartesian(7, -3))
\end{Soutput}
\begin{Sinput}
> yacas("Exp(3*I)")
\end{Sinput}
\begin{Soutput}
expression(complex_cartesian(cos(3), sin(3)))
\end{Soutput}
\end{Schunk}

\begin{Schunk}
\begin{Sinput}
> I^2
\end{Sinput}
\begin{Soutput}
expression(-1)
\end{Soutput}
\begin{Sinput}
> 7+3*I
\end{Sinput}
\begin{Soutput}
expression(complex_cartesian(7, 3))
\end{Soutput}
\begin{Sinput}
> Conjugate(7+3*I)
\end{Sinput}
\begin{Soutput}
expression(complex_cartesian(7, -3))
\end{Soutput}
\begin{Sinput}
> exp(3*I)
\end{Sinput}
\begin{Soutput}
expression(complex_cartesian(cos(3), sin(3)))
\end{Soutput}
\end{Schunk}


\subsection{Recall the most recent line -- the \texttt{\%} operator}

The operator \texttt{\%} automatically recalls the result from the
previous line. 
\begin{Schunk}
\begin{Sinput}
> yacas("(1+x)^3")
\end{Sinput}
\begin{Soutput}
expression((x + 1)^3)
\end{Soutput}
\begin{Sinput}
> yacas("%")
\end{Sinput}
\begin{Soutput}
expression((x + 1)^3)
\end{Soutput}
\begin{Sinput}
> yacas("z:= %")
\end{Sinput}
\begin{Soutput}
expression((x + 1)^3)
\end{Soutput}
\end{Schunk}

\begin{Schunk}
\begin{Sinput}
> (1+x)^3
\end{Sinput}
\begin{Soutput}
expression((xs + 1)^3)
\end{Soutput}
\begin{Sinput}
> zs <- Sym("%")
> zs
\end{Sinput}
\begin{Soutput}
expression((xs + 1)^3)
\end{Soutput}
\end{Schunk}



\subsection{Printing with PrettyForm, PrettyPrint, TexForm and  TeXForm} 
\label{sec:printing}

There are different ways of displaying the output. 


\subsubsection{Standard form}
The (standard)
yacas form is:
\begin{Schunk}
\begin{Sinput}
> yacas("A:={{a,b},{c,d}}")
\end{Sinput}
\begin{Soutput}
expression(list(list(a, b), list(c, d)))
\end{Soutput}
\begin{Sinput}
> yacas("B:= (1+x)^2+k^3")
\end{Sinput}
\begin{Soutput}
expression((x + 1)^2 + k^3)
\end{Soutput}
\begin{Sinput}
> yacas("A")
\end{Sinput}
\begin{Soutput}
expression(list(list(a, b), list(c, d)))
\end{Soutput}
\begin{Sinput}
> yacas("B")
\end{Sinput}
\begin{Soutput}
expression((x + 1)^2 + k^3)
\end{Soutput}
\end{Schunk}

\begin{Schunk}
\begin{Sinput}
> as <- Sym("as"); bs <- Sym("bs"); cs <- Sym("cs"); ds <- Sym("ds")
> A <- List(List(as,bs), List(cs,ds))
> ks <- Sym("ks")
> B <- (1+xs)^2+ks^3
> A
\end{Sinput}
\begin{Soutput}
expression(list(list(as, bs), list(cs, ds)))
\end{Soutput}
\begin{Sinput}
> B
\end{Sinput}
\begin{Soutput}
expression((xs + 1)^2 + ks^3)
\end{Soutput}
\end{Schunk}


\subsubsection{Pretty form}
The Pretty form is:
\begin{Schunk}
\begin{Sinput}
> yacas("PrettyForm(A)")
\end{Sinput}
\begin{Soutput}
/              \
| ( a ) ( b )  |
|              |
| ( c ) ( d )  |
\              /
\end{Soutput}
\begin{Sinput}
> yacas("PrettyForm(B)")
\end{Sinput}
\begin{Soutput}
         2    3
( x + 1 )  + k 
\end{Soutput}
\end{Schunk}

\begin{Schunk}
\begin{Sinput}
> PrettyForm(A)
\end{Sinput}
\begin{Soutput}
/                \
| ( as ) ( bs )  |
|                |
| ( cs ) ( ds )  |
\                /
\end{Soutput}
\begin{Sinput}
> PrettyForm(B)
\end{Sinput}
\begin{Soutput}
          2     3
( xs + 1 )  + ks 
\end{Soutput}
\end{Schunk}



\subsubsection{TeX form}

The output can be displayed in TeX form:
\begin{Schunk}
\begin{Sinput}
> yacas("TeXForm(B)", retclass = "character")
\end{Sinput}
\begin{Soutput}
'$\left( x + 1\right)  ^{2} + k ^{3}$'
\end{Soutput}
\end{Schunk}

\section{Commands} 
\label{sec:commands}


\subsection{Factorial}

\begin{Schunk}
\begin{Sinput}
> yacas("40!")
\end{Sinput}
\begin{Soutput}
expression(8.15915283247898e+47)
\end{Soutput}
\begin{Sinput}
> yacas("40!", retclass = "character")
\end{Sinput}
\begin{Soutput}
815915283247897734345611269596115894272000000000 
\end{Soutput}
\end{Schunk}

\begin{Schunk}
\begin{Sinput}
> Factorial(40)
\end{Sinput}
\begin{Soutput}
expression(8.15915283247898e+47)
\end{Soutput}
\end{Schunk}

\subsection{Taylor expansions} 

Expand $\exp(x)$ in three terms
around $0$ and $a$:
\begin{Schunk}
\begin{Sinput}
> yacas("Taylor(x,0,3) Exp(x)")
\end{Sinput}
\begin{Soutput}
expression(x + x^2/2 + x^3/6 + 1)
\end{Soutput}
\begin{Sinput}
> yacas("Taylor(x,a,3) Exp(x)")
\end{Sinput}
\begin{Soutput}
expression(exp(a) + exp(a) * (x - a) + (x - a)^2 * exp(a)/2 + 
    (x - a)^3 * exp(a)/6)
\end{Soutput}
\end{Schunk}

\begin{Schunk}
\begin{Sinput}
> xs <- Sym("xs")
> Taylor(exp(xs),xs,0,3)
\end{Sinput}
\begin{Soutput}
expression(xs + xs^2/2 + xs^3/6 + 1)
\end{Soutput}
\begin{Sinput}
> as <- Sym("as")
> Taylor(exp(x),x,as,3)
\end{Sinput}
\begin{Soutput}
expression(exp(as) + exp(as) * (xs - as) + (xs - as)^2 * exp(as)/2 + 
    (xs - as)^3 * exp(as)/6)
\end{Soutput}
\end{Schunk}

The \code{InverseTaylor()} function builds the Taylor series expansion of the
inverse of an expression. For example, the Taylor expansion in two
terms of the inverse of $\exp(x)$ around $x=0$ (which is the Taylor
expansion of $Ln(y)$ around $y=1$):
\begin{Schunk}
\begin{Sinput}
> yacas("InverseTaylor(x,0,2)Exp(x)")
\end{Sinput}
\begin{Soutput}
expression(x - 1 - (x - 1)^2/2)
\end{Soutput}
\begin{Sinput}
> yacas("Taylor(y,1,2)Ln(y)")
\end{Sinput}
\begin{Soutput}
expression(y - 1 - (y - 1)^2/2)
\end{Soutput}
\end{Schunk}

\begin{Schunk}
\begin{Sinput}
> InverseTaylor(exp(xs),xs,0,2)
\end{Sinput}
\begin{Soutput}
expression(xs + xs^2/2 + 1)
\end{Soutput}
\begin{Sinput}
> Taylor(log(ys),ys,1,2)
\end{Sinput}
\begin{Soutput}
expression(ys - 1 - (ys - 1)^2/2)
\end{Soutput}
\end{Schunk}

\subsection{Solving equations}


\subsubsection{Solving equations symbolically}

Solve equations symbolically with the \code{Solve()} function:
\begin{Schunk}
\begin{Sinput}
> yacas("Solve(x/(1+x) == a, x)")
\end{Sinput}
\begin{Soutput}
expression(list(x == a/(1 - a)))
\end{Soutput}
\begin{Sinput}
> yacas("Solve(x^2+x == 0, x)")
\end{Sinput}
\begin{Soutput}
expression(list(x == 0, x == -1))
\end{Soutput}
\end{Schunk}

\begin{Schunk}
\begin{Sinput}
> Solve(xs/(1+xs) == as, xs)
\end{Sinput}
\begin{Soutput}
expression(list(xs == as/(1 - as)))
\end{Soutput}
\begin{Sinput}
> Solve(xs^2+xs == 0, xs)
\end{Sinput}
\begin{Soutput}
expression(list(xs == 0, xs == -1))
\end{Soutput}
\end{Schunk}

\begin{Schunk}
\begin{Sinput}
> Solve(List(xs^2+ys^2==6, xs-ys==3), List(xs,ys))
\end{Sinput}
\begin{Soutput}
expression(list(list(xs == root(6 - ys^2, 2), ys == ys)))
\end{Soutput}
\end{Schunk}

\begin{Schunk}
\begin{Sinput}
> mu <- Sym("mu") # mean
> v <- Sym("v") # variance
> Solve(List(mu==(xs/(xs+ys)), v==((xs*ys)/(((xs+ys)^2) * (xs+ys+1)))),
+     List(xs,ys))
\end{Sinput}
\begin{Soutput}
expression(list(list(xs == mu^2 * (1 - mu)/v - mu, ys == xs/mu - 
    xs)))
\end{Soutput}
\end{Schunk}

(Note the use of the == operator, which does not evaluate to anything,
to denote an "equation" object.) 

\subsubsection{Solving equations numerically}
To solve an equation (in one variable) like $sin(x)-exp(x)=0$ numerically taking $0.5$
as initial guess and an accuracy of $0.0001$ do:
\begin{Schunk}
\begin{Sinput}
> yacas("Newton(Sin(x)-Exp(x),x, 0.5, 0.0001)")
\end{Sinput}
\begin{Soutput}
expression(-3.18306)
\end{Soutput}
\end{Schunk}

\begin{Schunk}
\begin{Sinput}
> Newton(sin(xs)-exp(xs),xs, 0.5, 0.0001) 
\end{Sinput}
\begin{Soutput}
expression(-3.18306)
\end{Soutput}
\end{Schunk}

\subsection{Expanding polynomials} 
\begin{Schunk}
\begin{Sinput}
> yacas(expression(Expand((1+x)^3)))
\end{Sinput}
\begin{Soutput}
expression(x^3 + 3 * x^2 + 3 * x + 1)
\end{Soutput}
\end{Schunk}

\begin{Schunk}
\begin{Sinput}
> Expand((x+1)^3)
\end{Sinput}
\begin{Soutput}
expression(xs^3 + 3 * xs^2 + 3 * xs + 1)
\end{Soutput}
\end{Schunk}



\subsection{Simplifying an expression}

The function Simplify() attempts to reduce an expression
to a simpler form. 
\begin{Schunk}
\begin{Sinput}
> y <- Sym("y")
> yacas("(x+y)^3-(x-y)^3")
\end{Sinput}
\begin{Soutput}
expression((x + y)^3 - (x - y)^3)
\end{Soutput}
\begin{Sinput}
> yacas("Simplify(%)")
\end{Sinput}
\begin{Soutput}
expression(6 * (x^2 * y) + 2 * y^3)
\end{Soutput}
\begin{Sinput}
> 
\end{Sinput}
\end{Schunk}

\begin{Schunk}
\begin{Sinput}
> (x+y)^3-(x-y)^3
\end{Sinput}
\begin{Soutput}
expression((xs + y)^3 - (xs - y)^3)
\end{Soutput}
\begin{Sinput}
> Simplify("%")
\end{Sinput}
\begin{Soutput}
expression(6 * (xs^2 * y) + 2 * y^3)
\end{Soutput}
\end{Schunk}


\subsection{Analytical derivatives}

Analytical derivatives of functions can be evaluated with the
\code{D()} and \code{deriv()} functions:
\begin{Schunk}
\begin{Sinput}
> yacas("D(x) Sin(x)")
\end{Sinput}
\begin{Soutput}
expression(cos(x))
\end{Soutput}
\end{Schunk}

\begin{Schunk}
\begin{Sinput}
> deriv(sin(xs), xs)
\end{Sinput}
\begin{Soutput}
expression(cos(xs))
\end{Soutput}
\end{Schunk}

These functions also accepts an argument specifying how often the
derivative has to be taken, e.g:
\begin{Schunk}
\begin{Sinput}
> yacas("D(x,4)Sin(x)")
\end{Sinput}
\begin{Soutput}
expression(sin(x))
\end{Soutput}
\end{Schunk}

\begin{Schunk}
\begin{Sinput}
> deriv(sin(xs),xs,4)
\end{Sinput}
\begin{Soutput}
expression(sin(xs))
\end{Soutput}
\end{Schunk}

\subsection{Integration}


\begin{Schunk}
\begin{Sinput}
> #yacas("Integrate(x,a,b)Sin(x)")
> #yacas("Integrate(x,a,b)Ln(x)+x")
> #yacas("Integrate(x)1/(x^2-1)")
> yacas("Integrate(x)Sin(a*x)^2*Cos(b*x)")
\end{Sinput}
\begin{Soutput}
expression((2 * sin(b * x)/b - (sin(-2 * x * a - b * x)/(-2 * 
    a - b) + sin(-2 * x * a + b * x)/(-2 * a + b)))/4)
\end{Soutput}
\end{Schunk}

\begin{Schunk}
\begin{Sinput}
> #Integrate(sin(x),x,a,b)
> #Integrate(log(x),x,a,b)
> #Integrate(1/(x^2-1),x)
> a <- Sym("a")
> b <- Sym("b")
> Integrate(sin(a*x)^2*cos(b*x),x)
\end{Sinput}
\begin{Soutput}
expression((2 * sin(b * xs)/b - (sin(-2 * xs * a - b * xs)/(-2 * 
    a - b) + sin(-2 * xs * a + b * xs)/(-2 * a + b)))/4)
\end{Soutput}
\end{Schunk}


\subsection{Limits}
\begin{Schunk}
\begin{Sinput}
> #yacas("Limit(x,0)Sin(x)/x")
> yacas("Limit(n,Infinity)(1+(1/n))^n")
\end{Sinput}
\begin{Soutput}
expression(exp(1))
\end{Soutput}
\begin{Sinput}
> yacas("Limit(h,0) (Sin(x+h)-Sin(x))/h")
\end{Sinput}
\begin{Soutput}
expression(cos(x))
\end{Soutput}
\end{Schunk}

\begin{Schunk}
\begin{Sinput}
> #Limit(sin(x)/x,x,0)
> ns <- Sym("ns")
> Limit((1+(1/ns))^ns,ns,Infinity)
\end{Sinput}
\begin{Soutput}
expression(exp(1))
\end{Soutput}
\begin{Sinput}
> hs <- Sym("hs")
> Limit((sin(xs+hs)-sin(xs))/hs,hs,0)
\end{Sinput}
\begin{Soutput}
expression(cos(xs))
\end{Soutput}
\end{Schunk}

\subsection{Variable substitution}

\begin{Schunk}
\begin{Sinput}
> yacas("Subst(x,Cos(a))x+x")
\end{Sinput}
\begin{Soutput}
expression(2 * cos(a))
\end{Soutput}
\end{Schunk}

\begin{Schunk}
\begin{Sinput}
> Subst(xs+xs,xs,cos(as))
\end{Sinput}
\begin{Soutput}
expression(2 * cos(as))
\end{Soutput}
\end{Schunk}

\subsection{Solving ordinary differential equations}

\begin{Schunk}
\begin{Sinput}
> yacas("OdeSolve(y''==4*y)")
\end{Sinput}
\begin{Soutput}
expression(C1586 * exp(2 * x) + C1590 * exp(-2 * x))
\end{Soutput}
\begin{Sinput}
> yacas("OdeSolve(y'==8*y)")
\end{Sinput}
\begin{Soutput}
expression(C1620 * exp(8 * x))
\end{Soutput}
\end{Schunk}








\section{Matrices}
\label{sec:matrices}
\begin{Schunk}
\begin{Sinput}
> yacas("E4:={ {u1,u1,0},{u1,0,u2},{0,u2,0} }")
\end{Sinput}
\begin{Soutput}
expression(list(list(u1, u1, 0), list(u1, 0, u2), list(0, u2, 
    0)))
\end{Soutput}
\begin{Sinput}
> yacas("PrettyForm(E4)")
\end{Sinput}
\begin{Soutput}
/                       \
| ( u1 ) ( u1 ) ( 0 )   |
|                       |
| ( u1 ) ( 0 )  ( u2 )  |
|                       |
| ( 0 )  ( u2 ) ( 0 )   |
\                       /
\end{Soutput}
\end{Schunk}

\begin{Schunk}
\begin{Sinput}
> u1 <- Sym("u1")
> u2 <- Sym("u2")
> E4 <- List(List(u1, u1, 0), List(u1, 0, u2), List(0, u2, 0))
> PrettyForm(E4)
\end{Sinput}
\begin{Soutput}
/                       \
| ( u1 ) ( u1 ) ( 0 )   |
|                       |
| ( u1 ) ( 0 )  ( u2 )  |
|                       |
| ( 0 )  ( u2 ) ( 0 )   |
\                       /
\end{Soutput}
\end{Schunk}


\subsection{Inverse} 

\begin{Schunk}
\begin{Sinput}
> yacas("E4i:=Inverse(E4)")
\end{Sinput}
\begin{Soutput}
expression(list(list(u2^2/(u1 * u2^2), 0, -(u1 * u2)/(u1 * u2^2)), 
    list(0, 0, u1 * u2/(u1 * u2^2)), list(-(u1 * u2)/(u1 * u2^2), 
        u1 * u2/(u1 * u2^2), u1^2/(u1 * u2^2))))
\end{Soutput}
\begin{Sinput}
> yacas("Simplify(E4i)")
\end{Sinput}
\begin{Soutput}
expression(list(list(1/u1, 0, -1/u2), list(0, 0, 1/u2), list(-1/u2, 
    1/u2, u1/u2^2)))
\end{Soutput}
\begin{Sinput}
> yacas("PrettyForm(Simplify(E4i))")
\end{Sinput}
\begin{Soutput}
/                        \
| / 1  \ ( 0 )  / -1 \   |
| | -- |        | -- |   |
| \ u1 /        \ u2 /   |
|                        |
| ( 0 )  ( 0 )  / 1  \   |
|               | -- |   |
|               \ u2 /   |
|                        |
| / -1 \ / 1  \ / u1  \  |
| | -- | | -- | | --- |  |
| \ u2 / \ u2 / |   2 |  |
|               \ u2  /  |
\                        /
\end{Soutput}
\end{Schunk}


\begin{Schunk}
\begin{Sinput}
> E4i <- Inverse(E4)
> Simplify(E4i)
\end{Sinput}
\begin{Soutput}
expression(list(list(1/u1, 0, -1/u2), list(0, 0, 1/u2), list(-1/u2, 
    1/u2, u1/u2^2)))
\end{Soutput}
\begin{Sinput}
> PrettyForm(Simplify(E4i))
\end{Sinput}
\begin{Soutput}
/                        \
| / 1  \ ( 0 )  / -1 \   |
| | -- |        | -- |   |
| \ u1 /        \ u2 /   |
|                        |
| ( 0 )  ( 0 )  / 1  \   |
|               | -- |   |
|               \ u2 /   |
|                        |
| / -1 \ / 1  \ / u1  \  |
| | -- | | -- | | --- |  |
| \ u2 / \ u2 / |   2 |  |
|               \ u2  /  |
\                        /
\end{Soutput}
\end{Schunk}


\subsection{Determinant}

\begin{Schunk}
\begin{Sinput}
> yacas("Determinant(E4)")
\end{Sinput}
\begin{Soutput}
expression(-(u1 * u2^2))
\end{Soutput}
\begin{Sinput}
> yacas("Determinant(E4i)")
\end{Sinput}
\begin{Soutput}
expression(-(u1 * u2 * (u1 * u2^3))/(u1 * u2^2)^3)
\end{Soutput}
\begin{Sinput}
> yacas("Simplify(E4i)")
\end{Sinput}
\begin{Soutput}
expression(list(list(1/u1, 0, -1/u2), list(0, 0, 1/u2), list(-1/u2, 
    1/u2, u1/u2^2)))
\end{Soutput}
\begin{Sinput}
> yacas("Simplify(Determinant(E4i))")
\end{Sinput}
\begin{Soutput}
expression(-1/(u1 * u2^2))
\end{Soutput}
\end{Schunk}


\begin{Schunk}
\begin{Sinput}
> determinant(E4)
\end{Sinput}
\begin{Soutput}
expression(-(u1 * u2^2))
\end{Soutput}
\begin{Sinput}
> determinant(E4i)
\end{Sinput}
\begin{Soutput}
expression(-(u1 * u2 * (u1 * u2^3))/(u1 * u2^2)^3)
\end{Soutput}
\begin{Sinput}
> Simplify(E4i)
\end{Sinput}
\begin{Soutput}
expression(list(list(1/u1, 0, -1/u2), list(0, 0, 1/u2), list(-1/u2, 
    1/u2, u1/u2^2)))
\end{Soutput}
\begin{Sinput}
> Simplify(determinant(E4i))
\end{Sinput}
\begin{Soutput}
expression(-1/(u1 * u2^2))
\end{Soutput}
\end{Schunk}



%%% -------------------------------------------------------------
\section{Miscellaneous} 
\label{sec:misc}



Note that the value returned by \yacas\ can be of different types:
\begin{Schunk}
\begin{Sinput}
> yacas(expression(Factor(x^2-1)),retclass='unquote')
\end{Sinput}
\begin{Soutput}
*" ("+" (x ,1 ),"- (x ,1 ))
\end{Soutput}
\begin{Sinput}
> yacas(expression(Factor(x^2-1)),retclass='character')
\end{Sinput}
\begin{Soutput}
"*" ("+" (x ,1 ),"-" (x ,1 ))
\end{Soutput}
\end{Schunk}


\end{document}

